\documentclass[12pt]{article}

\usepackage{amsmath}
\usepackage{unicode-math}
\usepackage{xltxtra}
\usepackage{xgreek}

\setmainfont{Liberation Serif}

\usepackage{tabularx}

\pagestyle{empty}

\usepackage{geometry}
 \geometry{a4paper, total={190mm,280mm}, left=10mm, top=10mm}

 \usepackage{graphicx}
 \graphicspath{ {images/} }

 \usepackage{wrapfig}

\begin{document}

\begin{table}
  \small
  \begin{tabularx}{\textwidth}{ c X r }
    \begin{tabular}{ l }
      Εισηγητής: Λόλας Κωνσταντίνος \\
      Επαναληπτικό: Πράξεις, Διάταξη, Απόλυτες, Ρίζες
    \end{tabular}
     &  &
    \begin{tabular}{ r }
      Θεσσαλονίκη, 05 / 12 / 2023
    \end{tabular}
  \end{tabularx}
\end{table}

\part*{\centering{Διαγώνισμα Άλγεβρα Α Λυκείου}}

\section*{Θέμα Α}
\noindent
\begin{enumerate}
  \item \textbf{[Μονάδες 15]} Να αποδείξετε ότι για κάθε $α\ge 0$, $β\ge 0$, ισχύει $| α\cdot β | =|α|\cdot |β|$.
  \item \textbf{[Μονάδες 10]} Να χαρακτηρίσετε τις παρακάτω προτάσεις με Σωστό ή Λάθος
        \begin{enumerate}
          \item [α)] $\sqrt{x+y}=\sqrt{x}+\sqrt{y}$ για κάθε $x$ και $y\in\mathbb{R}$.
          \item [β)] Αν $α^2+β^2\le 0 \Rightarrow α=β=0$.
          \item [γ)] Αν $α<β \Rightarrow α^2<β^2$ για κάθε $α$ και $β\in\mathbb{R}$.
          \item [δ)] $|-α|=|α|$ για κάθε $α\in\mathbb{R}$.
          \item [ε)] $|α|+|β|=|α+β|$ για κάθε $α$ και $β\in\mathbb{R}$.
        \end{enumerate}
\end{enumerate}

\section*{Θέμα Β}
\noindent
Έστω $Α=\sqrt{3}-1$
\begin{enumerate}
  \item \textbf{[Μονάδες 7]} Να υπολογίσετε την παράσταση $\dfrac{1}{Α}+\dfrac{1}{Α+2}$
  \item \textbf{[Μονάδες 6]} Να δείξετε ότι $Α^3=6\sqrt{3}-10$
  \item \textbf{[Μονάδες 5]} Να υπολογίσετε την $\sqrt[3]{6\sqrt{3}-10}$
  \item \textbf{[Μονάδες 7]} Να συγκρίνετε τους αριθμούς $\sqrt{109}$ και $6\sqrt{3}$
\end{enumerate}

\section*{Θέμα Γ}
\noindent
Έστω $|2x+5|<3$.
\begin{enumerate}
  \item \textbf{[Μονάδες 6]} Να δείξετε ότι $-4<x<-1$
  \item \textbf{[Μονάδες 6]} Να δώσετε γεωμετρική ερμηνεία της παράστασης $Α=|x+4|+|x+1|$
  \item \textbf{[Μονάδες 6]} Να αποδείξετε ότι $Α=3$
\end{enumerate}

\section*{Θέμα Δ}
\noindent
Έστω ότι $-1\le α<3$, $-2<β\le 2$ και $γ\in\mathbb{R}$
\begin{enumerate}
  \item \textbf{[Μονάδες 5]} Να δείξετε ότι $(α+1)(β-2)=αβ-2-2α+β$.
  \item \textbf{[Μονάδες 5]} Να δείξετε ότι $αβ-2\le 2α-β$.
  \item \textbf{[Μονάδες 5]} Να δείξετε ότι $-8\le 2α-3β \le 12$.
  \item \textbf{[Μονάδες 5]} Να δείξετε ότι $γ^2-6γ \ge -9$.
  \item \textbf{[Μονάδες 5]} Αν επιπλέον ισχύει $2α-αβ-β+γ^2-6γ+11=0$ να βρείτε τα $α$, $β$ και $γ$.
\end{enumerate}


\part*{\centering{Καλή επιτυχία}}

\end{document}
