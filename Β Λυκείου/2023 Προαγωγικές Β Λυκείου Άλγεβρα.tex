\documentclass[12pt]{extarticle}

\usepackage{amsmath}
\usepackage{unicode-math}
\usepackage{xltxtra}
\usepackage{xgreek}

\setmainfont{Liberation Serif}

\usepackage{tabularx}

\pagestyle{empty}

\usepackage{geometry}
\geometry{a4paper, total={190mm,275mm}, left=10mm, top=10mm}

\usepackage{graphicx}
\graphicspath{ {images/} }

\usepackage{wrapfig}

\begin{document}
\renewcommand{\labelenumi}{\alph{enumi})}
\renewcommand{\labelenumii}{\roman{enumii}.}

\begin{table}
    \small
    \begin{tabularx}{\textwidth}{ c X r }
        \begin{tabular}{ c }
            \includegraphics[scale=0.4]{ελληνική}         \\
            ΕΛΛΗΝΙΚΗ ΔΗΜΟΚΡΑΤΙΑ                           \\
            ΥΠΟΥΡΓΕΙΟ ΠΑΙΔΕΙΑΣ \& ΘΡΗΣΚΕΥΜΑΤΩΝ            \\
            ΠΕΡΙΦΕΡΕΙΑΚΗ Δ/ΝΣΗ Α/ΘΜΙΑΣ \& Β/ΘΜΙΑΣ ΕΚΠ/ΣΗΣ \\
            ΚΕΝΤΡΙΚΗΣ ΜΑΚΕΔΟΝΙΑΣ                          \\
            Δ/ΝΣΗ Β/ΘΜΙΑΣ ΕΚΠ/ΣΗΣ ΑΝ. ΘΕΣ/ΝΙΚΗΣ           \\
            10ο ΓΕΝΙΚΟ ΛΥΚΕΙΟ ΘΕΣ/ΝΙΚΗΣ
        \end{tabular}
         &  &
        \begin{tabular}{ r }
            Σχολικό Έτος: 2022 - 2023     \\
            Εξ. Περίοδος: Μαΐου - Ιουνίου \\
            Μάθημα: Άλγβρα Β Λυκείου      \\
            Εισηγητής: Κράντας , Λόλας    \\ \\
            Θεσσαλονίκη, 30 / 05 / 2023
        \end{tabular}
    \end{tabularx}
\end{table}

\part*{\centering{Θέματα}}
\section*{Θέμα 1}
\noindent

\begin{enumerate}
    \item Να αποδείξετε ότι αν $α>0$ με $α \ne 1$, τότε για οποιαδήποτε $θ_1$, $θ_2>0$, ισχύει $$\log_α θ_1+\log_α θ_2=\log_α(θ_1 θ_2)$$ \hspace*{\fill} \textbf{Μονάδες 15}

    \item Να χαρακτηρίσετε τις παρακάτω προτάσεις με Σωστό ή Λάθος
          \begin{enumerate}
              \item Για οποιουσδήποτε θετικούς αριθμούς $x_1$, $x_2$ ισχύει $\log\dfrac{x_1}{x_2}=\dfrac{\log x_1}{\log x_2}$
              \item Η συνάρτηση $f(x)=α^x$ με $0<α<1$ είναι γνησίως φθίνουσα
              \item Η γραφική παράσταση μιας περιττής συνάρτησης $f$ έχει άξονα συμμετρίας τον $y'y$
              \item Το μηδενικό πολυώνυμο έχει βαθμό 0
              \item Για κάθε $ω\in\mathbb{R}$ ισχύει $ημ^2ω+συν^2ω=1$\hspace*{\fill}\textbf{Μονάδες 10}
          \end{enumerate}
\end{enumerate}

\section*{Θέμα 2 (15047)}
\noindent
Δίνεται το πολυώνυμο $P(x)=x^4-x^3-5x^2+7x-2$
\begin{enumerate}
    \item[α)] Να αποδείξετε ότι ο αριθμός $1$ είναι ρίζα του πολυωνύμου \hspace*{\fill} \textbf{Μονάδες 10}
    \item[β)] Να εξετάσετε αν το πολυώνυμο έχει και άλλη ακέραια ρίζα \hspace*{\fill} \textbf{Μονάδες 15}
\end{enumerate}

\section*{Θέμα 3}
\noindent

Δίνεται η συνάρτηση $f(x)=\ln \dfrac{1+x}{1-x}$
\begin{enumerate}
    \item[α)] Να βρεθεί το πεδίο ορισμού της $f$. \hspace*{\fill} \textbf{Μονάδες 9}
    \item[β)] Να λυθεί η εξίσωση $f(x)=0$ \hspace*{\fill} \textbf{Μονάδες 8}
    \item[γ)] Να λυθεί η ανίσωση $f(x)>0$. \hspace*{\fill} \textbf{Μονάδες 8}
\end{enumerate}

\section*{Θέμα 4 (20943)}
\noindent

Δίνεται η γωνία $x$ με $\dfrac{3π}{2}<x<2π$ και οι παραστάσεις:

$$Α=ημ^2\left( π-x \right)+ημ^2\left( π+x \right)+συν^2\left( -x \right)$$

$$Β=\dfrac{ημx}{1+συνx}+\dfrac{1+συνx}{ημx}$$
\begin{enumerate}
    \item[α)] Να αποδείξετε ότι $Α=ημ^2x+1$ \hspace*{\fill} \textbf{Μονάδες 08}

    \item[β)] Να απλοποιήσετε την παράσταση $Β$ \hspace*{\fill} \textbf{Μονάδες 08}
    \item[γ)] Να εξετάσετε αν υπάρχει γωνία $x$ για την οποία οι παραστάσεις $Α$ και $Β$ να είναι ίσες \hspace*{\fill} \textbf{Μονάδες 09}
\end{enumerate}

\begin{table}[htb]
    \begin{tabularx}{\textwidth}{ X c X c X}
         &
        \begin{tabular}[t]{ c }
            Ο Δ/ντης \\ \\ \\ \\
            Παπαδημητρίου Χρήστος
        \end{tabular}
         &   &
        \begin{tabular}[t]{ c }
            Οι εισηγητές       \\ \\ \\ \\
            Λόλας Κωνσταντίνος \\ \\ \\ \\
            Κράντας Στυλιανός
        \end{tabular}
         &
    \end{tabularx}
\end{table}
\end{document}