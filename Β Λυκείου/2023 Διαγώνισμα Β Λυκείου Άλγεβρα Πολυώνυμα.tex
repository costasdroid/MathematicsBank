\documentclass[12pt]{article}

\usepackage{amsmath}
\usepackage{unicode-math}
\usepackage{xltxtra}
\usepackage{xgreek}

\setmainfont{Liberation Serif}

\usepackage{tabularx}

\pagestyle{empty}

\usepackage{geometry}
 \geometry{a4paper, total={190mm,280mm}, left=10mm, top=10mm}

\usepackage{graphicx}
\graphicspath{ {images/} }

\usepackage{wrapfig}

\begin{document}
\part*{\centering{Θέματα για ομορφόπαιδα (τα υπόλοιπα)}}

\section*{Θέμα Α}
Δίνεται ο κύκλος $C$ που έχει κέντρο την αρχή των αξόνων και διέρχεται από το σημείο $Α(-3,4)$. Να βρείτε
\begin{enumerate}
  \item την εξίσωση του κύκλου
  \item την εφαπτομένη του κύκλου στο σημείο $Α$
  \item την εξίσωση της χορδής του κύκλου που έχει μέσον το σημείο $Β(1,-2)$
  \item τη σχετική θέση της ευθείας $ε:y=x-10$ ως προς τον κύκλο και μετά, τη μέγιστη και ελάχιστη απόσταση ενός σημείου του κύκλου $C$ από την ευθεία $ε$
  \item το γεωμετρικό τόπο των σημείων $Μ$, από τα οποία οι εφαπτόμενες προς τον κύκλο είναι κάθετες
\end{enumerate}

\section*{Θέμα Β}
Δίνεται η εξίσωση: $x^2+y^2+λ(x-y+2)=2$, $λ\in\mathbb{R}$
\begin{enumerate}
  \item Να βρείτε τις τιμές του $λ$, ώστε η εξίσωση να παριστάνει κύκλο
  \item Για $λ\ne 2$
        \begin{enumerate}
          \item έχουν τα κέντρα τους σε μια ευθεία $ε$
          \item διέρχονται από σταθερό σημείο
          \item εφάπτονται στην ευθεία $η: x-y+2=0$
          \item εφάπτονται μεταξύ τους
        \end{enumerate}
\end{enumerate}

\end{document}
