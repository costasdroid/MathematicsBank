\documentclass[12pt]{article}

\usepackage{amsmath}
\usepackage{unicode-math}
\usepackage{xltxtra}
\usepackage{xgreek}

\setmainfont{Liberation Serif}

\usepackage{tabularx}

\pagestyle{empty}

\usepackage{geometry}
 \geometry{a4paper, total={190mm,280mm}, left=10mm, top=10mm}

 \usepackage{graphicx}
 \graphicspath{ {images/} }

 \usepackage{wrapfig}
\usepackage{lipsum}%% a garbage package you don't need except to create examples.

\begin{document}

\part*{\centering{Λύσεις}}

\section*{Θέμα Α}
  \noindent
  \begin{enumerate}
    \item Απόδειξη από το βιβλίο.
    \item Ορισμός από το βιβλίο.
    \item Λ, Λ, Σ, Σ, Σ
  \end{enumerate}

\section*{Θέμα Β}
  \noindent
  Δίνονται τα διανύσματα $\vec{α}=(1,2)$ και $\vec{β}=(-2,κ)$ και το σημείο $Δ=(2,1)$
  \begin{enumerate}
    \item $\overrightarrow{ΓΔ}=ΟΔ-ΟΓ \Rightarrow ΟΓ=ΟΔ-\overrightarrow{ΓΔ}=(2,1)-(1,2)=(1,-1)$
    \item Με ορίζουσα ή με $λ$ ή με το μάτι είναι το $κ=-4$
    \item $-4+κ=0 \Rightarrow κ=4$

    Αν $κ=3$
    \item $συν\widehat{(\vec{α},\vec{β})}=\frac{\vec{α}\vec{β}}{|\vec{α}||\vec{β}|}=\frac{4}{\sqrt{65}}$.
    \item Με $κ$, $λ$ ή με το μάτι, $\vec{γ}=\vec{α}+\vec{β}$.

  \end{enumerate}

\section*{Θέμα Γ}
  \noindent
  Δίνονται τα σημεία $Ο=(0,0)$, $Α=(-5,1)$ και $Β=(2,2)$
  \begin{enumerate}
    \item \textbf{[Μονάδες 10]} Να δειχθεί ότι το $ΟΑΒ$ είναι τρίγωνο.
    \item \textbf{[Μονάδες 10]} Να δειχθεί ότι η εξίσωση της ευθείας $ΑΒ$ είναι η $7x-y=12$.
    \item \textbf{[Μονάδες 10]} Να βρεθεί η εξίσωση της μεσοκαθέτου του $ΑΒ$.
    \item \textbf{[Μονάδες 10]} Να δειχθεί ότι η εξίσωση της διχοτόμου της γωνίας $\widehat{ΑΟΒ}$ είναι η $3x+10y=0$.
    \item \textbf{[Μονάδες 10]} Να βρεθεί το εμβαδό του τριγώνου $ΟΑΒ$.

  \end{enumerate}

\section*{Θέμα Δ}
  \noindent
  Έστω η εξίσωση $ x^2+y^2+2λx+λy-15=0 $.
  \begin{enumerate}
    \item \textbf{[Μονάδες 5]} Να βρείτε τις τιμές του $λ$ ώστε η εξίσωση να παριστάνει κύκλο.
    \item \textbf{[Μονάδες 5]} Να βρείτε τον γεωμετρικό τόπο των κέντρων των κύκλων του προηγούμενου ερωτήματος.

    Για $λ=-2$,
    \item \textbf{[Μονάδες 10]} Να δείξετε ότι το σημείο $(6,4)$ είναι εξωτερικό του κύκλου.
    \item \textbf{[Μονάδες 10]} Να βρείτε τις εφαπτομένες του κύκλου που διέρχονται από το σημείο $(6,4)$.
    \item \textbf{[Μονάδες 10]} Να βρείτε την ελάχιστη απόσταση του σημείου $(6,4)$ από τον κύκλο.

  \end{enumerate}

\vspace{3\baselineskip}

\part*{\centering{Καλή επιτυχία}}
\begin{table}[htb]
    \begin{tabularx}{\textwidth}{ X c X c X}
      &
      \begin{tabular}[t]{ c }
        Ο Δ/ντης
        \\ \\ \\ \\ \\
        Παπαδημητρίου Χρήστος
      \end{tabular}
      & &
      \begin{tabular}[t]{ c }
        Ο εισηγητής
        \\ \\ \\ \\ \\
        Λόλας Κωνσταντίνος
      \end{tabular}
      &
    \end{tabularx}
\end{table}

\vspace*{\fill}
 \textbf{Οδηγίες}
 \begin{enumerate}
   \item Να απαντήσετε σε όλα τα θέματα
   \item Μην ξεχάσετε να γράψετε το ονοματεπώνυμό σας σε κάθε φύλλο που σας δώσουν.
   \item Όλες οι απαντήσεις να δωθούν στο φύλλο απαντήσεων. Οτιδήποτε γραφτεί στη σελίδα με τα θέματα δεν θα ληφθεί υπόψιν.
   \item Τα Σωστό - Λάθος δεν χρειάζονται αιτιολόγηση.
 \end{enumerate}
\end{document}
