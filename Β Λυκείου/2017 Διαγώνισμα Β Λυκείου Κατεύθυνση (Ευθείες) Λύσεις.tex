\documentclass[12pt]{article}

\usepackage{amsmath}
\usepackage{unicode-math}
\usepackage{xltxtra}
\usepackage{xgreek}

\setmainfont{Liberation Serif}

\usepackage{tabularx}

\pagestyle{empty}

\usepackage{geometry}
 \geometry{a4paper, total={190mm,280mm}, left=10mm, top=10mm}

 \usepackage{graphicx}
 \graphicspath{ {images/} }

 \usepackage{wrapfig}

\begin{document}

\begin{table}
    \small
    \begin{tabularx}{\textwidth}{ c X r }
      \begin{tabular}{ l }
        Εισηγητής: Λόλας Κωνσταντίνος \\
        Επαναληπτικό: (Ευθείες)
      \end{tabular}
      & &
      \begin{tabular}{ r }
        Θεσσαλονίκη, 16 / 03 / 2018
      \end{tabular}
    \end{tabularx}
\end{table}

\part*{\centering{Διαγώνισμα Κατεύθυνση Β Λυκείου}}

Δίνονται τα σημεία $Α(1,-2)$, $Β(-1,2)$ και η ευθεία $ε:y=-x+3$. Να βρείτε:
\begin{enumerate}
  \item τη παράλληλη της $ε$ που περνάει από το $Α$
  \item τη μεσοκάθετο του τμήματος $ΑΒ$
  \item το σημείο $Γ$ της ευθείας $ε$ που ισαπέχει από τα $Α$ και $Β$
  \item το σημείο $Δ$ της ευθείας $ε$, ώστε το τρίγωνο $ΑΔΒ$ να είναι ορθογώνιο στο $Δ$
  \item πού κινείται το σημείο $Ν$, όταν το σημείο $Ρ$ κινείται στην ευθεία $ε$ και ισχύει $\overrightarrow{ΑΝ}=2\overrightarrow{ΒΡ}$
  \item το εμβαδό του τριγώνου που σχηματίζεται από την $ε$ και τους άξονες
\end{enumerate}

\vspace{3\baselineskip}

\part*{\centering{Λύσεις}}

\begin{enumerate}
  \item $λ=-1$ άρα η ευθεία είναι η $y+2=-(x-1)$
  \item $λ_{ΑΒ}=\frac{4}{-2}=-2$ άρα η ζηττούμενη έχει $λ=\frac{1}{2}$. Το μέσο του $ΑΒ$ είναι το $(0,0)$ έτσι η ζητούμενη ευθεία είναι η $y=\frac{1}{2}x$
  \item Το σημείο βρίσκεται στη τομή των δύο ευθειών (της μεσοκαθέτου και της $ε$), έτσι $-2x+6=x \Rightarrow x=2$, και $y=1$
  \item Αν $Δ=(x,y)$ Θα πρέπει $ΑΔ^2+ΒΔ^2=ΑΒ^2$ και $y=-x+3$, έτσι
$$ (x-1)^2+(-x+3+2)^2 + (x+1)^2+(-x+3-2)^2 = 2^2+4^2$$
με λύσεις $x=1$ και $x=2$, με αντίστοιχα $y=2$ και $y=1$
  \item $\overrightarrow{ΑΝ}=(a-1, b+2)$ και $2\overrightarrow{ΒΡ}=2(x+1, -x+3-2)$. Έτσι αν λυθεί το σύστημα βρίσκετε $a-1=2x+2$ και $a+2=-2x+2$ Έτσι $a+b+1=4$
  \item Για $x=0 \Rightarrow y=3$ και για $y=0 \Rightarrow x=3$. Το εμβαδό είναι λοιπόν $Ε=\frac{9}{2}$
\end{enumerate}

\end{document}
