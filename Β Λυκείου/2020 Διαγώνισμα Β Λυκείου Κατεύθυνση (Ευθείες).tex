\documentclass[12pt]{article}

\usepackage{amsmath}
\usepackage{unicode-math}
\usepackage{xltxtra}
\usepackage{xgreek}

\setmainfont{Liberation Serif}

\usepackage{tabularx}

\pagestyle{empty}

\usepackage{geometry}
 \geometry{a4paper, total={190mm,280mm}, left=10mm, top=10mm}

 \usepackage{graphicx}
 \graphicspath{ {images/} }

 \usepackage{wrapfig}

\begin{document}

\begin{table}
    \small
    \begin{tabularx}{\textwidth}{ c X r }
      \begin{tabular}{ l }
        Εισηγητής: Λόλας Κωνσταντίνος \\
        Επαναληπτικό: (Ευθείες)
      \end{tabular}
      & &
      \begin{tabular}{ r }
        Θεσσαλονίκη, 03 / 03 / 2020
      \end{tabular}
    \end{tabularx}
\end{table}

\part*{\centering{Τεστ Κατεύθυνση Β Λυκείου}}

Δίνονται τα σημεία $Α(1,-2)$, $Β(-1,2)$ και η ευθεία $y=-x+3$. Να βρείτε:
\begin{enumerate}
  \item τη παράλληλη της $ε$ που περνάει από το $Α$
  \item τη μεσοκάθετο του τμήματος $ΑΒ$
  \item το σημείο $Γ$ της ευθείας $ε$ που ισαπέχει από τα $Α$ και $Β$
  \item το σημείο $Δ$ της ευθείας $ε$, ώστε το τρίγωνο $ΑΔΒ$ να είναι ορθογώνιο στο $Δ$
  \item πού κινείται το σημείο $Ν$, όταν το σημείο $Ρ$ κινείται στην ευθεία $ε$ και ισχύει $\overrightarrow{ΑΝ}=2\overrightarrow{ΒΡ}$
  \item το εμβαδό του τριγώνου που σχηματίζεται από την $ε$ και τους άξονες
  \item το εμβαδό του τετραγώνου $ΑΖΗΘ$ που τα σημεία $Ζ$ και $Η$ είναι στην ευθεία $ε$
\end{enumerate}

\vspace{3\baselineskip}

\part*{\centering{Καλή επιτυχία}}

\end{document}
