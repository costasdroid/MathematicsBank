\documentclass[12pt]{article}

\usepackage{amsmath}
\usepackage{unicode-math}
\usepackage{xltxtra}
\usepackage{xgreek}

\setmainfont{Liberation Serif}

\usepackage{tabularx}

\pagestyle{empty}

\usepackage{geometry}
 \geometry{a4paper, total={190mm,280mm}, left=10mm, top=10mm}

 \usepackage{graphicx}
 \graphicspath{ {images/} }

 \usepackage{wrapfig}

\begin{document}

\begin{table}
 \small
 \begin{tabularx}{\textwidth}{ c X r }
  \begin{tabular}{ l }
   Εισηγητής: Λόλας Κωνσταντίνος \\
   Επαναληπτικό: (Ευθείες)
  \end{tabular}
   &  &
  \begin{tabular}{ r }
   Θεσσαλονίκη, 04 / 03 / 2022
  \end{tabular}
 \end{tabularx}
\end{table}

\part*{\centering{Ένα ωραίο τεστάκι, μια ηλιόλουστη μέρα του Μαρτίου}}

Στην αυλή ενός σχολείου, σε ώρα διαλείμματος, τρεις παρέες βρίσκονται στα σημεία Α(0,2), Β(4,2) και Γ(4,0).

\begin{enumerate}
 \item Σε ποια θέση πρέπει να βρίσκεται ο εφημερεύων καθηγητής ώστε να ισαπέχει από τις παρέες και να προλάβει να δώσει μια ψιλή σε όποιον από αυτούς δεν φοράει μάσκα.
 \item Οι 3 παρέες αποφασίζουν να γράψουν ένα σύνθημα στην αυλή σε τρίγωνο που σχηματίζεται με κορυφές τα σημεία που στέκονται εκείνη τη στιγμή. Τι εμβαδόν θα έχει το τρίγωνο που σχηματίστηκε?
 \item Κάποιος από μια παρέα που παίζει βόλεϊ κάνει καρφί με την μπάλα (θεωρήστε την σημειακή) να διαγράφει πορεία πάνω στην y=x. Αν ο καθηγητής βρίσκεται στο σημείο Κ(2,1), θα τον χτυπήσει η μπάλα? Αν όχι σε πόση απόσταση θα περάσει από αυτόν?
\end{enumerate}

\vspace{3\baselineskip}

\part*{\centering{Καλή επιτυχία}}

\pagebreak

\part*{\centering{Λύσεις}}
1. Θα πρέπει οι αποστάσεις από το σημείο $K(x,y)$ και τις παρέες να είναι ίσες. Άρα
$$d(A,K)=d(B,K)\implies \sqrt{(x-0)^2+(y-2)^2}=\sqrt{(x-4)^2+(y-2)^2}\implies\ldots\implies x=2$$
και
$$d(B,K)=d(Γ,K)\implies \sqrt{(x-4)^2+(y-2)^2}=\sqrt{(x-4)^2+(y-0)^2}\implies\ldots\implies y=1$$
Δηλαδή το σημείο είναι το $K(2,1)$.


2. Το τρίγωνο έχει εμβαδό
$$\frac{1}{2}\|\vec{AB},\vec{AΓ}\|=\ldots=4$$

3. Το σημείο $(2,1)$ δεν ανήκει στην ευθεία $y=x$, άρα δεν θα χτυπήσεις τον καθηγητή. Η απόσταση που θα περάσει θα είναι
$$d=\frac{|1\cdot 2-1\cdot1+0|}{\sqrt{1^1+(-1)^2}}=\frac{\sqrt{2}}{2}$$


\end{document}
