\documentclass[12pt]{article}

\usepackage{amsmath}
\usepackage{unicode-math}
\usepackage{xltxtra}
\usepackage{xgreek}

\setmainfont{Liberation Serif}

\usepackage{tabularx}

\pagestyle{empty}

\usepackage{geometry}
 \geometry{a4paper, total={190mm,280mm}, left=10mm, top=10mm}

 \usepackage{graphicx}
 \graphicspath{ {images/} }

 \usepackage{wrapfig}
 \usepackage{multicol}
 \usepackage{enumitem}
 \usepackage{hyperref}

\begin{document}

\begin{table}
      \small
      \begin{tabularx}{\textwidth}{ c X r }
            \begin{tabular}{ l }
                  Εισηγητής: Λόλας Κωνσταντίνος \\
                  Επαναληπτικό: Τριγωνομετρία
            \end{tabular}
             &  &
            \begin{tabular}{ r }
                  Θεσσαλονίκη, 11 / 12 / 2023
            \end{tabular}
      \end{tabularx}
\end{table}

\part*{\centering{Διαγώνισμα Άλγεβρα Β Λυκείου}}

\section*{Θέμα Α}
\noindent
\begin{enumerate}
      \item \textbf{[Μονάδες 12]} Να αποδείξετε τις εξής ταυτότητες:

            \begin{multicols}{2}
                  \begin{enumerate}[label=(\roman*).]
                        \item $συν^2ω=\frac{1}{1+εφ^2ω}$
                        \item $ημ^2ω=\frac{εφ^2ω}{1+εφ^2ω}$
                  \end{enumerate}
            \end{multicols}

      \item \textbf{[Μονάδες 3]} Πότε μια συνάρτηση $f$ λέγεται περιοδική;

      \item \textbf{[Μονάδες 10]} Να χαρακτηρίσετε τις παρακάτω προτάσεις με Σωστό ή Λάθος
            \begin{enumerate}
                  \item [α)] Οι αντίθετες γωνίες έχουν αντίθετο συνημίτονο
                  \item [β)] $εφ(2kπ+ω)=εφω$, $k\in\mathbb{Z}$
                  \item [γ)] Η συνάρτηση $f(x)=ημx$ είναι περιοδική με περίοδο $2π$
                  \item [δ)] Η εξίσωση $εφx=εφω$ έχει μία λύση
                  \item [ε)] Η συνάρτηση $f(x)=ρσυν(ωx)$, $ρ>0$, $ω>0$ έχει μέγιστη τιμή το $ρ$, ελάχιστη τιμή το $-ρ$ και περίοδο $Τ=\frac{2π}{ω}$
            \end{enumerate}
\end{enumerate}

\section*{Θέμα Β}
\noindent
Έστω γωνία $ω\in\left( \frac{π}{2},π \right)$ για την οποία ισχύει η σχέση:
$$ημ(π-ω)+συν(\frac{π}{2}-ω)=\frac{6}{5}$$
\begin{enumerate}
      \item \textbf{[Μονάδες 7]} Να αποδείξετε ότι $ημω=\frac{3}{5}$
      \item \textbf{[Μονάδες 6]} Να βρείτε τους άλλους τριγωνομετρικούς αριθμούς της γωνίας $ω$
      \item \textbf{[Μονάδες 6]} Να βρείτε τους τριγωνομετρικούς αριθμούς της γωνίας $\frac{15π}{2}+ω$
      \item \textbf{[Μονάδες 6]} Να αποδείξετε ότι $\frac{3π}{4}<ω<\frac{5π}{6}$
\end{enumerate}
%\vspace{7\baselineskip}

\section*{Θέμα Γ}
\noindent
Δίνεται η συνάρτηση $f(x)=4συν^2x-4συνx+2$, $x\in (0,2π)$.
\begin{enumerate}
      \item \textbf{[Μονάδες 8]} Να λύσετε τις εξισώσεις
            \begin{multicols}{2}
                  \begin{enumerate}[label=(\roman*).]
                        \item $f(x)=2$
                        \item $f(x)=1$
                  \end{enumerate}
            \end{multicols}
      \item \textbf{[Μονάδες 9]} Να αποδείξετε ότι η συνάρτηση $f$
            \begin{enumerate}[label=\roman*.]
                  \item είναι περιοδική με περίοδο $Τ=2π$
                  \item είναι άρτια
                  \item δεν είναι γνησίως μονότονη
            \end{enumerate}
      \item \textbf{[Μονάδες 8]} Να αποδείξετε ότι $f(x)\ge 1$ για κάθε $x\in (0,2π)$ και στη συνέχεια να βρείτε τα $x$ για τα οποία η $f$ παρουσιάζει ελάχιστο
\end{enumerate}

\section*{Θέμα Δ}
\noindent
Δίνεται η συνάρτηση $f(x)=ρ\cdot ημ(ωx)+κ$, όπου $ρ>0$, $ω>0$ και $κ\in \mathbb{R}$, τέτοια ώστε:
\begin{itemize}
      \item έχει περίοδο $Τ=π$
      \item έχει ελάχιστη τιμή το 1
      \item και η $C_f$ τέμνει τον άξονα $y'y$ στο σημείο με τεταγμένη $2$
\end{itemize}
\begin{enumerate}
      \item \textbf{[Μονάδες 6]} να υπολογίσετε τα $ρ$, $ω$ και $κ$

            Αν $ρ=1$, $ω=κ=2$, τότε

      \item \textbf{[Μονάδες 4]} Να σχεδιάσετε τη γραφική παράσταση της συνάρτησης $f$ στο διάστημα $[0,π]$
      \item \textbf{[Μονάδες 8]} Αν η ευθεία $y=\frac{5}{2}$ τέμνει τη γραφική παράσταση της $f$ στο διάστημα $[0,π]$ στα σημεία $Κ$ και $Λ$, τότε να υπολογίσετε το εμβαδόν και την περίμετρο του τριγώνου $ΟΚΛ$, όπου $Ο$ η αρχή των αξόνων
      \item \textbf{[Μονάδες 7]} Να λύσετε την εξίσωση $f(2x)=f(3x)$ στο διάστημα $[0,π]$
\end{enumerate}

\vspace{3\baselineskip}

\part*{\centering{Καλή επιτυχία}}

\end{document}
