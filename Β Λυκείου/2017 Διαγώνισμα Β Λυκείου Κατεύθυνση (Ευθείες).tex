\documentclass[12pt]{article}

\usepackage{amsmath}
\usepackage{unicode-math}
\usepackage{xltxtra}
\usepackage{xgreek}

\setmainfont{Liberation Serif}

\usepackage{tabularx}

\pagestyle{empty}

\usepackage{geometry}
 \geometry{a4paper, total={190mm,280mm}, left=10mm, top=10mm}

 \usepackage{graphicx}
 \graphicspath{ {images/} }

 \usepackage{wrapfig}

\begin{document}

\begin{table}
 \small
 \begin{tabularx}{\textwidth}{ c X r }
  \begin{tabular}{ l }
   Εισηγητής: Λόλας Κωνσταντίνος \\
   Επαναληπτικό: (Ευθείες)
  \end{tabular}
   &  &
  \begin{tabular}{ r }
   Θεσσαλονίκη, 09 / 02 / 2023
  \end{tabular}
 \end{tabularx}
\end{table}

\part*{\centering{Τεστ Κατεύθυνση Β Λυκείου}}

Δίνονται τα σημεία $Α(1,-2)$, $Β(-1,2)$ και η ευθεία $(ε):y=-x+3$. Να βρείτε:
\begin{enumerate}
 \item{\textbf{[Μονάδες 10]}} το εμβαδό του τριγώνου που σχηματίζεται από την $(ε)$ και τους άξονες
 \item{\textbf{[Μονάδες 10]}} την απόσταση του $Α$ από την ευθεία $(ε)$
 \item{\textbf{[Μονάδες 10]}} τη παράλληλη της $(ε)$ που περνάει από το $Α$
 \item{\textbf{[Μονάδες 12]}} τη μεσοκάθετο του τμήματος $ΑΒ$
 \item{\textbf{[Μονάδες 13]}} το σημείο $Γ$ της ευθείας $(ε)$ που ισαπέχει από τα $Α$ και $Β$
 \item{\textbf{[Μονάδες 15]}} το σημείο $Δ$ της ευθείας $(ε)$, ώστε το τρίγωνο $ΑΔΒ$ να είναι ορθογώνιο στο $Δ$
 \item{\textbf{[Μονάδες 15]}} τη γωνία μεταξύ της ευθείας που ανήκει το $ΑΒ$ και της ευθείας $(ε)$
 \item{\textbf{[Μονάδες 15]}} πού κινείται το σημείο $Ν$, όταν το σημείο $Ρ$ κινείται στην ευθεία $(ε)$ και ισχύει $\overrightarrow{ΑΝ}=2\overrightarrow{ΒΡ}$
\end{enumerate}

\vspace{3\baselineskip}

\part*{\centering{Καλή επιτυχία}}

\end{document}
