\documentclass[12pt]{article}

\usepackage{amsmath}
\usepackage{unicode-math}
\usepackage{xltxtra}
\usepackage{xgreek}

\setmainfont{Liberation Serif}

\usepackage{tabularx}

\pagestyle{empty}

\usepackage{geometry}
 \geometry{a4paper, total={190mm,280mm}, left=10mm, top=10mm}

 \usepackage{graphicx}
 \graphicspath{ {images/} }

 \usepackage{wrapfig}

\begin{document}

\begin{table}
    \small
    \begin{tabularx}{\textwidth}{ c X r }
      \begin{tabular}{ l }
        Εισηγητής: Λόλας Κωνσταντίνος \\
        Επαναληπτικό: (Διανύσματα)
      \end{tabular}
      & &
      \begin{tabular}{ r }
        Θεσσαλονίκη, 19 / 01 / 2018
      \end{tabular}
    \end{tabularx}
\end{table}

\part*{\centering{Λύσεις Κατεύθυνση Β Λυκείου}}

\section*{Θέμα Α}
  \noindent
  \begin{enumerate}
    \item Θεωρία.
    \item Θεωρία.
    \item Λ,Λ,Λ,Λ,Σ
  \end{enumerate}

\section*{Θέμα Β}
  \noindent
  Αν $\vec{α}=(1,2)$ και $\vec{β}=(-2,1)$.
  \begin{enumerate}
    \item Να βρεθεί το διάνυσμα $2\vec{α}+\vec{β}=(0,5)$.
    \item $-\vec{α}=(-1,-2)$.
    \item $\vec{μ}=(\frac{-1}{2},\frac{3}{2})$
    \item Από το 1. $\vec{γ}=2\vec{α}+\vec{β}$
  \end{enumerate}
  %\vspace{7\baselineskip}

\section*{Θέμα Γ}
  \noindent
  Δίνονται τα διανύσματα $\vec{α}=(κ-2,-2κ)$ και $\vec{β}=(-κ-3,κ-2)$, $κ>0$
  \begin{enumerate}
    \item $-κ^2-3κ+2κ+6-2κ^2+4κ=0 \Rightarrow -3κ^2+3κ+6=0$. Άρα $κ=-1$, $κ=-2$
    \item $\sqrt{κ^2-4κ+4+4κ^2}=\sqrt{3}$ άρα $5κ^2-4κ+1=0$ άρα δεν υπάρχει $κ$
  \end{enumerate}
  Αν $\vec{γ}=\vec{β}+(6,2)=(3-κ,κ)$,
  \begin{enumerate}
    \item[3.] Θα πρέπει η ορίζουσα να είναι ίση με $0$ άρα $κ^2-2κ-(-6κ+2κ^2)=-κ^2+8κ=0$. Άρα $κ=0$ ή $κ=8$.
  \end{enumerate}
  %\vspace{7\baselineskip}

\section*{Θέμα Δ}
  \noindent
  \begin{enumerate}
    \item Από τα κάθετα βγάζουμε ότι $\vec{α}^2-9\vec{β}^2=0$ και από το μέτρο $\vec{α}^2-10\vec{α}\vec{β}+25\vec{β}^2=400$ και άρα $|\vec{α}|=3|\vec{β}|$ και $|\vec{α}|^2-450+25|\vec{β}|^2=400$. Δηλαδή $|\vec{α}|=15$ και $|\vec{β}|=5$.
    \item $$|\vec{α}+3\vec{β}|^2=\vec{α}^2+6\vec{α}\vec{β}+9\vec{β}^2=225+6\cdot 45+9\cdot 45=720=36\cdot 20$$ Δηλαδή $|\vec{α}+3\vec{β}|=6\sqrt{20}$.
    \item $$συνθ=\frac{\vec{α}(\vec{α}+3\vec{β})}{|\vec{α}||\vec{α}+3\vec{β}|}=\frac{\vec{α}^2+3\vec{α}\vec{β}}{15\cdot 6\sqrt{20}}=\frac{360}{90\sqrt{20}}=\frac{2\sqrt{5}}{5}$$
  \end{enumerate}

\vspace{3\baselineskip}

\part*{\centering{Καλή επιτυχία}}

\end{document}
