\documentclass[12pt]{article}

\usepackage{amsmath}
\usepackage{unicode-math}
\usepackage{xltxtra}
\usepackage{xgreek}

\setmainfont{Liberation Serif}

\usepackage{tabularx}

\pagestyle{empty}

\usepackage{geometry}
 \geometry{a4paper, total={190mm,280mm}, left=10mm, top=10mm}

 \usepackage{graphicx}
 \graphicspath{ {images/} }

 \usepackage{wrapfig}

\begin{document}

\begin{table}
  \small
  \begin{tabularx}{\textwidth}{ c X r }
    \begin{tabular}{ l }
      Εισηγητής: Λόλας Κωνσταντίνος \\
      Επαναληπτικό: (Διανύσματα)
    \end{tabular}
     &  &
    \begin{tabular}{ r }
      Θεσσαλονίκη, 14 / 12 / 2023
    \end{tabular}
  \end{tabularx}
\end{table}

\part*{\centering{Διαγώνισμα Κατεύθυνση Β Λυκείου}}

\section*{Θέμα Α}
\noindent
\begin{enumerate}
  \item \textbf{[Μονάδες 10]} Να αποδείξετε ότι ο γεωμετρικός μέσος $\overrightarrow{ΟΜ}$ ενός ευθύγραμμου τμήματος με άκρα $Α=(x_1,y_1)$ και $Β=(x_2,y_2)$ είναι $\overrightarrow{OΜ}=\left( \frac{x_1+x_2}{2},\frac{y_1+y_2}{2}\right)$.
  \item \textbf{[Μονάδες 5]} Πώς ορίζεται το $συνθ$ της γωνίας $θ$ δύο διανυσμάτων $\vec{α}$ και $\vec{β}$.
  \item \textbf{[Μονάδες 10]} Να χαρακτηρίσετε τις παρακάτω προτάσεις με Σωστό ή Λάθος
        \begin{enumerate}
          \item [α)] $\vec{α}\perp\vec{β}\Leftrightarrow \vec{α}\cdot\vec{β}=-1$ για κάθε $\vec{α}$ και $\vec{β}$.
          \item [β)] $|\vec{α}\cdot\vec{β}| = |\vec{α}|\cdot|\vec{β}|$ για κάθε $\vec{α}$ και $\vec{β}$.
          \item [γ)] Η κλίση ενός διανύσματος $(x,y)$ είναι $λ=\frac{y}{x}$ για κάθε $(x,y)$.
          \item [δ)] Το διάνυσμα με άκρα τα $Α=(x_1,y_1)$ και $Β=(x_2,y_2)$ είναι το $\left( x_1+x_2,y_1+y_2\right)$.
          \item [ε)] $|\vec{α}|^2=\vec{α}^2$ για κάθε $\vec{α}$.
        \end{enumerate}
\end{enumerate}

\section*{Θέμα Β}
\noindent
Αν $\vec{α}=(1,2)$ και $\vec{β}=(-2,1)$.
\begin{enumerate}
  \item \textbf{[Μονάδες 5]} Να βρεθεί το διάνυσμα $2\vec{α}+\vec{β}$.
  \item \textbf{[Μονάδες 5]} Να βρεθεί το αντίθετο διάνυσμα του $\vec{α}$.
  \item \textbf{[Μονάδες 7]} Να βρεθεί ο διανυσματικός μέσος $\vec{μ}$ των $\vec{α}$ και $\vec{β}$.
  \item \textbf{[Μονάδες 8]} Να γραφτεί το διάνυσμα $(0,5)$ ως γραμμικός συνδυασμός των $\vec{α}$ και $\vec{β}$.
\end{enumerate}
%\vspace{7\baselineskip}

\section*{Θέμα Γ}
\noindent
Δίνονται τα διανύσματα $\vec{α}=(κ-2,-2κ)$ και $\vec{β}=(-κ-3,κ-2)$, $κ>0$
\begin{enumerate}
  \item \textbf{[Μονάδες 7]} Να βρεθεί το $κ$ ώστε τα διανύσματα να είναι κάθετα.
  \item \textbf{[Μονάδες 7]} Να βρεθεί το $κ$ ώστε $|\vec{α}|=\sqrt{5}$.
\end{enumerate}
Αν $\vec{γ}=\vec{β}+(6,2)$,
\begin{enumerate}
  \item[3.] \textbf{[Μονάδες 11]} Να βρεθεί το $κ$ ώστε τα διανύσματα $\vec{α}$ και $\vec{γ}$ να είναι παράλληλα.
\end{enumerate}
%\vspace{7\baselineskip}

\section*{Θέμα Δ}
\noindent
Δίνονται τα διανύσματα $\vec{α}$ και $\vec{β}$ για τα οποία ισχύουν $\vec{α}\cdot\vec{β}=45$, $\left(\vec{α}+3\vec{β}\right)\perp\left(\vec{α}-3\vec{β}\right)$ και $|\vec{α}-5\vec{β}|=20$.
\begin{enumerate}
  \item \textbf{[Μονάδες 10]} Να δείξετε ότι $|\vec{α}|=15$ και $|\vec{β}|=5$.
  \item \textbf{[Μονάδες 10]} Να δείξετε ότι $|\vec{α}+3\vec{β}|=6\sqrt{20}$.
  \item \textbf{[Μονάδες 5]} Αν $θ$ η γωνία των διανυσμάτων $\vec{α}$ και $\vec{α}+3\vec{β}$, να δείξετε ότι $συνθ=\frac{2\sqrt{5}}{5}$.
\end{enumerate}

\vspace{3\baselineskip}

\part*{\centering{Καλή επιτυχία}}

\end{document}
