\documentclass[12pt]{extarticle}

\usepackage{amsmath}
\usepackage{unicode-math}
\usepackage{xltxtra}
\usepackage{xgreek}

\setmainfont{Liberation Serif}

\usepackage{tabularx}

\pagestyle{empty}

\usepackage{geometry}
\geometry{a4paper, total={190mm,275mm}, left=10mm, top=10mm}

\usepackage{graphicx}
\graphicspath{ {images/} }

\usepackage{wrapfig}

\begin{document}
\renewcommand{\labelenumi}{\alph{enumi})}
\renewcommand{\labelenumii}{\roman{enumii}.}

\begin{table}
    \small
    \begin{tabularx}{\textwidth}{ c X r }
        \begin{tabular}{ c }
            \includegraphics[scale=0.4]{ελληνική}         \\
            ΕΛΛΗΝΙΚΗ ΔΗΜΟΚΡΑΤΙΑ                           \\
            ΥΠΟΥΡΓΕΙΟ ΠΑΙΔΕΙΑΣ \& ΘΡΗΣΚΕΥΜΑΤΩΝ            \\
            ΠΕΡΙΦΕΡΕΙΑΚΗ Δ/ΝΣΗ Α/ΘΜΙΑΣ \& Β/ΘΜΙΑΣ ΕΚΠ/ΣΗΣ \\
            ΚΕΝΤΡΙΚΗΣ ΜΑΚΕΔΟΝΙΑΣ                          \\
            Δ/ΝΣΗ Β/ΘΜΙΑΣ ΕΚΠ/ΣΗΣ ΑΝ. ΘΕΣ/ΝΙΚΗΣ           \\
            10ο ΓΕΝΙΚΟ ΛΥΚΕΙΟ ΘΕΣ/ΝΙΚΗΣ
        \end{tabular}
         &  &
        \begin{tabular}{ r }
            Σχολικό Έτος: 2022 - 2023       \\
            Εξ. Περίοδος: Μαΐου - Ιουνίου   \\
            Μάθημα: Μαθηματικά Β Κατεύθυνση \\
            Εισηγητές: Κράντας, Λόλας       \\ \\
            Θεσσαλονίκη, 13 / 06 / 2023
        \end{tabular}
    \end{tabularx}
\end{table}

\part*{\centering{Θέματα}}
\section*{Θέμα 1}
\noindent

\begin{enumerate}
    \item[α)] Αν $Α(x_1,y_1)$ και $Β(x_2,y_2)$ είναι δύο σημεία του επιπέδου, να αποδείξετε ότι οι συντεταγμένες του μέσου $Μ$ του τμήματος $ΑΒ$ δίνονται από τους τύπους

        $$x_μ=\dfrac{x_1+x_2}{2} \text{ και } y_μ=\dfrac{y_1+y_2}{2}$$

        \hspace*{\fill} \textbf{Μονάδες 15}

    \item[β)] Να χαρακτηρίσετε τις παρακάτω προτάσεις με Σωστό ή Λάθος
        \begin{enumerate}
            \item Αν τα διανύσματα $\vec{α}$ και $\vec{β}$ είναι παράλληλα τότε $\vec{α}\cdot\vec{β}=|\vec{α}||\vec{β}|$.

            \item Κάθε εξίσωση της μορφής $Αx+Βy+Γ=0$ με $Α\ne0$ ή $Β\ne 0$ παριστάνει ευθεία
            \item Το διάνυσμα $(Β,Α)$ είναι κάθετο στην ευθεία $Αx+Βy+Γ=0$
            \item Αν $λ\in\mathbb{R}$ και $μ\in\mathbb{R}$ με $λ\vec{α}=μ\vec{α}$ τότε $λ=μ$
            \item Το κέντρο $Κ$ του κύκλου $x^2+y^2+Αx+Βy+Γ=0$ είναι το $Κ(-\dfrac{Α}{2},-\dfrac{Β}{2})$\hspace*{\fill}\textbf{Μονάδες 10}
        \end{enumerate}
\end{enumerate}

\section*{Θέμα 2 (15044)}
\noindent
Δίνονται τα σημεία $Α(0,5)$ και $Β(6,-1)$
\begin{enumerate}
    \item[α)]
        \begin{enumerate}
            \item Να βρείτε τον συντελεστή διεύθυνσης της ευθείας που διέρχεται από τα σημεία $Α$ και $Β$\hspace*{\fill} \textbf{Μονάδες 5}
            \item Να αποδείξετε ότι το μέσο του ευθύγραμμου τμήματος $ΑΒ$, είναι το σημείο $Μ(3,2)$\hspace*{\fill} \textbf{Μονάδες 5}
        \end{enumerate}
    \item[β)] Να βρείτε την εξίσωση της μεσοκάθετης ευθείας $(ε)$ του ευθύγραμμου τμήματος $ΑΒ$ \hspace*{\fill} \textbf{Μονάδες 15}
\end{enumerate}


\section*{Θέμα 3}
\noindent
Έστω $|\vec{α}|=1$, $|\vec{β}|=2$, $(\widehat{\vec{α},\vec{β}})=\dfrac{π}{3}$ και τα διανύσματα $\vec{u}=2\vec{α}-\vec{β}$ και $\vec{v}=\vec{α}-\vec{β}$
\begin{enumerate}
    \item[α)] Να υπολογίσετε το εσωτερικό γινόμενο $\vec{α}\cdot \vec{β}$ \hspace*{\fill} \textbf{Μονάδες 6}

        Αν $\vec{α}\cdot \vec{β}=1$ να υπολογίσετε

    \item[β)] Το μέτρο του διανύσματος $\vec{u}$ \hspace*{\fill} \textbf{Μονάδες 6}
    \item[γ)] Το μέτρο του διανύσματος $\vec{v}$ \hspace*{\fill} \textbf{Μονάδες 6}
    \item[δ)] Τη γωνία των διανυσμάτων $\vec{u}$ και $\vec{v}$ \hspace*{\fill} \textbf{Μονάδες 7}
\end{enumerate}

\section*{Θέμα 4 (20700)}
\noindent

Δίνεται το τετράγωνο $ΜΜ_1ΟΜ_2$ με $Μ(4,4)$, $Μ_1(4,0)$, $Μ_2(0,4)$. Αν $Ο$ η αρχή των αξόνων του καρτεσιανού συστήματος συντεταγμένων, τότε:
\begin{enumerate}
    \item[α)] Να δείξετε ότι ο κύκλος που διέρχεται από τις κορυφές του τετραγώνου $ΜΜ_1ΟΜ_2$ έχει εξίσωση

        $C:(x-2)^2+(y-2)^2=8$

        \hspace*{\fill} \textbf{Μονάδες 8}
    \item[β)] Να αποδείξετε ότι η ευθεία $ε:x+y=8$ είναι εφαπτομένη του παραπάνω κύκλου $C$ \hspace*{\fill} \textbf{Μονάδες 8}
    \item[γ)] Να βρείτε το σημείο επαφής της ευθείας $ε$ με τον κύκλο $C$ \hspace*{\fill} \textbf{Μονάδες 9}
\end{enumerate}

\begin{table}[htb]
    \begin{tabularx}{\textwidth}{ X c X c X}
         &
        \begin{tabular}[t]{ c }
            Ο Δ/ντης \\ \\ \\ \\
            Παπαδημητρίου Χρήστος
        \end{tabular}
         &   &
        \begin{tabular}[t]{ c }
            Ο εισηγητές        \\ \\ \\ \\
            Κράντας Στυλιανός  \\ \\ \\ \\
            Λόλας Κωνσταντίνος \\ \\
        \end{tabular}
         &
    \end{tabularx}
\end{table}
\end{document}