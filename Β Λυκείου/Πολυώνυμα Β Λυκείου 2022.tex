\documentclass[12pt]{article}

\usepackage{amsmath}
\usepackage{unicode-math}
\usepackage{xltxtra}
\usepackage{xgreek}

\setmainfont{Liberation Serif}

\usepackage{tabularx}

\pagestyle{empty}

\usepackage{geometry}
 \geometry{a4paper, total={190mm,280mm}, left=10mm, top=10mm}

\usepackage{graphicx}
\graphicspath{ {images/} }

\usepackage{wrapfig}

\begin{document}
\part*{\centering{Ήρθε το τέλος!}}

Θεωρήστε το πολυώνυμο $P(x)=2x^4-5x^3+4x^2-5x+2$
\begin{enumerate}
 \item[1.] \textbf{[Μονάδες 2]} Να δείξετε ότι το 0 δεν είναι ρίζα του.
 \item[2.] \textbf{[Μονάδες 2]} Να δείξετε ότι το υπόλοιπο της διαίρεσης με το $x-1$ είναι $-2$.
 \item[3.] \textbf{[Μονάδες 3]} Να γράψετε την ταυτότητα της διαίρεσης του $P(x)$ με το $x^3+1$
 \item[4.] \textbf{[Μονάδες 3]} Να δείξετε ότι αν ένας αριθμός $ρ$ είναι ρίζα του πολυωνύμου τότε και ο αριθμός $\dfrac{1}{ρ}$ είναι ρίζα.
 \item[5.] \textbf{[Μονάδες 2]} Να γράψετε όλες τις πιθανές ακέραιες ρίζες του πολυωνύμου.
 \item[6.] \textbf{[Μονάδες 4]} Να λύσετε την εξίσωση $P(x)=0$.
 \item[7.] \textbf{[Μονάδες 4]} Να λύσετε την ανίσωση $P(x)-2x^4+5x^3-4x^2+6x-1>\dfrac{2}{x}$.
\end{enumerate}

\pagebreak


Θεωρήστε το πολυώνυμο $P(x)=2x^4-5x^3+4x^2-5x+2$
\begin{enumerate}
 \item[1.] \textbf{[Μονάδες 2]} Να δείξετε ότι το 0 δεν είναι ρίζα του.
 \item[2.] \textbf{[Μονάδες 2]} Να δείξετε ότι το υπόλοιπο της διαίρεσης με το $x-1$ είναι $-2$.
 \item[3.] \textbf{[Μονάδες 3]} Να γράψετε την ταυτότητα της διαίρεσης του $P(x)$ με το $x^3+1$
 \item[4.] \textbf{[Μονάδες 3]} Να δείξετε ότι αν ένας αριθμός $ρ$ είναι ρίζα του πολυωνύμου τότε και ο αριθμός $\dfrac{1}{ρ}$ είναι ρίζα.
 \item[5.] \textbf{[Μονάδες 2]} Να γράψετε όλες τις πιθανές ακέραιες ρίζες του πολυωνύμου.
 \item[6.] \textbf{[Μονάδες 4]} Να λύσετε την εξίσωση $P(x)=0$.
 \item[7.] \textbf{[Μονάδες 4]} Να λύσετε την ανίσωση $P(x)-2x^4+5x^3-4x^2+6x-1>\dfrac{2}{x}$.
\end{enumerate}

\part*{\centering{Λύσεις}}

\begin{enumerate}
 \item[1.] Έχουμε $P(0)=2 \ne 0$ άρα δεν είναι ρίζα.
 \item[2.] Το υπόλοιπο της διαίρεσης είναι το $P(1)=-2$.
 \item[3.] Με κάθετη διαίρεση προκύπτει $π(x)=2x-5$ και $υ(x)=4x^2-7x+7$, άρα $$P(x)=(2x-5)(x^3+1)+4x^2-7x+7$$
 \item[3.] Έχουμε ότι $P(ρ)=0$ άρα
 $$2ρ^4-5ρ^3+4ρ^2-5ρ+2=0 \implies ρ^4(2-5\dfrac{1}{ρ}+4(\dfrac{1}{ρ})^2-5(\dfrac{1}{ρ})^3+2(\dfrac{1}{ρ})^4)=0\implies \dfrac{1}{ρ} \text{ ρίζα}$$
 \item[5.] Είναι όλοι οι διαιρέτες του $2$, δηλαδή $\{-2,-1,1,2\}$.
 \item[6.] Η μία προφανής ρίζα είναι το 2. Από το ερώτημα 3 έχουμε ότι είναι και το $\dfrac{1}{2}$. Με Horner παίρνουμε
 $$P(x)=(x-2)(x-1/2)(2x^2+2)$$
 Ο τρίτος όρος δεν μηδενίζει οπότε οι δύο ρίζες είναι και μοναδικές
 \item[7.] Έχουμε
 $$P(x)-2x^4+5x^3-4x^2+6x-1>\dfrac{2}{x} \implies x+1>\dfrac{2}{x} \implies \dfrac{(x-1)(x-2)}{x}>0$$
 Που με πινακάκι έχουμε $x>2$ ή $0<x<1$
\end{enumerate}


\end{document}
