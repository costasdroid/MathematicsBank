\documentclass[12pt]{article}

\usepackage{amsmath}
\usepackage{unicode-math}
\usepackage{xltxtra}
\usepackage{xgreek}

\setmainfont{Liberation Serif}

\usepackage{tabularx}

\pagestyle{empty}

\usepackage{geometry}
 \geometry{a4paper, total={190mm,280mm}, left=10mm, top=10mm}

 \usepackage{graphicx}
 \graphicspath{ {images/} }

 \usepackage{wrapfig}
 \usepackage{multicol}
 \usepackage{enumitem}
 \usepackage{hyperref}

\begin{document}

\section*{Θέμα Α}
\noindent
\begin{enumerate}
 \item \textbf{[Μονάδες 12]} Να αποδείξετε τις εξής ταυτότητες:

       \begin{multicols}{2}
        \begin{enumerate}[label=(\roman*).]
         \item $συν^2ω=\frac{1}{1+εφ^2ω}$
         \item $ημ^2ω=\frac{εφ^2ω}{1+εφ^2ω}$
        \end{enumerate}
       \end{multicols}

       Από βιβλίο

 \item \textbf{[Μονάδες 3]} Πότε μια συνάρτηση $f$ λέγεται περιοδική;

       Θεωρία

 \item \textbf{[Μονάδες 10]} Να χαρακτηρίσετε τις παρακάτω προτάσεις με Σωστό ή Λάθος
       \begin{enumerate}
        \item [α)] \emph{Λάθος / } Οι αντίθετες γωνίες έχουν αντίθετο συνημίτονο
        \item [β)] \emph{Σωστό / } $εφ(2kπ+ω)=εφω$, $k\in\mathbb{Z}$
        \item [γ)] \emph{Σωστό / } Η συνάρτηση $f(x)=ημx$ είναι περιοδική με περίοδο $2π$
        \item [δ)] \emph{Λάθος / } Η εξίσωση $εφx=εφω$ έχει μία λύση
        \item [ε)] \emph{Σωστό / } Η συνάρτηση $f(x)=ρσυν(ωx)$, $ρ>0$, $ω>0$ έχει μέγιστη τιμή το $ρ$, ελάχιστη τιμή το $-ρ$ και περίοδο $Τ=\frac{2π}{ω}$
       \end{enumerate}
\end{enumerate}

\section*{Θέμα Β}
\noindent
Έστω γωνία $ω\in\left( \frac{π}{2},π \right)$ για την οποία ισχύει η σχέση:
$$ημ(π-ω)+συν(\frac{π}{2}-ω)=\frac{6}{5}$$
\begin{enumerate}
 \item \textbf{[Μονάδες 7]} Να αποδείξετε ότι $ημω=\frac{3}{5}$

       Αναγωγή στο 1ο Τεταρτημόριο:

       $ημ(ω)+ημ(ω)=\frac{6}{5}\implies 2ημ(ω)=\frac{6}{5}\implies ημ(ω)=\frac{3}{5}$

 \item \textbf{[Μονάδες 6]} Να βρείτε τους άλλους τριγωνομετρικούς αριθμούς της γωνίας $ω$

       Ταυτότητες:

       $ημ^2ω+συν^2ω=1\implies \frac{9}{25}+συν^2ω=1\implies συνω=\pm \frac{4}{5}$. Η γωνία είναι στο 2ο τεταρτημόριο άρα $συνω=-\frac{4}{5}$

       $εφω=\frac{ημω}{συνω}=-\frac{3}{4}$

       $σφω=\frac{1}{εφω}=-\frac{4}{3}$

 \item \textbf{[Μονάδες 6]} Να βρείτε τους τριγωνομετρικούς αριθμούς της γωνίας $\frac{15π}{2}+ω$

       Αναγωγή στο 1ο Τεταρτημόριο:

       $\frac{15π}{2}+ω=\frac{16π}{2}-\frac{π}{2}+ω=8π-\frac{π}{2}+ω$

       $ημ(\frac{15π}{2}+ω)=ημ(8π-\frac{π}{2}+ω)=ημ(-\frac{π}{2}+ω)=-ημ(\frac{π}{2}-ω)=-συνω$

       $συν(\frac{15π}{2}+ω)=συν(8π-\frac{π}{2}+ω)=συν(-\frac{π}{2}+ω)=συν(\frac{π}{2}-ω)=ημω$

       $εφ(\frac{15π}{2}+ω)=-σφω$

       $σφ(\frac{15π}{2}+ω)=-εφω$

 \item \textbf{[Μονάδες 6]} Να αποδείξετε ότι $\frac{3π}{4}<ω<\frac{5π}{6}$

       Μονοτονία $ημx$:

       $ημ\frac{3π}{4}=\frac{\sqrt{2}}{2}$

       $ημ\frac{5π}{6}=ημ(π-\frac{π}{6})=ημ\frac{π}{6}=\frac{1}{2}$

       Και στο 2ο τεταρτημόριο η $ημx$ είναι φθίνουσα. Αρκεί να δείξω ότι $\frac{1}{2}<\frac{3}{5}<\frac{\sqrt{2}}{2}$ ή $\frac{1}{4}<\frac{9}{25}<\frac{1}{2}$ ή $\frac{25}{100}<\frac{36}{100}<\frac{50}{100}$

\end{enumerate}
%\vspace{7\baselineskip}

\section*{Θέμα Γ}
\noindent
Δίνεται η συνάρτηση $f(x)=4συν^2x-4συνx+2$, $x\in (0,2π)$.
\begin{enumerate}
 \item \textbf{[Μονάδες 8]} Να λύσετε τις εξισώσεις
       \begin{enumerate}[label=(\roman*).]
        \item $f(x)=2$

              Τριγωνομετρικές Εξισώσεις:

              $4συν^2x-4συνx+2=2\implies 4συν^2x-4συνx=0 \implies 4συνx(συνx-1)=0 \implies συνx=0$ ή $συνx=1$. Στο 1ο κύκλο σημαίνει ότι $x=\frac{π}{2}$ ή $x=\frac{3π}{2}$ ή $x=0$. Από αυτά μόνο τα $x=\frac{π}{2}$ ή $x=\frac{3π}{2}$ δεχόμαστε αφού $x\in (0,2π)$

        \item $f(x)=1$

              $4συν^2x-4συνx+2=1\implies 4συν^2x-4συνx+1=0 \implies (2συνx-1)^2=0 \implies συνx=\frac{1}{2}$. Στο 1ο κύκλο σημαίνει ότι $x=\frac{π}{3}$ ή $x=2π-\frac{π}{3}$.

       \end{enumerate}
 \item \textbf{[Μονάδες 9]} Να αποδείξετε ότι η συνάρτηση $f$
       \begin{enumerate}[label=\roman*.]
        \item είναι περιοδική με περίοδο $Τ=2π$

              Περίοδος:

              $f(x+2π)=4συν^2(x+2π)-4συν(x+2π)+2=4συν^2x-4συνx+2=f(x)$

        \item είναι άρτια

              Άρτιες περιττές:

              $f(-x)=4συν^2(-x)-4συν(-x)+2=4συν^2x-4συνx+2=f(x)$

        \item δεν είναι γνησίως μονότονη

              Μονοτονία:

              Παρατηρώ ότι $f(π/2)=2=f(3π/2)$ άρα δεν μπορεί να είναι μονότονη

       \end{enumerate}
 \item \textbf{[Μονάδες 8]} Να αποδείξετε ότι $f(x)\ge 1$ για κάθε $x\in (0,2π)$ και στη συνέχεια να βρείτε τα $x$ για τα οποία η $f$ παρουσιάζει ελάχιστο

       Μέγιστα - Ελάχιστα:

       Θα δείξω ότι $f(x)\ge 1$. Έχουμε $4συν^2x-4συνx+2=4συν^2x-4συνx+1+1=(2συνx-1)^2+1\ge 1$. Η ισότητα ισχύει για $x=\frac{1}{2}$ που έχει λυθεί πιο πριν

\end{enumerate}

\section*{Θέμα Δ}
\noindent
Δίνεται η συνάρτηση $f(x)=ρ\cdot ημ(ωx)+κ$, όπου $ρ>0$, $ω>0$ και $κ\in \mathbb{R}$, τέτοια ώστε:
\begin{itemize}
 \item έχει περίοδο $Τ=π$

       Περίοδος τριγωνομετρικής:

       $ω=2π/Τ\implies ω=2π/π=2$

 \item έχει ελάχιστη τιμή το 1

       Μέγιστο ελάχιστο τριγωνομετρικής:

       $-1\le ημ(ωx) \le 1\implies κ-ρ\le ρ\cdot ημ(ωx)+κ \le κ+ρ$. Άρα $κ-ρ=1$. Από το επόμενο βρήκα $κ=2$ άρα $ρ=1$

 \item και η $C_f$ τέμνει τον άξονα $y'y$ στο σημείο με τεταγμένη $2$

       Σημείο συνάρτησης:

       $f(0)=2\implies ημ0+κ=2\implies κ=2$

\end{itemize}
\begin{enumerate}
 \item \textbf{[Μονάδες 6]} να υπολογίσετε τα $ρ$, $ω$ και $κ$

       Αν $ρ=1$, $ω=κ=2$, τότε

 \item \textbf{[Μονάδες 4]} Να σχεδιάσετε τη γραφική παράσταση της συνάρτησης $f$ στο διάστημα $[0,π]$

       Γραφική παράσταση:

       Από $ημω$, η περίοδος είναι $π$ άρα προσαρμογή στον $x'x$ και μεταφορά 2 επάνω.

 \item \textbf{[Μονάδες 8]} Αν η ευθεία $y=\frac{5}{2}$ τέμνει τη γραφική παράσταση της $f$ στο διάστημα $[0,π]$ στα σημεία $Κ$ και $Λ$, τότε να υπολογίσετε το εμβαδόν και την περίμετρο του τριγώνου $ΟΚΛ$, όπου $Ο$ η αρχή των αξόνων

       Άξονες, τρίγωνα, αποστάσεις:

       $f(x)=\frac{5}{2}\implies ημ(2x)=\frac{1}{2}\implies 2x=\frac{π}{6}$ ή $x=\frac{5π}{6}$. Το εμβαδό του τριγώνου είναι $\frac{βυ}{2}=\frac{(\frac{5π}{6}-\frac{π}{6})\cdot \frac{5}{2}}{2}=\frac{20π}{24}$. Για το εμβαδό εργαζόμαστε ανάλογα, υπολογίζοντας με Πυθ. Θεώρ. τις 2 πλευρές.

 \item \textbf{[Μονάδες 7]} Να λύσετε την εξίσωση $f(2x)=f(3x)$ στο διάστημα $[0,π]$

       Τριγωνομετρικές εξισώσεις:

       $ημ(4x)+2=ημ(6x)+2\implies 6x=2kπ+4x$ ή $6x=2kπ+π-4x$.

       $2x=2kπ\implies x=κπ$ που στο $[0,π]$ έχει λύση μόνο τις $x=0$ και $x=π$

       ή

       $10x=2kπ+π\implies x=\frac{2k+1}{10}π$. Αλλά στο $[0,π]$ έχουμε μόνο τις $\frac{π}{10}$, $\frac{3π}{10}$, $\frac{5π}{10}$, $\frac{7π}{10}$, $\frac{9π}{10}$

\end{enumerate}

\end{document}
