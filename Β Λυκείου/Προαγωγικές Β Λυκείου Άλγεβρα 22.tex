\documentclass[12pt]{article}

\usepackage{amsmath}
\usepackage{unicode-math}
\usepackage{xltxtra}
\usepackage{xgreek}

\setmainfont{Liberation Serif}

\usepackage{tabularx}

\pagestyle{empty}

\usepackage{geometry}
 \geometry{a4paper, total={190mm,275mm}, left=10mm, top=10mm}

 \usepackage{graphicx}
 \graphicspath{ {images/} }

 \usepackage{wrapfig}

\begin{document}

\begin{table}
    \small
    \begin{tabularx}{\textwidth}{ c X r }
      \begin{tabular}{ c }
        \includegraphics[scale=0.4]{logo} \\
        ΕΛΛΗΝΙΚΗ ΔΗΜΟΚΡΑΤΙΑ \\
        ΥΠΟΥΡΓΕΙΟ ΠΑΙΔΕΙΑΣ \& ΘΡΗΣΚΕΥΜΑΤΩΝ \\
        ΠΕΡΙΦΕΡΕΙΑΚΗ Δ/ΝΣΗ ΠΡΩΤ. \& ΔΕΥΤ/ΜΙΑΣ  ΕΚΠ/ΣΗΣ \\
        ΚΕΝΤΡΙΚΗΣ ΜΑΚΕΔΟΝΙΑΣ \\
        Δ/ΝΣΗ ΔΕΥΤΕΡΟΒΑΘΜΙΑΣ ΕΚΠ/ΣΗΣ ΑΝ. ΘΕΣ/ΝΙΚΗΣ \\
        10ο ΓΕΝΙΚΟ ΛΥΚΕΙΟ ΘΕΣ/ΝΙΚΗΣ
      \end{tabular}
      & &
      \begin{tabular}{ r }
        Σχολικό Έτος: 2021 - 2022 \\
        Εξ. Περίοδος: Μαΐου - Ιουνίου \\
        Μάθημα: Άλγεβρα Β Λυκείου \\
        Εισηγητής: Λόλας \\ \\
        Θεσσαλονίκη, 27 / 05 / 2022
      \end{tabular}
    \end{tabularx}
\end{table}

\part*{\centering{Θέματα}}
\section*{Θέμα Α}
  \noindent
  \begin{enumerate}
    \item \textbf{[Μονάδες 15]} Να αποδείξετε ότι αν $α>0$ με $α \ne 1$, για οποιαδήποτε $θ_1$, $θ_2>0$, ισχύει $$\log_α θ_1+\log_α θ_2=\log_α(θ_1 θ_2)$$
    \item \textbf{[Μονάδες 10]}  Να χαρακτηρίσετε τις παρακάτω προτάσεις με Σωστό ή Λάθος
    \begin{enumerate}
      \item [α)] Ένα μη γραμμικό σύστημα μπορεί να έχει τρεις λύσεις.
      \item [β)] Αν ένα πολυώνυμο διαιρείται ακριβώς με το $\left(x-\frac{3}{2}\right)$ τότε δεν έχει ακέραιες ρίζες.
      \item [γ)] Η συνάρτηση $f(x)=\frac{1}{2^x}$ είναι γνησίως αύξουσα.
      \item [δ)] Για κάθε $a$, $b>0$ ισχύει $\ln a \cdot \ln b = \ln(a+b)$.
      \item [ε)] Για κάθε πολυώνυμο $P(x)$ με $P(\sqrt{2})=\sqrt{2}$ το πολυώνυμο έχει ρίζα το $2$.
    \end{enumerate}
  \end{enumerate}

\section*{Θέμα Β (21472)}
  \noindent
\begin{enumerate}
  \item \textbf{[Μονάδες 13]}  Να λύσετε την εξίσωση: $\ln(x+1)=\ln(2x)$
  \item \textbf{[Μονάδες 12]}  Να λύσετε την ανίσωση: $\ln(x+1)>\ln(2x)$
\end{enumerate}

\section*{Θέμα Γ}
  \noindent

  Δίνεται η παράσταση $Α=\frac{συνx}{1-ημx}+\frac{συνx}{1+ημx}$.
  \begin{enumerate}
    \item \textbf{[Μονάδες 12]}  Να δείξετε ότι $Α=\frac{2}{συνx}$
    \item \textbf{[Μονάδες 13]}  Να λύσετε την εξίσωση $\frac{Α}{2}=2συνx-1$.
  \end{enumerate}

\section*{Θέμα Δ (14973)}
  \noindent

  Δίνονται οι συναρτήσεις $φ(x)=3x^2$, $x\in\mathbb{R}$ και $f(x)=3x^2-6x+8$, $x\in\mathbb{R}$.

  \begin{enumerate}
    \item \textbf{[Μονάδες 4]}  Να ελέγξετε αν η συνάρτηση $φ$ είναι άρτια ή περιττή και να σχεδιάσετε τη γραφική της παράσταση.
    \item \textbf{[Μονάδες 4]}  Να αποδείξετε ότι $f(x)=3(x-1)^2+5$, $x\in\mathbb{R}$. Στη συνέχεια, με τη βοήθεια της γραφικής παράστασης της συνάρτησης $φ$, να παραστήσετε γραφικά τη συνάρτηση $f$, αιτιολογώντας την απάντησή σας.
    \item Με τη βοήθεια της γραφικής παράστασης της συνάρτησης $f$, να βρείτε:
    \begin{enumerate}
        \item \textbf{[Μονάδες 6]}  Τα διστήματα στα οποία η $f$ είναι γνήσια μονότονη και τον άξονα συμμετρίας της συνάρτησης $f$.
        \item \textbf{[Μονάδες 4]}  Το ολικό ακρότατο της $f$ και τη θέση του. Τι είδος ακρότατο είναι;
        \item \textbf{[Μονάδες 7]}  Το πλήθος των κοινών σημείων της γραφικής παράστασης της $f$ και της ευθείας με εξίσωση $y=λ$, $λ\in\mathbb{R}$, για τις διάφορες τιμές του πραγματικού αριθμού $λ$.
      \end{enumerate}
  \end{enumerate}

\part*{\centering{Καλή επιτυχία}}
\begin{table}[htb]
    \begin{tabularx}{\textwidth}{ X c X c X}
      &
      \begin{tabular}[t]{ c }
        Ο Δ/ντης \\ \\ \\ \\
        Παπαδημητρίου Χρήστος
      \end{tabular}
      & &
      \begin{tabular}[t]{ c }
        Ο εισηγητής \\ \\ \\ \\
        Λόλας Κωνσταντίνος
      \end{tabular}
      &
    \end{tabularx}
\end{table}


\vfill
 \textbf{Οδηγίες}
 \begin{enumerate}
   \item Μην ξεχάσετε να γράψετε το ονοματεπώνυμό σας σε κάθε φύλλο απαντήσεων που σας δώσουν.
   \item Όλες οι απαντήσεις να δωθούν στο φύλλο απαντήσεων. Οτιδήποτε γραφτεί στη σελίδα με τα θέματα δεν θα ληφθεί υπόψιν.
   \item Τα Σωστό - Λάθος δεν χρειάζονται αιτιολόγηση.
 \end{enumerate}
\end{document}
