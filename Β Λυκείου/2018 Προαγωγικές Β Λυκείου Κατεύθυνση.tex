\documentclass[12pt]{article}

\usepackage{amsmath}
\usepackage{unicode-math}
\usepackage{xltxtra}
\usepackage{xgreek}

\setmainfont{Liberation Serif}

\usepackage{tabularx}

\pagestyle{empty}

\usepackage{geometry}
 \geometry{a4paper, total={190mm,280mm}, left=10mm, top=10mm}

 \usepackage{graphicx}
 \graphicspath{ {images/} }

 \usepackage{wrapfig}
\usepackage{lipsum}%% a garbage package you don't need except to create examples.

\begin{document}

\begin{table}
    \small
    \begin{tabularx}{\textwidth}{ c X r }
      \begin{tabular}{ c }
        \includegraphics[scale=0.4]{logo} \\
        ΕΛΛΗΝΙΚΗ ΔΗΜΟΚΡΑΤΙΑ \\
        ΥΠΟΥΡΓΕΙΟ ΠΑΙΔΕΙΑΣ, ΕΡΕΥΝΑΣ \& ΘΡΗΣΚΕΥΜΑΤΩΝ \\
        ΠΕΡΙΦΕΡΕΙΑΚΗ Δ/ΝΣΗ ΠΡΩΤ. \& ΔΕΥΤ/ΜΙΑΣ  ΕΚΠ/ΣΗΣ \\
        ΚΕΝΤΡΙΚΗΣ ΜΑΚΕΔΟΝΙΑΣ \\
        Δ/ΝΣΗ ΔΕΥΤΕΡΟΒΑΘΜΙΑΣ ΕΚΠ/ΣΗΣ ΑΝ. ΘΕΣ/ΝΙΚΗΣ \\
        10ο ΓΕΝΙΚΟ ΛΥΚΕΙΟ ΘΕΣ/ΝΙΚΗΣ
      \end{tabular}
      & &
      \begin{tabular}{ r }
        Σχολικό Έτος: 2017 - 2018 \\
        Εξ. Περίοδος: Μαΐου - Ιουνίου \\
        Μάθημα: Μαθηματικά Κατεύθυνσης Β Λυκείου\\
        Εισηγητής: Λόλας \\ \\
        Θεσσαλονίκη, 12 / 06 / 2018
      \end{tabular}
    \end{tabularx}
\end{table}

\part*{\centering{Θέματα}}

\section*{Θέμα Α}
  \noindent
  \begin{enumerate}
    \item \textbf{[Μονάδες 10]} Να αποδείξετε ότι το μέσο ενός ευθύγραμμου τμήματος με άκρα τα σημεία $Α=(x_1,y_1)$ και $Β=(x_2,y_2)$ είναι το
    $$Μ=(\frac{x_1+x_2}{2},\frac{y_1+y_2}{2}) \text{.}$$
    \item \textbf{[Μονάδες 5]} Να ορίσετε την απόσταση των σημείων $Α=(x_1,y_1)$ και $Β=(x_2,y_2)$ συναρτήσει των συντεταγμένων τους.
    \item \textbf{[Μονάδες 10]} Να χαρακτηρίσετε τις παρακάτω προτάσεις με Σωστό ή Λάθος
    \begin{enumerate}
      \item [α)] Όλες οι ευθείες είναι της μορφής $y=αx+β$.
      \item [β)] Η απόσταση ενός σημείου $(x_0,y_0)$ από την ευθεία $Αx+Βy+Γ=0$ δίνεται από τον τύπο $d=\frac{Αx_0+Βy_0+Γ}{\sqrt{Α^2+Β^2}}$.
      \item [γ)] Αν τα διανύσματα $\vec{α}$ και $\vec{β}$ είναι παράλληλα τότε $\vec{α}\cdot\vec{β}=|\vec{α}||\vec{β}|$.
      \item [δ)] Ισχύει πάντα $\vec{α}^2=|\vec{α}|^2=|-\vec{α}|^2$.
      \item [ε)] Το διάνυσμα $\frac{\vec{α}}{|\vec{α}|}$ είναι το μοναδιαίο διάνυσμα στη κατεύθυνση του $\vec{α}$
    \end{enumerate}
  \end{enumerate}

\section*{Θέμα Β}
  \noindent
  Δίνονται τα διανύσματα $\vec{α}=(1,2)$ και $\vec{β}=(-2,κ)$ και το σημείο $Δ=(2,1)$
  \begin{enumerate}
    \item \textbf{[Μονάδες 5]} Να βρεθεί το σημείο $Γ$ ώστε $\vec{α}=\overrightarrow{ΓΔ}$.
    \item \textbf{[Μονάδες 5]} Να βρεθεί το $κ$ ώστε τα διανύσματα να είναι παράλληλα.
    \item \textbf{[Μονάδες 5]} Να βρεθεί το $κ$ ώστε τα διανύσματα να είναι κάθετα.

    Αν $κ=3$
    \item \textbf{[Μονάδες 5]} Να βρεθεί το $συν\widehat{(\vec{α},\vec{β})}$.
    \item \textbf{[Μονάδες 5]} Να γραφτεί το $\vec{γ}=(-1,5)$ ως γραμμικός συνδιασμός των $\vec{α}$ και $\vec{β}$.

  \end{enumerate}

\section*{Θέμα Γ}
  \noindent
  Δίνονται τα σημεία $Ο=(0,0)$, $Α=(1,-5)$ και $Β=(2,2)$
  \begin{enumerate}
    \item \textbf{[Μονάδες 5]} Να δειχθεί ότι το $ΟΑΒ$ είναι τρίγωνο.
    \item \textbf{[Μονάδες 5]} Να δειχθεί ότι η εξίσωση της ευθείας $ΑΒ$ είναι η $x-7y+12=0$.
    \item \textbf{[Μονάδες 5]} Να βρεθεί η εξίσωση της μεσοκαθέτου του $ΑΒ$.
    \item \textbf{[Μονάδες 5]} Να δειχθεί ότι η εξίσωση της διχοτόμου της γωνίας $\widehat{ΑΟΒ}$ είναι η $3x+10y=0$.
    \item \textbf{[Μονάδες 5]} Να βρεθεί το εμβαδό του τριγώνου $ΟΑΒ$.

  \end{enumerate}

\section*{Θέμα Δ}
  \noindent
  Έστω η εξίσωση $ x^2+y^2+2λx+λy-15=0 $.
  \begin{enumerate}
    \item \textbf{[Μονάδες 5]} Να βρείτε τις τιμές του $λ$ ώστε η εξίσωση να παριστάνει κύκλο.
    \item \textbf{[Μονάδες 5]} Να βρείτε τον γεωμετρικό τόπο των κέντρων των κύκλων του προηγούμενου ερωτήματος.

    Για $λ=-2$,
    \item \textbf{[Μονάδες 5]} Να δείξετε ότι το σημείο $(6,4)$ είναι εξωτερικό του κύκλου.
    \item \textbf{[Μονάδες 5]} Να βρείτε τις εφαπτομένες του κύκλου που διέρχονται από το σημείο $(6,4)$.
    \item \textbf{[Μονάδες 5]} Να βρείτε την ελάχιστη απόσταση του σημείου $(6,4)$ από τον κύκλο.

  \end{enumerate}

\vspace{3\baselineskip}

\part*{\centering{Καλή επιτυχία}}
\begin{table}[htb]
    \begin{tabularx}{\textwidth}{ X c X c X}
      &
      \begin{tabular}[t]{ c }
        Ο Δ/ντης
        \\ \\ \\ \\ \\
        Παπαδημητρίου Χρήστος
      \end{tabular}
      & &
      \begin{tabular}[t]{ c }
        Ο εισηγητής
        \\ \\ \\ \\ \\
        Λόλας Κωνσταντίνος
      \end{tabular}
      &
    \end{tabularx}
\end{table}

\vspace*{\fill}
 \textbf{Οδηγίες}
 \begin{enumerate}
   \item Να απαντήσετε σε όλα τα θέματα
   \item Μην ξεχάσετε να γράψετε το ονοματεπώνυμό σας σε κάθε φύλλο που σας δώσουν.
   \item Όλες οι απαντήσεις να δωθούν στο φύλλο απαντήσεων. Οτιδήποτε γραφτεί στη σελίδα με τα θέματα δεν θα ληφθεί υπόψιν.
   \item Τα Σωστό - Λάθος δεν χρειάζονται αιτιολόγηση.
 \end{enumerate}
\end{document}
