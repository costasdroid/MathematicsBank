\documentclass[12pt]{article}

\usepackage{amsmath}
\usepackage{unicode-math}
\usepackage{xltxtra}
\usepackage{xgreek}

\setmainfont{Liberation Serif}

\usepackage{tabularx}

\pagestyle{empty}

\usepackage{geometry}
 \geometry{a4paper, total={190mm,275mm}, left=10mm, top=10mm}

 \usepackage{graphicx}
 \graphicspath{ {images/} }

 \usepackage{wrapfig}
\usepackage{lipsum}%% a garbage package you don't need except to create examples.

\begin{document}

\begin{table}
    \small
    \begin{tabularx}{\textwidth}{ c X r }
      \begin{tabular}{ c }
        \includegraphics[scale=0.4]{logo} \\
        ΕΛΛΗΝΙΚΗ ΔΗΜΟΚΡΑΤΙΑ \\
        ΥΠΟΥΡΓΕΙΟ ΠΑΙΔΕΙΑΣ, ΕΡΕΥΝΑΣ \& ΘΡΗΣΚΕΥΜΑΤΩΝ \\
        ΠΕΡΙΦΕΡΕΙΑΚΗ Δ/ΝΣΗ ΠΡΩΤ. \& ΔΕΥΤ/ΜΙΑΣ  ΕΚΠ/ΣΗΣ \\
        ΚΕΝΤΡΙΚΗΣ ΜΑΚΕΔΟΝΙΑΣ \\
        Δ/ΝΣΗ ΔΕΥΤΕΡΟΒΑΘΜΙΑΣ ΕΚΠ/ΣΗΣ ΑΝ. ΘΕΣ/ΝΙΚΗΣ \\
        27ο ΓΕΝΙΚΟ ΛΥΚΕΙΟ ΘΕΣ/ΝΙΚΗΣ
      \end{tabular}
      & &
      \begin{tabular}{ r }
        Σχολικό Έτος: 2016 - 2017 \\
        Εξ. Περίοδος: Μαΐου - Ιουνίου \\
        Μάθημα: Άλγεβρα Β Λυκείου\\
        Εισηγητές: Λόλας, Φρύδας, Τερζόγλου \\ \\
        Θεσσαλονίκη, 19 / 05 / 2017
      \end{tabular}
    \end{tabularx}
\end{table}

\part*{\centering{Θέματα}}

\section*{Θέμα Α}
  \noindent
  \begin{enumerate}
    \item \textbf{[Μονάδες 15]} Να αποδείξετε ότι αν $α>0$ με $α \ne 1$, για οποιαδήποτε $θ_1$, $θ_2>0$, ισχύει $$\log_α θ_1+\log_α θ_2=\log_α(θ_1 θ_2)$$
    \item \textbf{[Μονάδες 10]}  Να χαρακτηρίσετε τις παρακάτω προτάσεις με Σωστό ή Λάθος
    \begin{enumerate}
      \item [α)] Ένα μη γραμμικό σύστημα μπορεί να έχει τρεις λύσεις.
      \item [β)] Αν ένα πολυώνυμο διαιρείται ακριβώς με το $\left(x-\frac{3}{2}\right)$ τότε δεν έχει ακέραιες ρίζες.
      \item [γ)] Η συνάρτηση $f(x)=\frac{1}{2^x}$ είναι γνησίως αύξουσα.
      \item [δ)] Για κάθε $a$, $b>0$ ισχύει $\ln a \cdot \ln b = \ln(a+b)$.
      \item [ε)] Για κάθε πολυώνυμο $P(x)$ με $P(\sqrt{2})=\sqrt{2}$ το πολυώνυμο έχει ρίζα το $2$.
    \end{enumerate}
  \end{enumerate}

\section*{Θέμα Β}
  \noindent
  Δίνεται το σύστημα $$\begin{cases}x^2+y^2=2+k \\ x^2+y=5\end{cases}$$
  \begin{enumerate}
    \item \textbf{[Μονάδες 10]}  Να βρείτε την τιμή του $k$ ώστε το σύστημα να έχει για μία λύση την $(2,1)$.
    \item \textbf{[Μονάδες 15]}  Να βρείτε τις υπόλοιπες λύσεις.
  \end{enumerate}

\section*{Θέμα Γ}
  \noindent
  Δίνεται το πολυώνυμο $P(x)=x^4+4x^3-x^2-8x+4$.
  \begin{enumerate}
    \item \textbf{[Μονάδες 3]} Να γράψετε όλες τις υποψήφιες ακέραιες ρίζες του πολυωνύμου.
    \item \textbf{[Μονάδες 8]} Να βρείτε το υπόλοιπο της διαίρεσης του $P(x)$ με το $(x+1)$.
    \item \textbf{[Μονάδες 8]} Να βρείτε όλες τις ρίζες του πολυωνύμου.
    \item \textbf{[Μονάδες 6]} Να λύσετε την ανίσωση $P(x) \ge 0$.
  \end{enumerate}

\section*{Θέμα Δ}
  \noindent
  Δίνεται η συνάρτηση $f(x)=\ln\left(\frac{2}{\ln x - 1}\right)$
  \begin{enumerate}
    \item \textbf{[Μονάδες 8]}  Να βρείτε το πεδίο ορισμού της.
    \item \textbf{[Μονάδες 8]}  Να δείξετε ότι $f(e^{1+2e})=-1$.
    \item \textbf{[Μονάδες 9]}  Να λύσετε την ανίσωση $f(x)<0$.
  \end{enumerate}

\part*{\centering{Καλή επιτυχία}}
\begin{table}[htb]
    \begin{tabularx}{\textwidth}{ X c X c X}
      &
      \begin{tabular}[t]{ c }
        Ο Δ/ντης \\ \\ \\ \\
        Δρ. Ιωαννίδης Νικόλαος
      \end{tabular}
      & &
      \begin{tabular}[t]{ c }
        Οι εισηγητές \\ \\
        \multicolumn{1}{l}{1. Λόλας Κωνσταντίνος} \\ \\
        \multicolumn{1}{l}{2. Φρύδας Βασίλειος} \\ \\
        \multicolumn{1}{l}{3. Τερζόγλου Ιωάννης}
      \end{tabular}
      &
    \end{tabularx}
\end{table}

\vfill
 \textbf{Οδηγίες}
 \begin{enumerate}
   \item Μην ξεχάσετε να γράψετε το ονοματεπώνυμό σας σε κάθε φύλλο απαντήσεων που σας δώσουν.
   \item Όλες οι απαντήσεις να δωθούν στο φύλλο απαντήσεων. Οτιδήποτε γραφτεί στη σελίδα με τα θέματα δεν θα ληφθεί υπόψιν.
   \item Τα Σωστό - Λάθος δεν χρειάζονται αιτιολόγηση.
 \end{enumerate}
\end{document}
