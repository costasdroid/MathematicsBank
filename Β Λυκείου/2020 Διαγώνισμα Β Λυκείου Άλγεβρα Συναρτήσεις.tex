\documentclass[12pt]{article}

\usepackage{amsmath}
\usepackage{unicode-math}
\usepackage{xltxtra}
\usepackage{xgreek}

\setmainfont{Liberation Serif}

\usepackage{tabularx}

\pagestyle{empty}

\usepackage{geometry}
 \geometry{a4paper, total={190mm,280mm}, left=10mm, top=10mm}

 \usepackage{graphicx}
 \graphicspath{ {images/} }

 \usepackage{wrapfig}

\begin{document}

\begin{table}
 \small
 \begin{tabularx}{\textwidth}{ c X r }
  \begin{tabular}{ l }
   Εισηγητής: Λόλας Κωνσταντίνος \\
   Τεστ: Συναρτήσεις
  \end{tabular}
   &  &
  \begin{tabular}{ r }
   Θεσσαλονίκη, 22 / 10 / 2020
  \end{tabular}
 \end{tabularx}
\end{table}

\part*{\centering{Τεστ Β Λυκείου Άλγεβρα}}

\section*{Θέμα Α}
  Δίνεται η συνάρτηση $f(x)=(x+1)\sqrt{x}$, $x>0$.
  \begin{enumerate}
   \item Να αποδείξετε ότι η συνάρτηση είναι γνησίως αύξουσα
   \item Να λύσετε την ανίσωση $f(x)>10$
   \item Να αποδείξετε ότι αν $0<α<β$ τότε $$(α+1)\sqrt{α}<(β+1)\sqrt{β}$$
  \end{enumerate}

\section*{Θέμα Β}
  Δίνονται οι συναρτήσεις:
  $$f(x)=x^2-10|x|+26 \text{, } x\in \mathbb{R}$$
  και
  $$g(x)=\frac{10x}{x^2+25}$$
  \begin{enumerate}
   \item Να αποδείξετε ότι η συνάρτηση $f$ είναι άρτια, ενώ η συνάρτηση $g$ είναι περιττή.
   \item Να αποδείξετε ότι ο αριθμός 1 είναι ολικό ελάχιστο της συνάρτησης $f$.
   \item Να αποδείξετε ότι η συνάρτηση $g$ παρουσιάζει ολικό μέγιστο στο $x_0=5$.
   \item Να λύσετε την εξίσωση $f(x)=g(x)$.
  \end{enumerate}

\section*{Θέμα Γ}
  Δίνεται η συνάρτηση $f(x)=3x^2-6x+7$, $x\in\mathbb{R}$.
  \begin{enumerate}
   \item Να γράψετε την συνάρτηση στη μορφή $f(x)=α(x-p)^2+q$
   \item Να αποδείξετε ότι η συνάρτηση έχει ολικό ελάχιστο
   \item Αν η συνάρτηση $g(x)=3x^2$ είναι αυτή του σχήματος, να σχεδιάσετε στο ίδιο γράφημα με μολύβι την συνάρτηση $f$ στο φύλλο απάντησής σας
  \end{enumerate}

\vspace{\baselineskip}

\section*{\centering{Καλή επιτυχία}}

\vspace{\baselineskip}

\includegraphics[scale=1.5]{"2020 Διαγώνισμα Β Λυκείου Άλγεβρα Συναρτήσεις.png"}

\end{document}
