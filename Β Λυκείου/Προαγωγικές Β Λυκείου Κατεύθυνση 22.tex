\documentclass[12pt]{article}

\usepackage{amsmath}
\usepackage{unicode-math}
\usepackage{xltxtra}
\usepackage{xgreek}

\setmainfont{Liberation Serif}

\usepackage{tabularx}

\pagestyle{empty}

\usepackage{geometry}
 \geometry{a4paper, total={190mm,280mm}, left=10mm, top=10mm}

 \usepackage{graphicx}
 \graphicspath{ {images/} }

 \usepackage{wrapfig}
\usepackage{lipsum}%% a garbage package you don't need except to create examples.

\begin{document}
\begin{table}
 \small
 \begin{tabularx}{\textwidth}{ c X r }
  \begin{tabular}{ c }
   \includegraphics[scale=0.4]{logo}              \\
   ΕΛΛΗΝΙΚΗ ΔΗΜΟΚΡΑΤΙΑ                            \\
   ΥΠΟΥΡΓΕΙΟ ΠΑΙΔΕΙΑΣ, ΕΡΕΥΝΑΣ \& ΘΡΗΣΚΕΥΜΑΤΩΝ    \\
   ΠΕΡΙΦΕΡΕΙΑΚΗ Δ/ΝΣΗ ΠΡΩΤ. \& ΔΕΥΤ/ΜΙΑΣ  ΕΚΠ/ΣΗΣ \\
   ΚΕΝΤΡΙΚΗΣ ΜΑΚΕΔΟΝΙΑΣ                           \\
   Δ/ΝΣΗ ΔΕΥΤΕΡΟΒΑΘΜΙΑΣ ΕΚΠ/ΣΗΣ ΑΝ. ΘΕΣ/ΝΙΚΗΣ     \\
   10ο ΓΕΝΙΚΟ ΛΥΚΕΙΟ ΘΕΣ/ΝΙΚΗΣ
  \end{tabular}
   &  &
  \begin{tabular}{ r }
   Σχολικό Έτος: 2021 - 2022     \\
   Εξ. Περίοδος: Μαΐου - Ιουνίου \\
   Μάθημα: Άλγεβρα Α Λυκείου     \\
   Εισηγητές: Λόλας Κ., Αδάμ Μ.  \\ \\
   Θεσσαλονίκη, 01 / 06 / 2022
  \end{tabular}
 \end{tabularx}
\end{table}

\part*{\centering{Θέματα}}

\section*{Θέμα 1}
\noindent
\begin{enumerate}
 \item \textbf{[Μονάδες 15]} Να αποδείξετε ότι το μέσο ενός ευθύγραμμου τμήματος με άκρα τα σημεία $Α=(x_1,y_1)$ και $Β=(x_2,y_2)$ είναι το
       $$Μ=(\frac{x_1+x_2}{2},\frac{y_1+y_2}{2}) \text{.}$$
 \item \textbf{[Μονάδες 10]} Να χαρακτηρίσετε τις παρακάτω προτάσεις με Σωστό ή Λάθος
       \begin{enumerate}
        \item [α)] Όλες οι ευθείες είναι της μορφής $y=αx+β$.
        \item [β)] Η απόσταση ενός σημείου $(x_0,y_0)$ από την ευθεία $Αx+Βy+Γ=0$ δίνεται από τον τύπο $d=\frac{Αx_0+Βy_0+Γ}{\sqrt{Α^2+Β^2}}$.
        \item [γ)] Αν τα διανύσματα $\vec{α}$ και $\vec{β}$ είναι παράλληλα τότε $\vec{α}\cdot\vec{β}=|\vec{α}||\vec{β}|$.
        \item [δ)] Ισχύει πάντα $\vec{α}^2=|\vec{α}|^2=|-\vec{α}|^2$.
        \item [ε)] Η εξίσωση $(x-x_0)^2+(y-y_0)^2=ρ^2$ παριστάνει κύκλο με κέντρο $Κ(x_0,y_0)$ και ακτίνα $ρ$
       \end{enumerate}
\end{enumerate}

\section*{Θέμα 2 (16580)}
\noindent
Σε καρτεσιανό επίπεδο $Οxy$ δίνονται τα σημεία $Α(2,4)$, $Β(11,5)$, $Γ(3,7)$ και ένα σημείο $Δ$ ώστε το $\overrightarrow{ΑΔ}$ να είναι ίσο με το άθροισμα των $\overrightarrow{ΑΒ}$ και $\overrightarrow{ΑΓ}$

Να υπολογίσετε τις συντεταγμένες:


\begin{enumerate}
 \item [α)] \textbf{[Μονάδες 12]} των διανυσμάτων $\overrightarrow{ΑΒ}$ και $\overrightarrow{ΑΓ}$
 \item [β)] \textbf{[Μονάδες 08]} του διανύσματος $\overrightarrow{ΑΔ}$
 \item [γ)] \textbf{[Μονάδες 05]} του σημείου $Δ$
\end{enumerate}
\includegraphics[scale=0.25]{"2022.png"}

\section*{Θέμα 3}
\noindent
Θεωρούμε τα σιανύσματα $\vec{α}$, $\vec{β}$ με $|\vec{α}|=2$, $|\vec{β}|=4$, $(\vec{α},\vec{β})=\frac{π}{3}$ και τα σιανύσματα $\vec{γ}=\vec{α}-\vec{β}$ και $\vec{δ}=2\vec{α}+\vec{β}$
\begin{enumerate}
 \item \textbf{[Μονάδες 10]} Να βρείτε το $\vec{α}\cdot\vec{β}$
 \item \textbf{[Μονάδες 10]} Να βρείτε το $\vec{γ}\cdot\vec{δ}$
 \item \textbf{[Μονάδες 5]} Να βρείτε τα $|\vec{γ}|$, $|\vec{δ}|$
 \item \textbf{[Μονάδες 5]} Να βρείτε τη γωνία $(\vec{γ},\vec{δ})$
\end{enumerate}

\section*{Θέμα 4 (21349)}
\noindent
Σε ορθοκανονικό σύστημα αξόνων με αρχή το σημείο $Ο$ θεωρούμε κύκλο (C) και ευθεία (ε) με εξισώσεις  $x^2+y^2-9x-3x+10=0$ (1) kai $4x+3y-10=0$ (2) αντίστοιχα.
\begin{enumerate}
 \item [α)]
 \begin{enumerate}
  \item [(i)]  \textbf{[Μονάδες 5]} Να βρείτε το κέντρο $Κ$ και την ακτίνα $R$ του κύκλου (C).
  \item [(ii)] \textbf{[Μονάδες 4]} Να υπολογίσετε την απόσταση του κέντρου $Κ$ από την ευθεία (ε) και να αποδείξετε ότι η ευθεία (ε) τέμνει τον κύκλο (C) σε δύο σημεία.
 \end{enumerate}

  \item [β)]  Αν είναι $Α(1,2)$ και $Β(4,-2)$, τότε:
  \begin{enumerate}
   \item [(i)]  \textbf{[Μονάδες 5]} Να υπολογίσετε το εσωτερικό γινόμενο $\overrightarrow{ΟΑ}\cdot\\overrightarrow{ΟΒ}$.
   \item [(ii)] \textbf{[Μονάδες 6]} Να αποδείξετε ότι ο κύκλος με διάμετρο $ΑΒ$ διέρχεται από το σημείο $Ο$.
  \end{enumerate}

\end{enumerate}


\vspace{3\baselineskip}

\part*{\centering{Καλή επιτυχία}}
\begin{table}[htb]
 \begin{tabularx}{\textwidth}{ X c X c X}
   &
  \begin{tabular}[t]{ c }
   Ο Δ/ντης
   \\ \\ \\ \\ \\
   Παπαδημητρίου Χρήστος
  \end{tabular}
   &   &
  \begin{tabular}[t]{ c }
   Οι εισηγητές                              \\ \\
   \multicolumn{1}{l}{1. Λόλας Κωνσταντίνος} \\ \\
   \multicolumn{1}{l}{2. Αδάμ Μιλτιάδης}
  \end{tabular}
   &
 \end{tabularx}
\end{table}

\vspace*{\fill}
\textbf{Οδηγίες}
\begin{enumerate}
 \item Να απαντήσετε σε όλα τα θέματα
 \item Μην ξεχάσετε να γράψετε το ονοματεπώνυμό σας σε κάθε φύλλο που σας δώσουν.
 \item Όλες οι απαντήσεις να δωθούν στο φύλλο απαντήσεων. Οτιδήποτε γραφτεί στη σελίδα με τα θέματα δεν θα ληφθεί υπόψιν.
 \item Τα Σωστό - Λάθος δεν χρειάζονται αιτιολόγηση.
\end{enumerate}
\end{document}
