\documentclass[12pt]{article}

\usepackage{amsmath}
\usepackage{unicode-math}
\usepackage{xltxtra}
\usepackage{xgreek}

\setmainfont{Liberation Serif}

\usepackage{tabularx}

\pagestyle{empty}

\usepackage{geometry}
 \geometry{a4paper, total={190mm,280mm}, left=10mm, top=10mm}

 \usepackage{graphicx}
 \graphicspath{ {images/} }

 \usepackage{wrapfig}

\begin{document}

\begin{table}
    \small
    \begin{tabularx}{\textwidth}{ c X r }
      \begin{tabular}{ l }
        Εισηγητής: Λόλας Κωνσταντίνος \\
        Επαναληπτικό: Συναρτήσεις
      \end{tabular}
      & &
      \begin{tabular}{ r }
        Θεσσαλονίκη, 22 / 10 / 2019
      \end{tabular}
    \end{tabularx}
\end{table}

\part*{\centering{Διαγώνισμα Κατεύθυνση Γ Λυκείου}}

\section*{Θέμα Α}
  \noindent
  \begin{enumerate}
    \item \textbf{[Μονάδες 10]} Πότε μία συνάρτηση $f$ λέγεται γνησίως φθίνουσα σ' ένα διάστημα $Δ$ του πεδίου ορισμού της;
    \item \textbf{[Μονάδες 5]} Έστω μία συνάρτηση $f$ με πεδίο ορισμού $Α$. Πότε λέμε ότι η $f$ παρουσιάζει στο $x_0\in Α$ (ολικό) μέγιστο το $f(x_0)$;
    \item \textbf{[Μονάδες 10]} Πότε δύο συναρτήσεις $f$, $g$ λέγονατι ίσες;
  \end{enumerate}

\section*{Θέμα Β}
  \noindent
  Δίνεται η συνάρτηση $f(x)=κ-e^{2-x}+x$, $k\in \mathbb{R}$.
  \begin{enumerate}
    \item \textbf{[Μονάδες 6]} Να δείξετε ότι η $f$ αντιστρέφεται.
    \item \textbf{[Μονάδες 6]} Αν $f^{-1}(2)=0$, να βρείτε την τική του $κ$.

  Για $κ=2+e^2$,

    \item \textbf{[Μονάδες 6]} Να βρείτε τα κοινά σημεία της γραφική παράστασης της συνάρτησης $f^{-1}$ με την ευθεία $y=x$, αν θεωρήσουμε γνωστό ότι $f(\mathbb{R})=\mathbb{R}$ και ότι η $f$ και η $f^{-1}$ έχουν κοινά σημεία μόνο στην $y=x$.
    \item \textbf{[Μονάδες 7]} Να λύσετε την ανίσωση $e^{2-x}<x+e^2$.
  \end{enumerate}
  %\vspace{7\baselineskip}

\section*{Θέμα Γ}
  \noindent
  Έστω $f$, $g:\mathbb{R}\to \mathbb{R}$ δύο συναρτήσεις, για τις οποίες ισχύει ότι η συνάρτηση $f\circ g$ είναι 1-1.
  \begin{enumerate}
    \item \textbf{[Μονάδες 5]} Να δείξετε ότι η $g$ είναι 1-1.
    \item \textbf{[Μονάδες 7]} Να λύσετε την εξίσωση $g\left( f(x)+x^3+x\right)=g\left( f(x)-\ln x+ 2\right)$.
    \item \textbf{[Μονάδες 5]} Αν $g(\mathbb{R})=(0,+\infty)$, να δείξετε ότι η εξίσωση $ae^{g(x)}=1$, έχει μοναδική λύση για κάθε $a\in (0,1)$.
  \end{enumerate}

  \section*{Θέμα Δ}
    \noindent
    Δίνεται η συνάρτηση $f(x)=\frac{1}{x\ln x}$
    \begin{enumerate}
      \item \textbf{[Μονάδες 5]} Να βρείτε το πεδίο ορισμού της $f$.
      \item \textbf{[Μονάδες 5]} Να μελετήσετε τη συνάρτηση $f$ ως προς την μονοτονία στο διάστημα $(1,+\infty)$
      \item \textbf{[Μονάδες 5]} Για κάθε $α$, $β\in (1,+\infty)$ με $α<β$, να αποδείξετε ότι
$$α^α<β^β$$
      \item \textbf{[Μονάδες 5]} Να λύσετε την εξίσωση $\frac{(x^4+2)^{x^4+2}}{(x^2+4)^{x^2+4}}=1$.
    \end{enumerate}

\vspace{3\baselineskip}

\part*{\centering{Καλή επιτυχία}}

\end{document}
