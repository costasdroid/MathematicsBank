\documentclass[14pt]{extarticle}

\usepackage{amsmath}
\usepackage{unicode-math}
\usepackage{xgreek}
\usepackage{xltxtra}

\setmainfont{Liberation Serif}

\usepackage{tabularx}

\pagestyle{empty}

\usepackage{geometry}
 \geometry{a4paper, total={190mm,275mm}, left=10mm, top=10mm}

\binoppenalty=10000
\relpenalty=10000

\begin{document}

\part*{\centering{Θέματα}}
\section*{Θέμα Α (14)}
  \noindent
  \begin{enumerate}
    \item \textbf{[Μονάδες 7]} Έστω δύο συναρτήσεις $f$, $g$ ορισμένες σε ένα διάστημα $Δ$. Αν οι $f$, $g$ είναι συνεχείς στο $Δ$ και $f'(x)=g'(x)$ για κάθε εσωτερικό σημείο $x$ του $Δ$, τότε να αποδείξετε ότι υπάρχει σταθερά $c\in \mathbb{R}$ τέτοια ώστε για κάθε $x\in Δ$ να ισχύει: $f(x)=g(x)+c$.
    \item \textbf{[Μονάδες 4]} Θεωρήστε τον παρακάτω ισχυρισμό:
      \begin{center}
        "Κάθε συνάρτηση $f$ συνεχής σε ένα σημείο $x_0\in \mathbb{R}$ είναι παραγωγίσιμη στο $x_0$."
      \end{center}
      \begin{enumerate}
        \item [α)] \textbf{[Μονάδες 1]} Να χαρακτηρίσετε τον παραπάνω ισχυρισμό με Α (αληθής) ή Ψ (ψευδής).
        \item [β)] \textbf{[Μονάδες 3]} Να αιτιολογήσετε την απάντησή σας.
      \end{enumerate}
    \item \textbf{[Μονάδες 4]}  Να γράψετε στο τετράδιό σας το γράμμα που αντιστοιχεί στη φράση η οποία συμπληρώνει σωστά την ημιτελή πρόταση:
      \begin{flushleft}
        Για κάθε συνεχή συνάρτηση $f:[α,β]\to \mathbb{R}$, αν ισχύει $f(α)f(β)>0$ τότε
      \end{flushleft}
      \begin{enumerate}
        \item [α)] Η εξίσωση $f(x)=0$ δεν έχει λύση στο $(α,β)$.
        \item [β)] Η εξίσωση $f(x)=0$ έχει ακριβώς μια λύση στο $(α,β)$.
        \item [γ)] Η εξίσωση $f(x)=0$ έχει τουλάχισοτν δύο λύσεις στο $(α,β)$.
        \item [δ)] Δεν μπορούμε να έχουμε συμπεράσματα για το πλήθος των λύσεων της εξίσωσης $f(x)=0$ στο $(α,β)$.
      \end{enumerate}
    \item \textbf{[Μονάδες 10]} Να χαρακτηρίσετε τις προτάσεις που ακολουθούν, γράφοντας στο τετράδιό σας, δίπλα στο γράμμα που αντιστοιχεί σε κάθε πρόταση, τη λέξη Σωστό, αν η πρόταση είναι σωστή, ή Λάθος, αν η πρόταση είναι λανθασμένη.
      \begin{enumerate}
        \item [α)] Το μικρότερο από τα τοπικά ελάχιστα μίας συνάρτησης δεν είναι απαραίτητα ελάχιστο της συνάρτησης.
        \item [β)] Μία συνάρτηση $f$ λέγεται γνησίως αύξουσα σε ένα διάστημα $Δ$ του πεδίου ορισμού της, αν υπάρχουν $x_1$, $x_2\in Δ$ με $x_1<x_2$ τέτοια ώστε $f(x_1)<f(x_2)$
        \item [γ)] Αν η συνάρτηση είναι παραγωγίσιμη στο $(α,β)$ και γνησίως φθίνουσα στο $[α,β]$, τότε υπάρχει $x_0\in (α,β)$ τέτοιο ώστε $f'(x)>0$.
        \item [δ)] Έστω μία συνάρτηση $f$ συνεχής σε ένα διάστημα $Δ$ και δύο φορές παραγωγίσιμη στο εσωτερικό του $Δ$. Αν $f''(x)>0$ για κάθε εσωτερικό σημείο $x$ του $Δ$, τότε η $f$ είναι κοίλη στο $Δ$.
        \item [ε)] Αν $0<α<1$, τότε $\lim_{x\to \infty}α^x=+\infty$.
      \end{enumerate}
  \end{enumerate}

\section*{Θέμα Β (21)}
  \noindent
  Δίνεται η συνάρτηση $f(x)=\frac{x^2-x-1}{x-2}$.
  \begin{enumerate}
    \item [α)] \textbf{[Μονάδες 7]} Να βρείτε τα διαστήματα μονοτονίας της $f$ καθώς και τις θέσεις και τις τιμές των τοπικών της ακροτάτων.
    \item [β)] \textbf{[Μονάδες 7]} Να μελετήσετε την $f$ ως προς την κυρτότητα.
    \item [γ)] \textbf{[Μονάδες 6]} Να βρείτε τις ασύμπτωτες της $f$.
    \item [δ)] \textbf{[Μονάδες 16]} Να βρείτε το σύνολο τιμών της $f$.
  \end{enumerate}

\section*{Θέμα Γ (17)}
  \noindent

  Δίνονται οι παραγωγίσιμες συναρτήσεις $f$, $g$ στο $[0,+\infty)$ για τις οποίες ισχύουν
  \begin{itemize}
    \item $f(0)=1$ και $f(x)\ne 0$ για κάθε $x\in (0,+\infty)$
    \item $f'$ συνεχής στο $[0,+\infty)$.
    \item $2g(x)=f^2(x)+2f(x)$ για κάθε $x\in (0,+\infty)$.
    \item $g'(x)=f(x)$ για κάθε $x\in [0,+\infty)$
  \end{itemize}
  \begin{enumerate}
    \item \textbf{[Μονάδες 8]}  Να δείξετε ότι οι $g$ και $f'$ έχουν το ίδιο είδος μονοτονίας στο $[0,+\infty)$.
    \item \textbf{[Μονάδες 5]}  Να δείξετε ότι $\lim_{x\to +\infty}x\left(f\left(\frac{1}{x}\right)-1\right)=\frac{1}{2}$.
    \item \textbf{[Μονάδες 6]}  Αν γνωρίζουμε ότι $\lim_{x\to +\infty}f(x)=+\infty$ να δείξετε ότι το σύνολο τιμών της $f'$ είναι το διάστημα $[\frac{1}{2},1)$
    \item \textbf{[Μονάδες 6]}  Να αποδείξετε ότι $5<2g(1)<2f(1)+3$.
  \end{enumerate}

\section*{Θέμα Δ (20)}
  \noindent

  Δίνεται η παραγωγίσιμη συνάρτηση $f:\mathbb{R}\to \mathbb{R}$ για την οποία ισχύουν:
  \begin{itemize}
    \item $e^{f(x)}-ημx\cdot e^{συνx-1+x}=e^{f(x)}\cdot f'(x)$, $x\in\mathbb{R}$
    \item $f(0)=0$
  \end{itemize}

  \begin{enumerate}
    \item \textbf{[Μονάδες 6]}  Να αποδείξετε ότι $f(x)=συνx+x-1$ για κάθε $x\in\mathbb{R}$.
    \item \textbf{[Μονάδες 5]}  Να αποδείξετε ότι η ευθεία $(ε_1)$ που διέρχεται από το σημείο $Α(π,π-2)$ και είναι παράλληλη στην εφαπτόμενη $(ε)$ της $C_f$ στο $x_0=0$, εφάπτεται της $C_f$ ακριβώς σε δύο σημεία στο διάστημα $[-π,π]$.
    \item \textbf{[Μονάδες 3]}  Να αποδείξετε ότι η $f$ αντιστρέφεται στο $(0,\frac{π}{2})$ καθώς και ότι ισχύει $f^{-1}>x$ για κάθε $x\in(0,\frac{π}{2}-1)$.
    \item
      \begin{enumerate}
        \item \textbf{[Μονάδες 5]}  Αν γνωρίζετε πως η $f^{-1}(x)$ είναι συνεχής στο $x_0=0$, να υπολογίσετε το όριο:
        $$\lim_{x\to 0^{+}}\frac{x-1+συν(f^{-1}(x))}{f^{-1}(x)}$$
        \item \textbf{[Μονάδες 6]}  Ένα κινητό $Μ$ ξεκινάει από την αρχή των αξόνων και κινείται κατά μήκος της $C_f$. Να αποδείξετε ότι δεν υπάρχει σημείο της καμπύλης $C_f$ στο διάστημα $[0,π]$ στο οποίο ο ρυθμός μεταβολής της τεταγμένης $y$ του $Μ$ να είναι διπλάσιος του ρυθμού μεταβολής της τετμημένης του $x$, αν υποτεθεί $x'(t)>0$ για κάθε $t\ge 0$.
      \end{enumerate}
  \end{enumerate}

\part*{\centering{Καλή επιτυχία}}
\end{document}
