\documentclass[12pt]{article}

\usepackage{amsmath}
\usepackage{unicode-math}
\usepackage{xltxtra}
\usepackage{xgreek}

\setmainfont{Liberation Serif}

\usepackage{tabularx}

\pagestyle{empty}

\usepackage{geometry}
 \geometry{a4paper, total={190mm,280mm}, left=10mm, top=10mm}

 \usepackage{graphicx}
 \graphicspath{ {images/} }

 \usepackage{wrapfig}

\begin{document}


\section*{Θέμα Α}
  \noindent
  Δίνεται η συνάρτηση $f:(1,+\infty)\to \mathbb{R}$ με $f(x)=e^x-\frac{x+1}{x-1}$.
  \begin{enumerate}
    \item \textbf{[Μονάδες 6]} Να δείξετε ότι η συνάρτηση $f$ γράφεται στη μορφή $f(x)=e^x-\frac{2}{x-1}-1$ και στη συνέχεια ότι είναι γνησίως αύξουσα.

    Με πράξεις έχουμε

    $$f(x)=e^x-\frac{x+1}{x-1}=f(x)=e^x-\frac{x-1+2}{x-1}=e^x-\frac{x-1}{x-1}-\frac{2}{x-1}=e^x-\frac{2}{x-1}-1$$.

    Επίσης έστω $x_1<x_2$. Έτσι $e^{x_1}<e^{x_2}$, $-\frac{2}{x_1-1}<-\frac{2}{x_1-1}$ μιας και η $\frac{1}{x}$ είναι γνησίως φθίνουσα και με πρόσθεση κατά μέλη

$$e^{x_1}-\frac{2}{x_1-1}-1<e^{x_2}-\frac{2}{x_2-1}-1\Rightarrow f(x_1)<f(x_2)$$


    \item \textbf{[Μονάδες 6]} Να δείξετε ότι ορίζεται η αντίστροφη συνάρτηση $f^{-1}$ και να βρείτε το πεδίο ορισμού της.

    Ως γνησίως αύξουσα είναι και 1-1, με πεδίο ορισμού της αντίστροφης να είναι το σύνολο τιμών της $f$. Η $f$ ως γνησίως αύξουσα
    στο διάστημα $(1,+\infty)$ θα έχει σύνολο τιμών το $(\lim_{x\to 1}f(x),\lim_{x\to +\infty}f(x))=(-\infty,\infty)$

    \item \textbf{[Μονάδες 6]} Να δείξετε ότι η συνάρτηση $f$ έχει μία ακριβώς ρίζα $x_0$ στο διάστημα $(1,2)$.

    Δείξαμε ότι $\lim_{x\to 1}f(x)=-\infty$ άρα υπάρχει $a$ κοντά στο $0$ με $f(a)<0$. Επειδή $f(2)=e^2-3>0$, με Bolzano στο $[a,2]$ έχουμε ότι υπάρχει ρίζα και στο $(1,2)$.

    \item \textbf{[Μονάδες 7]} Να δείξετε ότι η εξίσωση $e^x=\frac{x+1}{x-1}$ έχει δύο τουλάχιστον ρίζες αντίθετες.

    Μόλις δείξαμε ότι υπάρχει ρίζα στο $(1,2)$ για την $f(x)$ έστω την $x_0$. Δηλαδή ισχύει $\frac{1}{e^{x_0}}=\frac{x_0-1}{x_0+1}$. Θα δείξουμε ότι και η $-x_0$ είναι ρίζα της. Έχουμε

    $$f(-x_0)=e^{-x_0}-\frac{-x_0+1}{-x_0-1}=\frac{1}{e^{x_0}}-\frac{x_0-1}{x_0+1}=0$$

  \end{enumerate}

  \section*{Θέμα Β}
    \noindent
    Έστω $f:\mathbb{R}\to \mathbb{R}$ μία συνεχής συνάρτηση.
    \begin{enumerate}
      \item \textbf{[Μονάδες 6]} Αν $1<f(x)<e$, να δείξετε ότι η εξίσωση $f(x)=e^x$ έχει μία τουλάχιστον ρίζα στο διάστημα $(0,1)$.

      Με Bolzano στην $g(x)=f(x)-e^x$ στο $[0,1]$. Έχουμε $g$ συνεχής στο κλειστό ως πράξεις συνεχών. $g(0)=f(0)-1>0$ και $g(1)=f(1)-e<0$.
      \item \textbf{[Μονάδες 6]} Αν $f(0)>1$ και $\lim_{x\to +\infty}f(x)=-\infty$, να δείξετε ότι η εξίσωση $f(x)=e^x+x ημ\frac{1}{x}$ έχει μία τουλάχιστον θετική ρίζα.

      Με Bolzano στην $h(x)=f(x)-e^x-x ημ\frac{1}{x}$ στο $[a,b]$ όπου $a$ κοντά στο $0$ και $b$ αρκετά μεγάλο. Μένει να υπολογίσουμε τα όρια στο 0 και στο $+\infty$ της $g$. Έχουμε

      $$\lim_{x\to 0}x ημ\frac{1}{x}=0$$

      αφού

      $$-1\le ημ\frac{1}{x}\le 1 \Rightarrow -x \le x ημ\frac{1}{x} \le x$$

      για $x>0$. Και τα δύο πλευρικά όρια τείνουν στο 0. Τώρα για το όριο $\lim_{x\to 0}g(x)=f(0)-1-0>0$. Έτσι υπάρχει $a$ κοντά στο $0$ με $f(a)>0$. Ομοίως για το όριο στο $+\infty$ και γνωρίζοντας ότι με αλλαγή μεταβλητής $y=\frac{1}{x}$

$$\lim_{x\to +\infty}x ημ\frac{1}{x}=\lim_{y\to 0}\frac{ημy}{y}=1$$

Έτσι $\lim_{x\to +\infty}g(x)=-\infty-\infty-1=-\infty$. Άρα υπάρχει $b$ αρκετά μεγάλο ώστε $f(b)<0$ Έτσι στο $(a,b)$ υπάρχει μία τουλάχιστον ρίζα.

      \item \textbf{[Μονάδες 6]} Αν $f(a)+f(3a)=4a$, $a>0$ και η $f$ είναι γνησίως αύξουσα να δείξετε ότι η εξίσωση $\frac{f(x)-a}{x-3a}=\frac{f(x)-3a}{x-a}$, έχει μία τουλάχιστον ρίζα στο διάστημα $(a,3a)$.
      \item \textbf{[Μονάδες 7]} Θεωρούμε επιπλέον τη συνάρτηση $g:[1,3]\to \mathbb{R}$ με $g(x)=f(x)-x$. Να δείξετε ότι υπάρχει $x_0\in [1,3]$, ώστε $$g(x_0)=\frac{f(1)+2f(2)+3f(3)}{6}-\frac{7}{3}$$
    \end{enumerate}

    \section*{Θέμα Γ}
      \noindent
      Θεωρούμε τις συνεχείς συναρτήσεις $f$, $g:\mathbb{R}\to \mathbb{R}$ για τις οποίες ισχύουν:
      \begin{itemize}
        \item $f(0)=1$, $f(x)\ne x$ και $\frac{f(x)-x}{e^x}+\frac{e^x}{x-f(x)}=0$, για κάθε $x\in \mathbb{R}$
        \item $g(x)=\begin{cases} f(x) & \text{, }  x\le 0 \\ k-2x-ln(x+1) & \text{, }  x>0\end{cases}$
      \end{itemize}

      \begin{enumerate}
        \item \textbf{[Μονάδες 6]} Να δείξετε ότι $f(x)=e^x+x$, $x\in \mathbb{R}$ και να βρείτε την τιμή του $k$.
        \item \textbf{[Μονάδες 6]} Να βρείτε το σύνολο τιμών της συνάρτησης $g$ και να δείξετε ότι η $g$ έχει ακριβώς δύο ρίζες ετερόσημες.
        \item \textbf{[Μονάδες 6]} Να δείξετε ότι η εξίσωση $\frac{g(a)-1}{x-1}+\frac{g(β)-1}{x-2}=2019$ έχει μία τουλάχιστον ρίζα στο διάστημα $(1,2)$, για κάθε $a$, $β\ne 0$.
        \item \textbf{[Μονάδες 7]} Αν $x_1$, $x_2$ οι ρίζες του ερωτήματος $Γ2$ με $x_1 < x_2$ να δείξετε ότι η εξίσωση
        $$x+g(x)=ημ x$$
        έχει μία τουλάχιστον ρίζα στο διάστημα $(x_1,x_2)$.
      \end{enumerate}

\end{document}
