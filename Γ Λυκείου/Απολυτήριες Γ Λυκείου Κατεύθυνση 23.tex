\documentclass[12pt]{extarticle}

\usepackage{amsmath}
\usepackage{unicode-math}
\usepackage{xltxtra}
\usepackage{xgreek}

\setmainfont{Liberation Serif}

\usepackage{tabularx}

\pagestyle{empty}

\usepackage{geometry}
 \geometry{a4paper, total={190mm,275mm}, left=10mm, top=10mm}

 \usepackage{graphicx}
 \graphicspath{ {images/} }

 \usepackage{wrapfig}

\begin{document}

\begin{table}
  \small
  \begin{tabularx}{\textwidth}{ c X r }
    \begin{tabular}{ c }
      \includegraphics[scale=0.4]{ελληνική}          \\
      ΕΛΛΗΝΙΚΗ ΔΗΜΟΚΡΑΤΙΑ                            \\
      ΥΠΟΥΡΓΕΙΟ ΠΑΙΔΕΙΑΣ \& ΘΡΗΣΚΕΥΜΑΤΩΝ             \\
      ΠΕΡΙΦΕΡΕΙΑΚΗ Δ/ΝΣΗ Α/ΘΜΙΑΣ \& Β/ΘΜΙΑΣ  ΕΚΠ/ΣΗΣ \\
      ΚΕΝΤΡΙΚΗΣ ΜΑΚΕΔΟΝΙΑΣ                           \\
      Δ/ΝΣΗ Β/ΘΜΙΑΣ ΕΚΠ/ΣΗΣ ΑΝ. ΘΕΣ/ΝΙΚΗΣ            \\
      10ο ΓΕΝΙΚΟ ΛΥΚΕΙΟ ΘΕΣ/ΝΙΚΗΣ
    \end{tabular}
     &  &
    \begin{tabular}{ r }
      Σχολικό Έτος: 2022 - 2023     \\
      Εξ. Περίοδος: Μαΐου - Ιουνίου \\
      Μάθημα: Μαθηματικά Γ Λυκείου  \\
      Προσανατολισμού               \\
      Εισηγητές: Λόλας, Κράντας     \\ \\
      Θεσσαλονίκη, 25 / 05 / 2023
    \end{tabular}
  \end{tabularx}
\end{table}

\part*{\centering{Θέματα}}
\section*{Θέμα 1}
\noindent
\begin{enumerate}
  \item[α)] Έστω μία συνάρτηση $f$ παραγωγίσιμη σε ένα διάστημα $(α,β)$ με εξαίρεση ίσως ένα σημείο του $x_0$, στο οποίο όμως η $f$ είναι συνεχής. Αν $f'(x)>0$ στο $(α,x_0)$ και $f'(x)<0$ στο $(x_0,β)$, τότε να αποδείξετε ότι το $f(x_0)$ είναι τοπικό μέγιστο της $f$ \hspace*{\fill} \textbf{Μονάδες 10}
  \item[β)] Να δώσετε την γεωμετρική ερμηνεία του θεωρήματος μέσης τιμής του διαφορικού λογισμού. \hspace*{\fill} \textbf{Μονάδες 5}
  \item[γ)] Να χαρακτηρίσετε τις παρακάτω προτάσεις με Σωστό ή Λάθος
    \begin{enumerate}
      \item [α)] Αν $\lim\limits_{x \to x_0}{ f(x) }=0$ και $f(x)>0$ κοντά στο $x_0$, τότε $\lim\limits_{x \to x_0}{ \dfrac{1}{f(x)} }=+\infty$
      \item [β)] Ισχύει $\left( ημx \right)'=-συνx$.
      \item [γ)] Αν $f'(x)>0$ για κάθε $x \in D_f$ τότε η $f$ είναι γνησίως φθίνουσα.
      \item [δ)] Έστω $f$ μία συνεχής συνάρτηση σε ένα διάστημα $[α,β]$. Αν $G$ είναι μία παράγουσα της $f$ στο $[α,β]$, τότε $\int_α^β f(x)dx=G(β)-G(α)$
      \item [ε)] Ισχύει $\lim\limits_{x \to 0}{ \dfrac{1-συνx}{x} }=0$ \hspace*{\fill} \textbf{Μονάδες 10}
    \end{enumerate}

\end{enumerate}

\section*{Θέμα 2}
\noindent

Δίνονται οι συναρτήσεις $f(x)=\dfrac{1}{x-1}$, $x\ne 1$ και $g(x)=\dfrac{1}{e^x}$, $x\in \mathbb{R}$
\begin{enumerate}
  \item[α)] \begin{enumerate}
      \item[i.] Να ορίσετε το πεδίο ορισμού της συνάρτησης $h(x)=(f\circ g)(x)$ \hspace*{\fill} \textbf{Μονάδες 6}
      \item[ii.] Να βρείτε τον τύπο της συνάρτησης $h(x)=(f\circ g)(x)$ \hspace*{\fill} \textbf{Μονάδες 6}
    \end{enumerate}

    Αν $h(x)=\dfrac{e^x}{1-e^x}$, $x\in \mathbb{R}^*$ τότε:
  \item[β)] να αποδείξετε ότι η συνάρτηση $h$ είναι '1-1' \hspace*{\fill} \textbf{Μονάδες 7}
  \item[γ)] να υπολογίσετε το όριο: $\lim\limits_{x \to +\infty}{ h(x) }$ \hspace*{\fill} \textbf{Μονάδες 6}
\end{enumerate}
\newpage

\section*{Θέμα 3}
\noindent

Έστω συνάρτηση η $$f(x)=\begin{cases} e^x-x & \text{, } x \ne 0 \\ 1 & \text{, } x = 0  \end{cases}$$
\begin{enumerate}
  \item[α)] Να δείξετε ότι η $f$ είναι συνεχής στο $\mathbb{R}$.\hspace*{\fill} \textbf{Μονάδες 5}
  \item[β)] Να δείξετε ότι η εφαπτόμενη της $C_f$ στο $x=\ln2$ είναι η $y=x+2-2\ln2$.\hspace*{\fill} \textbf{Μονάδες 5}
  \item[γ)] Να δείξετε ότι η $f$ είναι γν. φθίνουσα στο $\left(-\infty,0\right)$ και γν. αύξουσα στο $\left(0,+\infty\right)$\hspace*{\fill} \textbf{Μονάδες 5}
  \item[δ)] Να βρείτε τα ακρότατα της $f$\hspace*{\fill} \textbf{Μονάδες 5}
  \item[ε)] Να αποδείξετε ότι $e^{\pi}-e^e > \pi-e$\hspace*{\fill} \textbf{Μονάδες 5}
\end{enumerate}

\section*{Θέμα 4}
\noindent

Δίνεται η συνάρτηση $f(x)=e^{2x}+x^3+2x$

\begin{enumerate}
  \item[α)] Να αποδείξετε ότι η συνάρτηση $f$ είναι γνησίως αύξουσα.\hspace*{\fill} \textbf{Μονάδες 8}
  \item[β)] Να αιτιολογήσετε γιατί η συνάρτηση $f$ αντιστρέφεται και να αποδείξετε ότι έχει σύνολο τιμών το $\mathbb{R}$

    \hspace*{\fill} \textbf{Μονάδες 7}
  \item[γ)] Να αποδείξετε ότι η αντίστροφη συνάρτηση της $f$ είναι επίσης γνησίως αύξουσα \hspace*{\fill} \textbf{Μονάδες 5}
  \item[δ)] Να λυθεί η εξίσωση $f^{-1}(x)=0$ \hspace*{\fill} \textbf{Μονάδες 5}
\end{enumerate}

\part*{\centering{Καλή επιτυχία}}
\begin{table}[htb]
  \begin{tabularx}{\textwidth}{ X c X c X}
     &
    \begin{tabular}[t]{ c }
      Ο Δ/ντης \\ \\ \\ \\
      Παπαδημητρίου Χρήστος
    \end{tabular}
     &   &
    \begin{tabular}[t]{ c }
      Οι εισηγητές      \\ \\ \\ \\
      Κράντας Στυλιανός \\ \\ \\ \\
      Λόλας Κωνσταντίνος
    \end{tabular}
     &
  \end{tabularx}
\end{table}


\vfill
\textbf{Οδηγίες}
\begin{enumerate}
  \item Μην ξεχάσετε να γράψετε το ονοματεπώνυμό σας σε κάθε φύλλο απαντήσεων που σας δώσουν.
  \item Όλες οι απαντήσεις να δωθούν στο φύλλο απαντήσεων. Οτιδήποτε γραφτεί στη σελίδα με τα θέματα δεν θα ληφθεί υπόψιν.
\end{enumerate}
\end{document}
