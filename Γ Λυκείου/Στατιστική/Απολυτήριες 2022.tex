\documentclass[12pt]{article}
\usepackage{fontspec}
\setmainfont{DejaVu Sans}
\usepackage{unicode-math}
\usepackage{xgreek}

\pagestyle{empty}

\usepackage{geometry}
 \geometry{a4paper, total={190mm,275mm}, left=10mm, top=10mm}

\binoppenalty=10000
\relpenalty=10000

\usepackage{graphicx}
\usepackage{float}
\usepackage{tabularx}\usepackage[table]{xcolor}

\graphicspath{ {../images/} }

\begin{document}
\begin{table}
    \small
    \begin{tabularx}{\textwidth}{ c X r }
      \begin{tabular}{ c }
        \includegraphics[scale=0.4]{ελληνική} \\
        ΕΛΛΗΝΙΚΗ ΔΗΜΟΚΡΑΤΙΑ \\
        ΥΠΟΥΡΓΕΙΟ ΠΑΙΔΕΙΑΣ, ΕΡΕΥΝΑΣ \& ΘΡΗΣΚΕΥΜΑΤΩΝ \\
        ΠΕΡΙΦΕΡΕΙΑΚΗ Δ/ΝΣΗ ΠΡΩΤ. \& ΔΕΥΤ/ΜΙΑΣ  ΕΚΠ/ΣΗΣ \\
        ΚΕΝΤΡΙΚΗΣ ΜΑΚΕΔΟΝΙΑΣ \\
        Δ/ΝΣΗ ΔΕΥΤΕΡΟΒΑΘΜΙΑΣ ΕΚΠ/ΣΗΣ ΑΝ. ΘΕΣ/ΝΙΚΗΣ \\
        10ο ΓΕΝΙΚΟ ΛΥΚΕΙΟ ΘΕΣ/ΝΙΚΗΣ
      \end{tabular}
      & &
      \begin{tabular}{ r }
        Σχολικό Έτος: 2021 - 2022 \\
        Εξ. Περίοδος: Μαΐου - Ιουνίου \\
        Μάθημα: Μαθηματικά Γ Στατιστική\\
        Εισηγητής: Λόλας\\ \\
        Θεσσαλονίκη, 25 / 05 / 22
      \end{tabular}
    \end{tabularx}
\end{table}

\part*{\centering{Θέματα}}

\section*{Θέμα Α}
\begin{enumerate}
 \item[Α1.] \textbf{[Μονάδες 8]} Αν $Α$ ένα ενδεχόμενο και $Α'$ το συμπληρωματικό του, να αποδείξετε ότι $P(Α')=1-P(Α)$
 \item[Α2.] \textbf{[Μονάδες 7]} Τι ονομάζουμε πληθυσμό και τι δείγμα στη στατιστική
 \item[Α3.] \textbf{[Μονάδες 10]} Να χαρακτηρίσετε τις παρακάτω προτάσεις με Σωστό ή Λάθος

 \begin{enumerate}
  \item[1.] Για κάθε $A$ και $B$ ενδεχόμενα, ισχύει $P(A\cup B)=P(A)+P(B)$.
  \item[2.] Ο αριθμός των σεισμών κατά την διάρκεια ενός έτους είναι συνεχής μεταβλητή.
  \item[3.] Στην κανονική κατανομή, η διάμεσος είναι ίση με τη μέση τιμή.
  \item[4.] Η ελάχιστη τιμή είναι μέτρο διασποράς.
  \item[5.] Η επαγωγική στατιστική περιγράφει ένα δείγμα.
 \end{enumerate}
\end{enumerate}

\section*{Θέμα Β}
Στο Ταμτούμ (ένας καταπληκτικός προορισμός) που ζούνε 200 άτομα, πραγματοποιήθηκε μια μελέτη για το κατά πόσο οι κάτοικοι συνδυάζουν το γεύμα τους με ψωμί και σαλάτα. Η έρευνα έδειξε ότι:
\begin{itemize}
 \item Το 62\% των ατόμων τρώει ψωμί (ενδεχόμενο Α)
 \item Το 53\% των ατόμων τρώει σαλάτα (ενδεχόμενο Β)
 \item Το 27\% των ατόμων τρώει σαλάτα και όχι ψωμί
\end{itemize}
Να απαντηθούν τα παρακάτω ερωτήματα:
\begin{enumerate}
 \item[Β1.] \textbf{[Μονάδες 6]} Πόσοι στο Ταμτούμ τρώνε σαλάτα;
 \item[Β2.] \textbf{[Μονάδες 6]} Να σχεδιάσετε τρία διαγράμματα Venn στα οποία να τοποθετείτε τα 3 ενδεχόμενα που δίνονται στην εκφώνηση
 \item[Β2.] \textbf{[Μονάδες 6]} Ποιά η πιθανότητα ένας Ταμτουμάθρωπος, να τρώει και σαλάτα και ψωμί;
 \item[Β3.] \textbf{[Μονάδες 7]} Ποιά η πιθανότητα ένας Ταμτουμάθρωπος, να τρώει μόνο το φαγητό του, χωρίς ψωμί και χωρίς σαλάτα;
\end{enumerate}
Όλες οι απαντήσεις θα πρέπει να συνοεδεύονται από αιτιολόγηση.

\section*{Θέμα Γ}
Ο παρακάτω πίνακας είναι τα ευρήματα μιας μελέτης 120 ατόμων του Ταμτούμ για το πόσες ώρες βλέπουν ΤαμFlix την ημέρα. Η μελέτη θα παρουσιαζόταν σε επενδυτές για να αναβαθμιστούν οι υπηρεσίες που παρέχονται, αλλά κατά την αντιγραφή του πίνακα, σβήστηκαν κάποια νούμερα.
\begin{center}
 \begin{tabularx}{0.8\textwidth}{
  | >{\centering\arraybackslash}X
  | >{\centering\arraybackslash}X
  | >{\centering\arraybackslash}X
  |>{\centering\arraybackslash}X
  |>{\centering\arraybackslash}X
  |>{\centering\arraybackslash}X|}
  \hline
  λεπτά     & $x_i$                    & $n_i$ & $f_i$ & $x_i\cdot n_i$ & $x_i^2\cdot n_i$ \\
  \hline
  $[0,20)$  & 10                       & 24    & (α)   &                &                  \\
  \hline
  $[20,40)$ & 30                       & (β)   & 0.3   &                &                  \\
  \hline
  $[40,60]$ & 50                       & (γ)   & (δ)   &                &                  \\
  \hline
  Σύνολο    & \cellcolor[HTML]{999999} & (ε)   & (ζ)   &                &                  \\
  \hline
 \end{tabularx}
\end{center}
\begin{enumerate}
 \item[Γ1.] \textbf{[Μονάδες 5]} Να υπολογίσετε τις τιμές (α), (β), (γ), (δ), (ε) και (ζ)
 \item[Γ2.] \textbf{[Μονάδες 5]} Να συμπληρώσετε τον υπόλοιπο πίνακα
 \item[Γ3.] \textbf{[Μονάδες 5]} Να βρείτε την μέση τιμή και την τυπική απόκλιση των λεπτών παρακολούθησης
 \item[Γ4.] \textbf{[Μονάδες 5]} Να σχεδιάσετε το ιστόγραμμα με βάση τον πίνακα
 \item[Γ5.] \textbf{[Μονάδες 5]} Να σχεδιάσετε το ιστόγραμμα με βάση τον πίνακα
\end{enumerate}

\section*{Θέμα Δ}
Στην χώρα Ταμτούμ, χρησιμοποιούν το 3αδικό σύστημα αρίθμησης, δηλαδή μόνο 3 ψηφία για νούμερα, το 0 το 1 και το 2.
\begin{enumerate}
 \item[Δ1.] \textbf{[Μονάδες 8]} Πόσα διαφορετικά νούμερα δημιουργούν αν χρησιμοποιήσουν 8 ψήφιους αριθμούς; Προσοχή, για να είναι οχταψήφιος δεν πρέπει το πρώτο ψηφίο να είναι 0.
 \item[Δ2.] \textbf{[Μονάδες 5]} Ποιά η πιθανότητα ο αριθμός αυτός να περιέχει τουλάχιστον ένα δυάρι;
\end{enumerate}
Γνωρίζουμε επίσης ότι το ύψος των Ταμτουμάνθρωπων ακολουθεί κανονική κατανομή με μέση τιμή 1,2 μέτρα και τυπική απόκλιση 0,1 μέτρα.
\begin{enumerate}
 \item[Δ3.] \textbf{[Μονάδες 6]} Ποιά η πιθανότητα να δούμε έναν Ταμτουμάνθρωπο στο δρόμο με ύψος πάνω από 1,2;
 \item[Δ4.] \textbf{[Μονάδες 6]} Πόσοι στο Ταμτούμ είναι από 1,1 έως 1,3 μέτρα;
\end{enumerate}

Δίνονται οι τύποι:
\begin{itemize}
 \item $\bar{x}=\frac{x_1n_1+x_2n_2+x_3n_3}{N}$, $s=\sqrt{\frac{x_1^2n_1+x_2^2n_2+x_3^2n_3}{N}-\bar{x}^2}$
 \item $P(\bar{x}-s,\bar{x}+s)=68\%$, $P(\bar{x}-2s,\bar{x}+2s)=95\%$
\end{itemize}
\part*{\centering{Καλή επιτυχία}}

\begin{table}[htb]
    \begin{tabularx}{\textwidth}{ X c X c X}
      &
      \begin{tabular}[t]{ c }
        Ο Δ/ντης \\ \\ \\ \\
        Παπαδημητρίου Χρήστος
      \end{tabular}
      & &
      \begin{tabular}[t]{ c }
        Ο εισηγητής \\ \\ \\ \\
        Λόλας Κωνσταντίνος
      \end{tabular}
      &
    \end{tabularx}
\end{table}

\vfill
 \textbf{Οδηγίες}
 \begin{enumerate}
   \item Μην ξεχάσετε να γράψετε το ονοματεπώνυμό σας σε κάθε φύλλο απαντήσεων που σας δώσουν.
   \item Όλες οι απαντήσεις να δωθούν στο φύλλο απαντήσεων. Οτιδήποτε γραφτεί στη σελίδα με τα θέματα δεν θα ληφθεί υπόψιν.
   \item Τα Σωστό - Λάθος δεν χρειάζονται αιτιολόγηση.
 \end{enumerate}
\end{document}
