\documentclass[10pt]{article}

\usepackage{amsmath}
\usepackage{unicode-math}
\usepackage{xltxtra}
\usepackage{xgreek}

\setmainfont{Liberation Serif}

\usepackage{tabularx}

\pagestyle{empty}

\usepackage{geometry}
 \geometry{a4paper, total={190mm,275mm}, left=10mm, top=10mm}

 \usepackage{graphicx}
 \graphicspath{ {images/} }

 \usepackage{wrapfig}

\begin{document}

\begin{table}
    \small
    \begin{tabularx}{\textwidth}{ c X r }
      \begin{tabular}{ c }
        \includegraphics[scale=0.4]{ελληνική} \\
        ΕΛΛΗΝΙΚΗ ΔΗΜΟΚΡΑΤΙΑ \\
        ΥΠΟΥΡΓΕΙΟ ΠΑΙΔΕΙΑΣ, ΕΡΕΥΝΑΣ \& ΘΡΗΣΚΕΥΜΑΤΩΝ \\
        ΠΕΡΙΦΕΡΕΙΑΚΗ Δ/ΝΣΗ ΠΡΩΤ. \& ΔΕΥΤ/ΜΙΑΣ  ΕΚΠ/ΣΗΣ \\
        ΚΕΝΤΡΙΚΗΣ ΜΑΚΕΔΟΝΙΑΣ \\
        Δ/ΝΣΗ ΔΕΥΤΕΡΟΒΑΘΜΙΑΣ ΕΚΠ/ΣΗΣ ΑΝ. ΘΕΣ/ΝΙΚΗΣ \\
        10ο ΓΕΝΙΚΟ ΛΥΚΕΙΟ ΘΕΣ/ΝΙΚΗΣ
      \end{tabular}
      & &
      \begin{tabular}{ r }
        Σχολικό Έτος: 2017 - 2018 \\
        Εξ. Περίοδος: Μαΐου - Ιουνίου \\
        Μάθημα: Μαθηματικά Γ Λυκείου Γενικής\\
        Εισηγητής: Λόλας\\ \\
        Θεσσαλονίκη, 31 / 05 / 2018
      \end{tabular}
    \end{tabularx}
\end{table}

\part*{\centering{Θέματα}}
\section*{Θέμα Α}
  \noindent
  \begin{enumerate}
    \item \textbf{[Μονάδες 15]} Να αποδείξετε ότι αν $Α$ ένα ενδεχόμενο ενός δειγματοχώρου και $Α'$ το συμπλήρωμά του, τότε $P(Α')=1-P(Α)$.
    \item \textbf{[Μονάδες 10]}  Να χαρακτηρίσετε τις παρακάτω προτάσεις με Σωστό ή Λάθος
    \begin{enumerate}
      \item [α)] Η μέση τιμή είναι μέτρο θέσης.
      \item [β)] Ο χρόνος είναι ποιοτική μεταβλητή.
      \item [γ)] Για δύο ασυμβίβαστα ενδεχόμενα $Α$ και $Β$ ισχύει $P(Α\cup Β)=P(Α)+P(Β)$.
      \item [δ)] Αν $CV < 0.1$ τότε τα δεδομένα είναι ομοιογενή.
      \item [ε)] Η διάμεσος είναι η μέση τιμή.
    \end{enumerate}
  \end{enumerate}

\section*{Θέμα Β}
  \noindent
Οι βαθμοί στα 4 μαθήματα Πανελλαδικών είναι $18$, $15$, $17$, $10$. Να βρείτε
  \begin{enumerate}
    \item \textbf{[Μονάδες 2]} την μέγιστη και την ελάχιστη βαθμολογία
    \item \textbf{[Μονάδες 2]} το εύρος
    \item \textbf{[Μονάδες 3]} την διάμεσο
    \item \textbf{[Μονάδες 5]} την μέση τιμή
    \item \textbf{[Μονάδες 6]} την διακύμανση
    \item \textbf{[Μονάδες 3]} την τυπική απόκλιση
    \item \textbf{[Μονάδες 4]} τον συντελεστή μεταβλητότητας
  \end{enumerate}
Δίνεται $\sqrt{9.5}=3,08$

\section*{Θέμα Γ}
  \noindent

Σε μία τάξη μετρήθηκαν τα άτομα ως προς το ύψος και προέκυψε ο παρακάτω πίνακας.
\begin{center}
\begin{tabular}{| c | c | c | c | c | c | c |} \hline
  Κλάσεις & Συχνότητα & Κεντρική Τιμή & Σχ. Συχνότητα & Σχ. Συχν. $\%$ &  Αθρ. Συχν. & Σχ. Αθρ. Συχν.\\ \hline
  Ύψος & $ν_i$ & $x_i$ & $f_i$ & $f_i\%$ & $Ν_i$ & $F_i$ \\ \hline
  $[1.50, 1.60)$ & 2 &   &      &    &    &  \\ \hline
  $[1.60, 1.70)$ &   &   & 0,20 &    &    &  \\ \hline
  $[1.70, 1.80)$ &   &   &      &    &    & 0,56 \\ \hline
  $[1.80, 1.90)$ &   &   &      & 24 &    &  \\ \hline
  $[1.90, 2.00]$ &   &   &      &    & 25 &  \\ \hline
  Σύνολο &  & - & &  & - & - \\ \hline
\end{tabular}
\end{center}


  \begin{enumerate}
   \item \textbf{[Μονάδες 7]}   Να συμπληρωθούν οι τιμές που λείπουν.
   \item \textbf{[Μονάδες 6]}   Τι ποσοστό ανθρώπων έχει ύψος από 1,60μ έως 1,70μ;
   \item \textbf{[Μονάδες 5]}  Τι ποσοστό έχει πάνω από 1,80μ;


   Υποθέστε ότι ο μέσος όρος των υψών είναι 1,77μ και η τυπική απόκλιση 0,11μ. Αν κάθε χρόνο το ύψος αυξάνεται των ανθρώπων αυξάνεται κατά 0,03μ

   \item \textbf{[Μονάδες 7]}  Ποιός θα είναι ο μέσος όρος των υψών την επόμενη χρονιά και ποιά θα είναι η νέα τυπική απόκλιση;
  \end{enumerate}

\section*{Θέμα Δ}
  \noindent

Στη σχολή που φοιτάτε, αγοράζετε καφέ το πρωί με πιθανότητα $60\%$ και τρώτε τυρόπιτα με πιθανότητα $24\%$. Δεν αγοράζετε τίποτα με πιθανότητα $26\%$.

  \begin{enumerate}
    \item \textbf{[Μονάδες 6]}  Να παραστήσετε τα παραπάνω ενδεχόμενα σε διάγραμμα Venn.
    \item \textbf{[Μονάδες 6]}  Ποιά η πιθανότητα να αγοράσετε και καφέ και τυρόπιτα;
    \item \textbf{[Μονάδες 6]}  Ποιά η πιθανότητα να αγοράσετε τουλάχιστον ένα από τα δύο;
    \item \textbf{[Μονάδες 7]}  Ποιά η πιθανότητα να αγοράσετε μόνο καφέ;
  \end{enumerate}

\part*{\centering{Καλή επιτυχία}}
\begin{table}[htb]
    \begin{tabularx}{\textwidth}{ X c X c X}
      &
      \begin{tabular}[t]{ c }
        Ο Δ/ντης \\ \\ \\ \\
        Παπαδημητρίου Χρήστος
      \end{tabular}
      & &
      \begin{tabular}[t]{ c }
        Ο εισηγητής \\ \\ \\ \\
        Λόλας Κωνσταντίνος
      \end{tabular}
      &
    \end{tabularx}
\end{table}


\vfill
 \textbf{Οδηγίες}
 \begin{enumerate}
   \item Μην ξεχάσετε να γράψετε το ονοματεπώνυμό σας σε κάθε φύλλο απαντήσεων που σας δώσουν.
   \item Όλες οι απαντήσεις να δωθούν στο φύλλο απαντήσεων. Οτιδήποτε γραφτεί στη σελίδα με τα θέματα δεν θα ληφθεί υπόψιν.
   \item Τα Σωστό - Λάθος δεν χρειάζονται αιτιολόγηση.
 \end{enumerate}
\end{document}
