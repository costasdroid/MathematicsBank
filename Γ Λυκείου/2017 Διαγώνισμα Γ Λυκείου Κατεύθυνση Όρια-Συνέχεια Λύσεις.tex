\documentclass[12pt]{article}

\usepackage{amsmath}
\usepackage{unicode-math}
\usepackage{xltxtra}
\usepackage{xgreek}

\setmainfont{Liberation Serif}

\usepackage{tabularx}

\pagestyle{empty}

\usepackage{geometry}
 \geometry{a4paper, total={190mm,280mm}, left=10mm, top=10mm}

 \usepackage{graphicx}
 \graphicspath{ {images/} }

 \usepackage{wrapfig}

\begin{document}

\begin{table}
    \small
    \begin{tabularx}{\textwidth}{ c X r }
      \begin{tabular}{ l }
        Εισηγητής: Λόλας Κωνσταντίνος \\
        Επαναληπτικό: Όρια - Συνέχεια
      \end{tabular}
      & &
      \begin{tabular}{ r }
        Θεσσαλονίκη, 13 / 12 / 2017
      \end{tabular}
    \end{tabularx}
\end{table}

\part*{\centering{Λύσεις}}

\section*{Θέμα Α}
  \noindent
  \begin{enumerate}
    \item Απόδειξη από βιβλίο.
    \item Ορισμός βιβλίου.
    \item Να χαρακτηρίσετε τις παρακάτω προτάσεις με Σωστό ή Λάθος
    \begin{enumerate}
      \item [α)] Σ Κάθε οριζόντια ευθεία τέμενει τη γραφική παράσταση μιας 1-1 συνάρτησης το πολύ σε ένα σημείο.
      \item [β)] Λ Αν οι συναρτήσεις $f$, $g$ είναι συνεχείς στο σημείο $x_0$, τότε και η σύνθεσή τους $g\circ f$ είναι συνεχής στο ίδιο σημείο.
      \item [γ)] Σ Το σύνολο τιμών ενός κλειστού διαστήματος μέσω μιας συνεχούς και μη σταθερής συνάρτησης είναι πάντοτε κλειστό διάστημα.
      \item [δ)] Σ $\lim_{x\to 0^+}\ln x=-\infty$
      \item [ε)] Λ Αν υπάρχει το $\lim_{x\to x_0}\left(f(x)+g(x)\right)$, τότε υποχρεωτικά υπάρχουν και τα όρια $\lim_{x\to x_0}f(x)+\lim_{x\to x_0}g(x)$.
    \end{enumerate}
  \end{enumerate}

\section*{Θέμα Β}
  \noindent
  \begin{enumerate}
    \item Θα δείξουμε ότι $\frac{4x}{x^2+4}\le f(2)$.
    $$4x\le x^2+4 \iff x^2-4x+4\ge 0 \iff (x-2)^2\ge 0$$
    \item Από τα δεδομένα έχουμε $\lim_{y\to -\infty}f(y)=4$ και θέτοντας
    $$h(x)=\frac{xf(x)}{3x-1} \iff f(x)=h(x)(3-\frac{1}{x}) \Rightarrow \lim_{x\to \infty}f(x)=6$$
    Το σύνολο τιμών είναι το
    $$\left[ f(2), \lim_{x\to -\infty}f(x) \right) \cup \left[ f(2), \lim_{x\to \infty}f(x) \right) = \left[1,4\right)\cup \left[1, 6\right)$$
    \item Από το σύνολο τιμών για $1< a < 4$ έχουμε δύο ρίζες, για $4 \le a < 6$ έχουμε μία ρίζα όπως και για $a=1$ και παντού αλλού καμία.
    \item Εφόσον $f(x)\ge 1$ και $g(x)\le 1$ όπως και $f(2)=g(2)=1$ τότε το $x=2$ είναι μοναδική λύση
  \end{enumerate}
  %\vspace{7\baselineskip}

\section*{Θέμα Γ}
  \noindent
  \begin{enumerate}
    \item Η αντίστροφη μίας συνάρτησης είναι συμμετρική ως προς την $y=x$ και αφού $f(x)\le x$ θα ισχύει $f^{-1}(x)\ge x$.
    \item Αφού $f^{-1}(x)\le e^x-1 \text{ για κάθε } x\in\mathbb{R}$, για $x=f(y)$, $y>-1$ θα ισχύει
    $$y\le e^{f(y)}-1 \iff \ln (y+1) \le f(x)$$
    \item Από την προηγούμενη σχέση με όρια $\lim_{x\to 0}\ln(x+1)\le \lim_{x\to 0} f(x) \le \lim_{x\to 0}x$. Αφού $0\le f^{-1}(x)\le e^x-1$ και πάλι με κριτήριο παρεμβολής.
    \item Έστω $h(x)=(x-1)f^{-1}(x)+(2-x)f(x)-x^2+2x-2$. Η $h$ συνεχής με $h(1)=f(1)-1\le 0$, και $h(2)=f^{-1}(2)-2\ge 0$. Έτσι αν $h(1)h(2)=0$ τότε $x_0=1$ ή $x_0=2$, διαφορετικά Bolzano.
  \end{enumerate}

  \section*{Θέμα Δ}
    \noindent
    \begin{enumerate}
      \item Κατασκευή ή λόγια.
      \item Για $x<x_0$ έχουμε $f(x)<f(x_0)$ και άρα με όρια $\lim_{x\to x_0^-}f(x)\le f(x_0)$. Όμοια από δεξιά. Άρα $\lim_{x\to x_0}f(x)=f(x_0)$
      \item 
      \begin{enumerate}
        \item [α)] Η $f$ διατηρεί πρόσημο και αφού $f(0)=1$ θα είναι πάντα $f(x)>0 \Rightarrow f(f(x))>f(0)=1$. Συνεπώς $g(x) > 1$.
        \item [β)] $$h(x) = x^3g\left(2x^4\right)+x^4g\left(x^2\right)+x^2f\left(x^2-1\right)-1$$ Η $h$ είναι συνεχής με $h(1)=g(2)+g(1)+f(0)-1>2$ και $h(-1)=-g(2)+g(1)+f(0)-1$. Αλλά $g(2)>g(1) \iff -g(2)+g(1)<0$... Bolzano
      \end{enumerate}
    \end{enumerate}

\vspace{3\baselineskip}

\end{document}
