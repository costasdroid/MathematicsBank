\documentclass[12pt]{article}

\usepackage{amsmath}
\usepackage{unicode-math}
\usepackage{xltxtra}
\usepackage{xgreek}

\setmainfont{Liberation Serif}

\usepackage{tabularx}

\pagestyle{empty}

\usepackage{geometry}
 \geometry{a4paper, total={190mm,275mm}, left=10mm, top=10mm}

 \usepackage{graphicx}
 \graphicspath{ {images/} }

 \usepackage{wrapfig}
\usepackage{lipsum}%% a garbage package you don't need except to create examples.

\begin{document}

\part*{\centering{Λύσεις Μαθηματικά Γ Λυκείου Κατεύθυνση}}

\section*{Θέμα Α}
  \noindent
  \begin{enumerate}
    \item \textbf{[Μονάδες 15]} Απόδειξη από το βιβλίο
    \item \textbf{[Μονάδες 10]}
    \begin{enumerate}
      \item [α)] Σωστό
      \item [β)] Λάθος
      \item [γ)] Λάθος
      \item [δ)] Λάθος
      \item [ε)] Λάθος
    \end{enumerate}
  \end{enumerate}

\section*{Θέμα Β}
  \noindent
  \begin{enumerate}
    \item $1$
    \item $3$
    \item $+\infty$
    \item $3$
    \item $\frac{1}{e}$
  \end{enumerate}

\section*{Θέμα Γ}
  \noindent
  \begin{enumerate}
    \item Η $f$ είναι συνεχής για $x\ne0$ και στο $0$ ισχύει $\lim_{x\to 0}f(x)=1=f(0)$, άρα $f$ συνεχής στο $\mathbb{R}$
    \item \begin{gather*}f'(x)=e^x-1 \\ f'(\ln2)=1 \\ f(\ln2)=2-\ln2 \\ y-f(2)=f'(2)(x-\ln2) \\ y=x+2-2\ln2 \end{gather*}
    \item $f'(x) \ge 0$ για $x>0$ και $f'(x) \le 0$ για $x<0$
    \item Ολικό ελάχιστο στο $x=0$ το $f(0)=1$
    \item Για $x>0$ η $f$ είναι γν. αύξουσα άρα \begin{gather*}e<\pi \\ f(e)<f(\pi) \\ \ldots \end{gather*}
  \end{enumerate}

\section*{Θέμα Δ}
  \noindent
  \begin{enumerate}
    \item $f'(x)=3x^2-2x^2+2 > 0$ Άρα $f$ γνήσια αύξουσα. Το σύνολο τιμών της είναι το $\mathbb{R}$ αφού τα όρια στα άπειρα είναι αντίστοιχα άπειρα
    \item Bolzano στο $(-1,0)$
    \item Ως γν. μονότονη έχει το πολύ μία που έχει βρεθεί. Άρα μοναδική
    \item Έστω έχει τουλάχιστον 3. Άρα Rolle $\implies$ η $f$ έχει τουλάχιστον 2 $\implies$ άτοπο.
    \item ΘΜΤ στο $(1,a)$
  \end{enumerate}

\end{document}
