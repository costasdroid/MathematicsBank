\documentclass[12pt]{article}

\usepackage{amsmath}
\usepackage{unicode-math}
\usepackage{xltxtra}
\usepackage{xgreek}

\setmainfont{Liberation Serif}

\usepackage{tabularx}

\pagestyle{empty}

\usepackage{geometry}
 \geometry{a4paper, total={190mm,275mm}, left=10mm, top=10mm}

 \usepackage{graphicx}
 \graphicspath{ {images/} }

 \usepackage{wrapfig}
\usepackage{lipsum}%% a garbage package you don't need except to create examples.

\begin{document}
  \part*{\centering{Πανελλήνιες Μαθηματικά Γ Λυκείου 2017}}

  \section*{Θέμα Β}
  Δίνονται οι συναρτήσεις $f(x)=\ln x$, $x>0$ και $g(x)=\frac{x}{1-x}$, $x \ne 1$,
  \begin{enumerate}
    \item [B1.] Να προσδιορίσετε τη συνάρτηση $f\circ g$.
    \item [B2.] Αν $h(x)=(f\circ g)(x)=\ln \left( \frac{x}{1-x}\right)$, $x\in (0,1)$, να αποδείξετε ότι η συάρτηση $h$ αντιστρέφεται και να βρείτε την αντίστροφή της.
    \item [B3.] Αν $φ(x)=h^{-1}(x)=\frac{e^x}{e^x+1}$, $x\in \mathbb{R}$, να μελετήσετε τη συνάρτηση $φ$ ως προς τη μονοτονία, τα ακρότατα, την κυρτότητα και τα σημεία καμπής.
    \item [B4.] Να βρείτε τις οριζόντιες ασύμπτωτες της γραφικής παράστασης της συνάρτησης $φ$ και να τη σχεδιάσετε. (Η γραφική παράσταση να σχεδιαστεί με στυλό.)
  \end{enumerate}

  \section*{Θέμα Γ}
  Δίνεται η συνάρτηση $f(x)=-ημx$, $x\in[0,\pi]$, και το σημείο $Α\left(\frac{\pi}{2},-\frac{\pi}{2}\right)$.
  \begin{enumerate}
    \item [Γ1.] Να αποδείξτε ότι υπάρχουν ακριβώς δύο εφαπτόμενες $(ε_1)$, $(ε_2)$ της γραφικής παράστασης της $f$ που άγονται από το $Α$, τις οποίες και να βρείτε.
    \item [Γ2.] Αν $(ε_1):y=-x$ και  $(ε_2):y=x-\pi$ είναι οι ευθείες του ερωτήματος Γ1, τότε να σχεδιάσετε τις $(ε_1)$, $(ε_2)$ και τη γραφική παράσταση της $f$ και να αποδείξετε ότι $\frac{E_1}{E_2}=\frac{\pi^2}{8}-1$, όπου:
      \begin{itemize}
        \item $E_1$ είναι το εμβαδόν του χωρίου που περικλείεται από τη γραφική παράσταση της $f$ και τις ευθείες $(ε_1)$, $(ε_2)$, και
        \item $E_2$ είναι το εμβαδόν του χωρίου που περικλείεται από τη γραφική παράσταση της $f$ και τον άξονα $x'x$.
      \end{itemize}
    \item [Γ3.] Να υπολογίσετε το όριο \[ \lim_{x\to \pi} \frac{f(x)+x}{f(x)-x+\pi} \]
    \item [Γ4.] Να αποδείξετε ότι \[ \int_{1}^{e}\frac{f(x)}{x}dx > e-1-\pi \]
  \end{enumerate}

  \section*{Θέμα Δ}
  Δίνεται η συνάρτηση \[ f(x)=\begin{cases} \sqrt[3]{x^4}, & x\in[-1,0) \\\ e^xημx, & x\in[0,\pi]\end{cases} \]
  \begin{enumerate}
    \item [Δ1.] Να δείξετε ότι η συνάρτηση $f$ είναι συνεχής στο διάστημα $[-1,\pi]$ και να βρείτε τα κρίσιμα σημεία της
    \item [Δ2.] Να μελετήσετε τη συνάρτηση $f$ ως προς τη μονοτονία και τα ακρότατα και να βρείτε το σύνολο τιμών της.
    \item [Δ3.] Να βρείτε το εμβαδόν του χωρίου που περικλείεται από τη γραφική παράσταση της $f$, τη γραφική παράσταση της $g$, με $g(x)=e^{5x}$, $x\in\mathbb{R}$, τον άξονα $y'y$ και την ευθεία $x=\pi$.
    \item [Δ4.] Να λύσετε την εξίσωση \[ 16e^{-\frac{3\pi}{4}}f(x)-e^{-\frac{3\pi}{4}}\left(4x-3\pi\right)^2=8\sqrt{2} \]
  \end{enumerate}

\end{document}
