\documentclass[14pt]{extarticle}
\usepackage{fontspec}
\setmainfont{Libertinus Sans}
\usepackage{unicode-math}
\usepackage{xgreek}

\pagestyle{empty}

\usepackage{geometry}
 \geometry{a4paper, total={190mm,275mm}, left=10mm, top=10mm}

\binoppenalty=10000
\relpenalty=10000

\begin{document}
\part*{\centering{Θέματα}}

\section*{Θέμα Α (3)}
\begin{enumerate}
 \item[Α1.] \textbf{[Μονάδες 3]} Πότε η ευθεία $x=x_0$ λέγεται κατακόρυφη ασύμπτωτη της γραφικής παράστασης μιας συνάρτησης $f$.
 \item[Α2.] \textbf{[Μονάδες 8]} Έστω συνάρτηση $f$ συνεχής σε ένα διάστημα $Δ$. Αν $f'(x)>0$ σε κάθε εσωτερικό σημείο $x$ του $Δ$, να δείξετε ότι η $f$ είναι γνησίως αύξουσα.
 \item[A3.] \textbf{[Μονάδες 4]} Θεωρήστε τον παρακάτω ισχυρισμό:
       
       "Για κάθε συνάρτηση $f$ η οποία είναι δύο φορές παραγωγίσιμη και κυρτή στο $\mathbb{R}$ ισχύει $f''(x)>0$ για κάθε $\in\mathbb{R}$."
       \begin{enumerate}
        \item Να χαρακτηρίσετε τον παραπάνω ισχυρισμό με Α (αληθής) ή Ψ (ψευδής) (Μονάδα 1)
        \item Να αιτιολογήσετε την απάντησή σας στο ερώτημα α (Μονάδες 3)
       \end{enumerate}
       
 \item[A4.] \textbf{[Μονάδες 10]} Να χαρακτηρίσετε τις προτάσεις που ακολουθούν, γράφοντας στο τετράδιό σας, δίπλα στο γράμμα που αντιστοιχεί σε κάθε πρόταση, τη λέξη Σωστό, αν η πρόταση είναι σωστή, ή Λάθος, αν η πρόταση είναι λανθασμένη.
       \begin{enumerate}
        \item Κάθε "1-1" συνάρτηση είναι γνησίως μονότονη
        \item Για κάθε $x\in\mathbb{R}$ ισχύει $|ημx|>|x|$.
        \item Αν η συνάρτηση $f$ δεν είναι συνεχής στο $x_0$, τότε δεν είναι παραγωγίσιμη στο $x_0$.
        \item Αν μια συνάρτηση $f$ είναι κοίλη σ' ένα διάστημα $Δ$, τότε η εφαπτομένη της γραφικής παράστασης της $f$ σε κάθε σημείο του $Δ$ βρίσκτεται κάτω από τη γραφική της παράσταση.
        \item Αν $C$ είναι η γραιφκή παράσταση μιας συνάρτησης $f$, τότε το πεδίο ορισμού της $f$ είναι το σύνολο των τεταγμένων των σημείων της $C$.
       \end{enumerate}
\end{enumerate}

\section*{Θέμα Β (39)}
Δίνονται οι συναρτήσεις $f(x)=e^{2x}-2e^{x}$, $x\ge 0$ και $g(x)=\ln x$.
\begin{enumerate}
 \item[Β1.] \textbf{[Μονάδες 3]} Να βρείτε τη συνάρτηση $f\circ g$.
       
       Δίνεται ότι $φ(x)=x^2-2x$, $x\ge 1$.
       
 \item[Β2.] \textbf{[Μονάδες 8]} Να αποδείξετε ότι η $φ$ είναι "1-1" και να βρείτε την αντίστροφη $φ^{-1}$ της $φ$.
 \item[Β3.] \textbf{[Μονάδες 5]} Να βρείτε τα κοινά σημεία της $φ^{-1}$ με την ευθεία $y=x$.
 \item[Β4.] \textbf{[Μονάδες 7]} Να σχεδιάσετε στο ίδιο σύστημα συντεταγμένων, τις γραφικές παραστάσεις των $φ$ και $φ^{-1}$.
\end{enumerate}

\section*{Θέμα Γ (10)}
Έστω η συνάρτηση $f:\mathbb{R}\to\mathbb{R}$ με $f(x)= ax+e^{-x}$, $a\in\mathbb{R}$. Αν ισχύει $ax+e^{-x}\ge 1$ με $a\ne 0$ για κάθε $x\in\mathbb{R}$ τότε:
\begin{enumerate}
 \item[Γ1.] \textbf{[Μονάδες 6]} Να αποδείξετε ότι $f(x)=x+e^{-x}$.
 \item[Γ2.] \textbf{[Μονάδες 6]} Να μελετήσετε την $f$ ως προς την μονοτονία και τα ακρότατα και να αποδείξετε ότι δεν υπάρχει οριζόνται ευθεία $y=k$ με $k\ne 1$ η οποία να τέμνει την γραφική παράσταση της $f$ σε ακριβώς ένα σημείο.
 \item[Γ3.] \textbf{[Μονάδες 5]} Να δείξετε ότι υπάρχει μοναδικό $x_0$ στο οποίο η εφαπτομένη της $C_f$ διέρχεται από την αρχή των αξόνων, την οποία και να την υπολογίσετε.
 \item[Γ4.] \textbf{[Μονάδες 8]} Ένα σημείο κινείται στη γραφική παράσταση της $f$ και η τετμημένη του αυξάνεται με ρυθμό $1 cm/s$. Τη χρονική στιγμή κατά την οποία η εφαπτόμενη της $f$ διέρχεται από την αρχή των αξόνων, να βρείτε το ρυθμό μεταβολής της γωνίας που σχηματίζει η εφαπτομένη με τον άξονα $x'x$
\end{enumerate}

\section*{Θέμα Δ (10)}
Δίνεται η παραγωγίσιμη συνάρτηση $f:\mathbb{R}\to\mathbb{R}$ για την οποία ισχύουν:
\begin{itemize}
 \item $f^2(x)(x^4+1)=\left(4e^{x^2-1}-\sqrt{2}xf(x)\right)\left(4e^{x^2-1}+\sqrt{2}xf(x)\right)$
 \item $e^x+x\ge f(1)x+1$ για κάθε $x\in\mathbb{R}$
\end{itemize}
\begin{enumerate}
 \item[Δ1.] \textbf{[Μονάδες 4]} Να αποδείξετε ότι $f(x)\ne 0$ για κάθε $x\in\mathbb{R}$ και $f(x)=\frac{4e^{x^2-1}}{x^2+1}$.
 \item[Δ2.] \textbf{[Μονάδες 6]} Να μελετήσετε την $f$ ως προς την μονοτονία και τα ακρότατα.
 \item[Δ3.] \textbf{[Μονάδες 4]} Να βρείτε τις εξισώσεις των εφαπτομένων της $C_f$ που διέρχονται από την αρχή των αξόνων.
 \item[Δ4.] \textbf{[Μονάδες 5]} Να λύσετε την ανίσωση $x^2+\ln(x+1)>x+\ln(x^2+1)$, $x>0$.
 \item[Δ5.] \textbf{[Μονάδες 6]} Να δείξετε ότι η $f$ δεν είναι "1-1" καθώς και να δείξετε:
       $$\lim_{x\to -x_0}\left[\frac{f(0)}{x_0+x}\cdot\left(\frac{4e^{x^2-1}}{x^2+1}-\frac{4e^{x_0^2-1}}{x_0^2+1}\right)\right]=-\frac{4}{e}f'(x_0)$$
\end{enumerate}

\end{document}
