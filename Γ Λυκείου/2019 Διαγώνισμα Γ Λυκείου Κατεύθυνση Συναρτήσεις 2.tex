\documentclass[12pt]{article}

\usepackage{amsmath}
\usepackage{unicode-math}
\usepackage{xltxtra}
\usepackage{xgreek}

\setmainfont{Liberation Serif}

\usepackage{tabularx}

\pagestyle{empty}

\usepackage{geometry}
 \geometry{a4paper, total={190mm,280mm}, left=10mm, top=10mm}

 \usepackage{graphicx}
 \graphicspath{ {images/} }

 \usepackage{wrapfig}

\begin{document}

\begin{table}
    \small
    \begin{tabularx}{\textwidth}{ c X r }
      \begin{tabular}{ l }
        Εισηγητής: Λόλας Κωνσταντίνος \\
        Επαναληπτικό: Συναρτήσεις
      \end{tabular}
      & &
      \begin{tabular}{ r }
        Θεσσαλονίκη, 22 / 10 / 2019
      \end{tabular}
    \end{tabularx}
\end{table}

\part*{\centering{Διαγώνισμα Κατεύθυνση Γ Λυκείου}}

\section*{Θέμα Α}
  \noindent
  \begin{enumerate}
    \item \textbf{[Μονάδες 8]} Έστω μία συνάρτηση $f$ με πεδίο ορισμού $Α$. Πότε λέμε ότι η $f$ παρουσιάζει στο $x_0\in Α$ (ολικό) ελάχιστο το $f(x_0)$;
    \item \textbf{[Μονάδες 8]} Πότε μία συνάρτηση $f:Α\to \mathbb{R}$ λέγεται 1-1;
    \item \textbf{[Μονάδες 9]} Αν $f$, $g$ είναι δύο συναρτήσεις με πεδίο ορισμού τα σύνολα $Α$, $Β$ αντιστοίχως, τι ονομάζουμε σύνθεση της $f$ με την $g$ και ποιο είναι το πεδίο ορισμού της;
  \end{enumerate}

\section*{Θέμα Β}
  \noindent
  Δίνεται η συνάρτηση $f(x):\mathbb{R}\to \mathbb{R}$ για την οποία ισχύει
  $$f(\ln x)=\frac{1}{x}-\ln x-1 \text{, για κάθε } x>0$$
  \begin{enumerate}
    \item \textbf{[Μονάδες 6]} Να δείξετε ότι $f(x)=e^{-x}-x-1$, $x\in \mathbb{R}$.
    \item \textbf{[Μονάδες 6]} Να εξετάσετε τη συνάρτηση $f$ ως προς την μονοτονία.
    \item \textbf{[Μονάδες 6]} Να λύσετε την ανίσωση $e^{-x}>x+1$.
    \item \textbf{[Μονάδες 7]} Να λύσετε την εξίσωση $(x^2+1)e^{x^2}=1$.
  \end{enumerate}
  %\vspace{7\baselineskip}

\section*{Θέμα Γ}
  \noindent
  Δίνεται η συνάρτηση $f(x)=\frac{e^x}{e^x-1}$.
  \begin{enumerate}
    \item \textbf{[Μονάδες 6]} Να βρείτε το πεδίο ορισμού της συνάρτησης $f$ και να εξετάσετε, αν η γραφική παράσταση της $f$ τέμνει τους άξονες $x'x$ και $y'y$.
    \item \textbf{[Μονάδες 6]} Να δείξετε ότι η $f$ είναι συνάρτηση 1-1.
    \item \textbf{[Μονάδες 6]} Να δείξετε ότι ορίζεται η αντίστροφη συνάρτηση $f^{-1}$ και να την βρείτε.
    \item \textbf{[Μονάδες 7]} Να λύσετε την εξίσωση
    $$f^{-1}\left(\frac{1}{1-e}+2-f(\ln x)\right)=-1$$
  \end{enumerate}

  \section*{Θέμα Δ}
    \noindent
    Δίνεται η συνάρτηση $f(x)=x+\ln(x^2+1)$, $x\ge 0$.
    \begin{enumerate}
      \item \textbf{[Μονάδες 6]} Να μελετήσετε ως προς τη μονοτονία τη συνάρτηση $f$ στο διάστημα $[0,+\infty)$ και στη συνέχεια να βρείτε την ελάχιστη τιμή της $f$.
      \item \textbf{[Μονάδες 6]} Να λύσετε την ανίσωση
      $$\ln (x^4+1)>\ln (2e)-x^2$$
      \item \textbf{[Μονάδες 6]} Να λύσετε την παρακάτω εξίσωση στο διάστημα $[-\frac{4}{3},+\infty)$
$$x^2-3x-4=\ln \frac{(3x+4)^2+1}{x^4+1}$$
      \item \textbf{[Μονάδες 7]} Να δείξετε ότι η $f$ αντιστρέφεται και στη συνέχεια να βρείτε τα κοινά σημεία των $C_f$ και $C_{f^{-1}}$, αν θεωρήσουμε γνωστό ότι η $f$ έχει σύνολο τιμών το $[0,+\infty)$ και ότι τέμνονται μόνο στη $y=x$.
    \end{enumerate}

\vspace{2\baselineskip}

\part*{\centering{Καλή επιτυχία}}

\end{document}
