\documentclass[12pt]{article}

\usepackage{amsmath}
\usepackage{unicode-math}
\usepackage{xltxtra}
\usepackage{xgreek}

\setmainfont{Liberation Serif}

\usepackage{tabularx}

\pagestyle{empty}

\usepackage{geometry}
 \geometry{a4paper, total={190mm,275mm}, left=10mm, top=10mm}

 \usepackage{graphicx}
 \graphicspath{ {images/} }

 \usepackage{wrapfig}
\usepackage{lipsum}%% a garbage package you don't need except to create examples.

\begin{document}

\begin{table}
    \small
    \begin{tabularx}{\textwidth}{ c X r }
      \begin{tabular}{ c }
        \includegraphics[scale=0.4]{ελληνική} \\
        ΕΛΛΗΝΙΚΗ ΔΗΜΟΚΡΑΤΙΑ \\
        ΥΠΟΥΡΓΕΙΟ ΠΑΙΔΕΙΑΣ, ΕΡΕΥΝΑΣ \& ΘΡΗΣΚΕΥΜΑΤΩΝ \\
        ΠΕΡΙΦΕΡΕΙΑΚΗ Δ/ΝΣΗ ΠΡΩΤ. \& ΔΕΥΤ/ΜΙΑΣ  ΕΚΠ/ΣΗΣ \\
        ΚΕΝΤΡΙΚΗΣ ΜΑΚΕΔΟΝΙΑΣ \\
        Δ/ΝΣΗ ΔΕΥΤΕΡΟΒΑΘΜΙΑΣ ΕΚΠ/ΣΗΣ ΑΝ. ΘΕΣ/ΝΙΚΗΣ \\
        27ο ΓΕΝΙΚΟ ΛΥΚΕΙΟ ΘΕΣ/ΝΙΚΗΣ
      \end{tabular}
      & &
      \begin{tabular}{ r }
        Σχολικό Έτος: 2016 - 2017 \\
        Εξ. Περίοδος: Μαΐου - Ιουνίου \\
        Μάθημα: Μαθηματικά Κατ. Γ Λυκείου \\
        Εισηγητές: Λόλας, Φρύδας, Χ''Σάββας \\ \\
        Θεσσαλονίκη, 19 / 05 / 2017
      \end{tabular}
    \end{tabularx}
\end{table}

\part*{\centering{Θέματα}}
\section*{Θέμα Α}
  \noindent
  \begin{enumerate}
    \item \textbf{[Μονάδες 15]} Να αποδείξετε ότι αν μία συνάρτηση είναι παραγωγίσιμη σε σημείο $x_0$ τότε είναι και συνεχής στο $x_0$
    \item \textbf{[Μονάδες 10]}  Να χαρακτηρίσετε τις παρακάτω προτάσεις με Σωστό ή Λάθος
    \begin{enumerate}
      \item [α)] Αν $f(x) \ge f(2)$ για κάθε $x \in D_f$ τότε στο $x=2$ έχω ολικό μέγιστο.
      \item [β)] Ισχύει $\left( ημx \right)'=-συνx$.
      \item [γ)] Αν $f'(x)>0$ για κάθε $x \in D_f$ τότε η $f$ είναι γνησίως φθίνουσα.
      \item [δ)] Αν $f'(3)=0$ τότε στο $x=3$ έχω ακρότατο.
      \item [ε)] Η συνάρτηση $f(x)=\ln x$ είναι συνεχής για $x>0$.
    \end{enumerate}
  \end{enumerate}

\section*{Θέμα Β}
  \noindent

Να υπολογιστούν τα παρακάτω όρια: \textbf{[Μονάδες 5, 5, 5, 5, 5]}
  \begin{table}[ht]
    \begin{tabularx}{\textwidth}{ X p{6em} X p{6em} X p{6em} X p{6em} X p{6em} X }
      & $$1. \lim_{x \to 0} x+1 $$ & & $$2. \lim_{x \to 1} \frac{x^2-1}{x-1} $$ & & $$3. \lim_{x \to 1^{+}} \frac{1}{x-1} $$ & & $$4. \lim_{x \to 0} \frac{ημ3x}{x} $$ & & $$5. \lim_{x \to e} \frac{\ln x-1}{x-e} $$ &
    \end{tabularx}
  \end{table}

\section*{Θέμα Γ}
  \noindent

  Έστω συνάρτηση η $$f(x)=\begin{cases} e^x-x & \text{, } x \ne 0 \\ 1 & \text{, } x = 0  \end{cases}$$.
  \begin{enumerate}
    \item \textbf{[Μονάδες 5]}  Να δείξετε ότι η $f$ είναι συνεχής στο $\mathbb{R}$.
    \item \textbf{[Μονάδες 5]}  Να δείξετε ότι η εφαπτόμενη της $C_f$ στο $x=\ln2$ είναι η $y=x+2-2\ln2$.
    \item \textbf{[Μονάδες 5]}  Να δείξετε ότι η $f$ είναι γν. φθίνουσα στο $\left(-\infty,0\right)$ και γν. αύξουσα στο $\left(0,+\infty\right)$
    \item \textbf{[Μονάδες 5]}  Να βρείτε τα ακρότατα της $f$
    \item \textbf{[Μονάδες 5]}  Να αποδείξετε ότι $e^{\pi}-e^e > \pi-e$
  \end{enumerate}

\section*{Θέμα Δ}
  \noindent

  Έστω $P(x)=x^3-x^2+2x+1$.

  \begin{enumerate}
    \item \textbf{[Μονάδες 5]}  Να βρείτε το σύνολο τιμών του $P(X)$.
    \item \textbf{[Μονάδες 5]}  Να αποδείξετε ότι το $P(x)$ έχει πραγματική ρίζα στο $(-1,0)$.
    \item \textbf{[Μονάδες 5]}  Να αποδείξετε ότι το $P(x)$ έχει μοναδική πραγματική ρίζα στο $\mathbb{R}$.
    \item \textbf{[Μονάδες 5]}  Να αποδείξετε ότι η εξίσωση $3x^4-4x^3+12x^2+12x-1=0$ έχει το πολύ δύο πραγματικές ρίζες.
    \item \textbf{[Μονάδες 5]}  Να αποδείξετε ότι για κάθε $α>1$ υπάρχει $ξ\in (1,α)$ ώστε $3ξ^2-2ξ+2=\frac{α^3-α^2+2α-2}{α-1}$.
  \end{enumerate}

\part*{\centering{Καλή επιτυχία}}
\begin{table}[htb]
    \begin{tabularx}{\textwidth}{ X c X c X}
      &
      \begin{tabular}[t]{ c }
        Ο Δ/ντης \\ \\ \\ \\
        Δρ. Ιωαννίδης Νικόλαος
      \end{tabular}
      & &
      \begin{tabular}[t]{ c }
        Οι εισηγητές \\ \\
        \multicolumn{1}{l}{1. Λόλας Κωνσταντίνος} \\ \\
        \multicolumn{1}{l}{2. Φρύδας Βασίλειος} \\ \\
        \multicolumn{1}{l}{3. Χ''Σάββας Δημήτριος}
      \end{tabular}
      &
    \end{tabularx}
\end{table}


\vfill
 \textbf{Οδηγίες}
 \begin{enumerate}
   \item Μην ξεχάσετε να γράψετε το ονοματεπώνυμό σας σε κάθε φύλλο απαντήσεων που σας δώσουν.
   \item Όλες οι απαντήσεις να δωθούν στο φύλλο απαντήσεων. Οτιδήποτε γραφτεί στη σελίδα με τα θέματα δεν θα ληφθεί υπόψιν.
   \item Τα Σωστό - Λάθος δεν χρειάζονται αιτιολόγηση.
 \end{enumerate}
\end{document}
