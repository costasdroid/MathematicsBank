\documentclass[12pt]{article}

\usepackage{amsmath}
\usepackage{unicode-math}
\usepackage{xltxtra}
\usepackage{xgreek}

\setmainfont{Liberation Serif}

\usepackage{tabularx}

\pagestyle{empty}

\usepackage{geometry}
 \geometry{a4paper, total={190mm,275mm}, left=10mm, top=10mm}

\begin{document}

  \begin{enumerate}
    \item
    \begin{enumerate}
      \item Έστω η δράση με $z \mapsto z+s$ και $y \mapsto y+t$. Τότε για να είναι σημείο του επιπέδου θα πρέπει να το επαληθεύει άρα $x \mapsto 2 \ldots$. Για τυχαία βάση θεώρησε ότι 3 σημεία θέλεις αρκεί τα διανύσματα διαφορών (ας είναι $b_1$ και $b_2$) να είναι γραμμικά ανεξάρτητα και γράψε και λύσε το σύστημα
      \begin{gather}
        (2,-2,3)-a_0=x_1b_1+x_2b_2 \\
        (2,-2,3)=a_0+x_1(a_1-a_0)+x_2(a_2-a_0) \\
        (2,-2,3)=(1-x_1-x_2)a_0+x_1a_1+x_2a_2 \ldots
      \end{gather}

      \item Βάζοντας τα $b_i$ στην αναπαράσταση του $a$ έχουμε
      \begin{gather}
        a=a_0+x_1(a_1-a_0)+x_2(a_2-a_0)+x_3(a_3-a_0) \\
        a=(1-x_1-x_2-x_3)a_0+x_1a_1+x_2a_2+x_3a_3 \\
      \end{gather}
      Άρα το $a_2$ γράφεται ως $(0,1,0)$, το $a_1$ ως $(1,0,0)$, το $a_3$ ως $(0,0,1)$, το $a_0$ ως $(0,0,0)$, άρα το $a_3-a_2=(0,-1,1)$, το $a_0-a_2=(0,-1,0)$, το $a_1-a_2=(1,-1,0)$. Έτσι
      \begin{gather}
        C=
        \begin{pmatrix}
          0 & -1 & 1 \\
          -1 & -1 & -1 \\
          1 & 0 & 0
        \end{pmatrix}
         \text{ και } d=a_2-a_0=
       \begin{pmatrix}
         0 \\
         1 \\
         0
       \end{pmatrix}
      \end{gather}

    \end{enumerate}

    \item
    \begin{enumerate}
      \item
        π.χ. για το $P_1$. Για $y=z=0$ έχεις το σημείο $a_0=(2,0,0)\ldots$. Άρα και βάσεις $(a_0,a_1-a_0,a_2-a_0)=\ldots$
      \item
        Γράψε τα $a'_1-a'_0$ και $a'_2-a'_0$ ως γραμμικό συνδιασμό των $a_1-a_0$ και $a_2-a_0$... Και για αντίστροφη εικόνα πάρε από τον $y=Ax+b$ το $x=A^{-1}y-A^{-1}b$
    \end{enumerate}

    \item
    \begin{enumerate}
      \item Όταν τα διανύσματα διαφορών $A_i-A_0$ είναι γραμμικά ανεξάρτητα. Αφινικό ανάπτυγμα είναι το σύνολο των αφινικών συνδιασμών τους. Είναι αφινική βάση του χώρου που παράγουν. Όχι γιατί αφού αποτελούν βάση κάθε σημείο θα γράφεται ως αφινικός συνδιασμός κάποιων $A_i$ άρα γρ. εξαρτημένα.
      \item
        Αν κάνουμε στοιχειώδεις γραμμοπράξεις θα έχουμε
          \[ \begin{vmatrix}
          x-a_01 & x-a_02 & x-a_03 & 0 \\
          a_01 & a_02 & a_03 & 1 \\
          a_11-a_01 & a_12-a_02 & a_13-a_03 & 0 \\
          a_21-a_01 & a_22-a_02 & a_23-a_03 & 0 \\
        \end{vmatrix} = \begin{vmatrix}
        x-a_01 & x-a_02 & x-a_03\\
        a_11-a_01 & a_12-a_02 & a_13-a_03\\
        a_21-a_01 & a_22-a_02 & a_23-a_03\\
      \end{vmatrix}=0 \] άρα αφού τα διανύσματα γραμμές 2 και 3 είναι γρ. ανεξάρτητα και η τάξη του πίνακα είναι $<3$ το διάνυσμα γραμμή 1 είναι γραμμικά εξαρτημένο από τα άλλα δύο...
      \item Αντικατέστησε τα σημεία και βρες τα $a$, $b$, $c$ και $d$ με σύστημα. Για εμβαδό, υπάρχει τύπος από Λύκειο με απόσταση σημείου από ευθεία ή από Πανεπιστήμιο με το μέτρο του διανύσματος $(a,b,c)$
  \end{enumerate}

    \item
    \begin{enumerate}
      \item
        Ορίζουν τετράεδρο γιατί έχουμε 3 γρ. ανεξάρτητα διανύσματα
      \item
        Ψάχνεις $y=Cx+b$. Λύσε το σύστημα $y=Cx$ με διανύσματα $x$ και $y$ τα σημεία των αξόνων.
        $C=\begin{pmatrix}
          3/4 & 0 & 0 \\
          0 & -3/2 & 0 \\
          0 & 0 & 1/2
        \end{pmatrix}$
        και για το $b$ αρκεί να διαλέξουμε διάνυσμα $OB-OA$ με $B$ σημείο από το ένα επίπεδο και $B$ από το άλλο.
    \end{enumerate}
  \end{enumerate}
\end{document}
