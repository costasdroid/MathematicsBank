\documentclass[12pt]{article}

\usepackage{amsmath}
\usepackage{unicode-math}
\usepackage{xltxtra}
\usepackage{xgreek}

\setmainfont{Liberation Serif}

\usepackage{tabularx}

\pagestyle{empty}

\usepackage{geometry}
 \geometry{a4paper, total={190mm,280mm}, left=10mm, top=10mm}

 \usepackage{graphicx}
 \graphicspath{ {images/} }

 \usepackage{wrapfig}

\begin{document}

\begin{table}
 \small
 \begin{tabularx}{\textwidth}{ c X r }
  \begin{tabular}{ l }
   Εισηγητής: Λόλας Κωνσταντίνος \\
   Τεστ: 2βάθμιες Ανισώσεις - Αριθμητική Πρόοδος
  \end{tabular}
   &  &
  \begin{tabular}{ r }
   Θεσσαλονίκη, 06 / 03 / 2020
  \end{tabular}
 \end{tabularx}
\end{table}

\part*{\centering{Διαγώνισμα Άλγεβρα Α Λυκείου}}

\section*{Θέμα Α}
Δίνεται αριθμητική πρόοδος $(α_ν)$, όπου $ν\in \mathbb{N}$. Αν οι τρεις πρώτοι όροι της προόδου είναι:
$$α_1=x \text{, } α_2=2x^2-3x-4 \text{, } α_3=x^2-2$$
τότε
\begin{enumerate}
 \item Να δείξετε ότι $x=3$
 \item Να βρεθεί ο ν-οστός όρος της προόδου και να βρείτε τον $α_{2020}$.
 \item Να υπολογίσετε το άθροισμα $S=α_1+α_2+ \cdots +α_{15}$.
\end{enumerate}

\section*{Θέμα Β}
Δίνεται το τριώνυμο $P(x)=x^2-(3λ-2)x-3λ+2$
\begin{enumerate}
 \item Να δείξετε ότι η διακρίνουσα της εξίσωσης $P(x)=0$ είναι η $Δ=9λ^2-4$.
 \item Να δείξετε ότι η εξίσωση $P(x)=0$ έχει πραγματικές ρίζες για $λ\in \left(-\infty,-\frac{2}{3}\right] \cup \left[\frac{2}{3},\infty\right)$.
 \item Αν $x_1$, $x_2$ οι δύο πραγματικές ρίζες της εξίσωσης $P(x)=0$ να βρείτε τις τιμές του $λ$ ώστε $x_1^2x_2+x_1x_2^2<-9$.
 \item Να βρείτε το πρόσημο της παράστασης $1,61804^2-(3-2)1,61804-3+2$.
\end{enumerate}
Δίνεται ότι $\frac{1+\sqrt{5}}{2}=1.61803...$


\part*{\centering{Καλή επιτυχία}}

\end{document}
