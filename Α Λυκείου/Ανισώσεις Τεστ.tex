\documentclass[12pt]{extarticle}

\usepackage{amsmath}
\usepackage{unicode-math}
\usepackage{xltxtra}
\usepackage{xgreek}

\setmainfont{Liberation Serif}

\usepackage{tabularx}

\pagestyle{empty}

\usepackage{geometry}
\geometry{a4paper, total={190mm,275mm}, left=10mm, top=10mm}

\usepackage{graphicx}
\graphicspath{ {images/} }

\usepackage{wrapfig}

\begin{document}
% \renewcommand{\labelenumi}{\alph{enumi})}
\renewcommand{\labelenumii}{\roman{enumii}.}

\part*{\centering{Ανισώσεις}}
\noindent

\section*{\centering{Τεστ}}

\begin{enumerate}
    \item Να αποδείξετε ότι για κάθε $x\in\mathbb{R}$ $$\frac{x^2+9}{6}\ge x$$
    \hfill \strong{Μονάδες 5}
    \item Να βρείτε τα $α$ και $β$, αν $$α^2+β^2-4β+4=0$$
    \hfill \strong{Μονάδες 6}
    \item Αν $2<x\le 3$ και $-3\le y <1$, να βρείτε τα όρια μεταξύ των οποίων περιέχεται η τιμή κάθε μίας από τις παρακάτω παραστάσεις
    \begin{enumerate}
        \item $2x-3$
        \item $x-y$
        \item $y^2$
    \end{enumerate} 
    \hfill \strong{Μονάδες 9}
\end{enumerate}

\end{document}