\documentclass[12pt]{article}

\usepackage{amsmath}
\usepackage{unicode-math}
\usepackage{xltxtra}
\usepackage{xgreek}

\setmainfont{Liberation Serif}

\usepackage{tabularx}

\pagestyle{empty}

\usepackage{geometry}
 \geometry{a4paper, total={190mm,280mm}, left=10mm, top=10mm}

 \usepackage{graphicx}
 \graphicspath{ {images/} }

 \usepackage{wrapfig}

\begin{document}

\begin{table}
    \small
    \begin{tabularx}{\textwidth}{ c X r }
      \begin{tabular}{ l }
        Εισηγητής: Λόλας Κωνσταντίνος \\
        Επαναληπτικό: 2.2, 2.3, 2.4, 3.1
      \end{tabular}
      & &
      \begin{tabular}{ r }
        Θεσσαλονίκη, 08 & 10 / 01 / 2018
      \end{tabular}
    \end{tabularx}
\end{table}

\part*{\centering{Διαγώνισμα Άλγεβρα Α Λυκείου}}

\section*{Θέμα Α}
  \noindent
  \begin{enumerate}
    \item \textbf{[Μονάδες 10]} Να αποδείξετε ότι για κάθε $α\ge 0$, $β\ge 0$, ισχύει $| α\cdot β | =|α|\cdot |β|$.
    \item \textbf{[Μονάδες 3/5]} Να δώσετε την γεωμετρική ερμηνεία της ανίσωσης $|α-β|<γ$ με $γ>0$ και να βρείτε πού βρίσκεται το $α$ ως συνάρτηση των $β$ και $γ$.
    \item \textbf{[Μονάδες 10]} Να χαρακτηρίσετε τις παρακάτω προτάσεις με Σωστό ή Λάθος
    \begin{enumerate}
      \item [α)] $\sqrt{x+y}=\sqrt{x}+\sqrt{y}$.
      \item [β)] $α^2+β^2\le 0 \Rightarrow α=β=0$.
      \item [γ)] $α<β \Rightarrow α^2<β^2$.
      \item [δ)] $|-α|=|α|=\left| -|α| \right|$.
      \item [ε)] $|α|+|β|=|α+β|$.
    \end{enumerate}
  \end{enumerate}

\section*{Θέμα Β}
  \noindent
  Έστω $Α=\sqrt{2}\sqrt[3]{3\sqrt{3}}$ και $Β=\frac{1}{5-2Α}+\frac{1}{5+2Α}$.
  \begin{enumerate}
    \item \textbf{[Μονάδες 8]} Να δείξετε ότι $Α=\sqrt{6}$.
    \item \textbf{[Μονάδες 9]} Να δείξετε ότι $Β=10$.
    \item \textbf{[Μονάδες 7]} Αν $|x-A|<B$ να βρεθεί το διάστημα στο οποίο ανήκει ο αριθμός $x$.
  \end{enumerate}
  %\vspace{7\baselineskip}

\section*{Θέμα Γ}
  \noindent
  Έστω ότι $-1\le α<3$, $-2<β\le 2$ και $γ\in\mathbb{R}$
  \begin{enumerate}
    \item \textbf{[Μονάδες 6]} Να δείξετε ότι $αβ-2\le 2α-β$.
    \item \textbf{[Μονάδες 6]} Να δείξετε ότι $γ^2-6γ \ge -9$.
    \item \textbf{[Μονάδες 6]} Να δείξετε ότι $-8\le 2α-3β \le 12$.
    \item \textbf{[Μονάδες 7]} Αν επιπλέον ισχύει $2α-αβ-β+γ^2-6γ+11=0$ να βρείτε τα $α$, $β$ και $γ$.
  \end{enumerate}
  %\vspace{7\baselineskip}

\section*{Θέμα Δ}
  \noindent
  Να βρείτε το $λ$ ώστε η εξίσωση $(λ^2-4)x=λ^2-2λ$:
  \begin{enumerate}
    \item \textbf{[Μονάδες 7]} Να είναι αδύνατη.
    \item \textbf{[Μονάδες 7]} Να είναι ταυτότητα.
    \item \textbf{[Μονάδες 7]} Να έχει μοναδική λύση.
    \item \textbf{[Μονάδες 4]} Να έχει λύση μόνο το $x=λ$.
  \end{enumerate}

\vspace{3\baselineskip}

\part*{\centering{Καλή επιτυχία}}

\end{document}
