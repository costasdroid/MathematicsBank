\documentclass[12pt]{article}

\usepackage{amsmath}
\usepackage{unicode-math}
\usepackage{xltxtra}
\usepackage{xgreek}

\setmainfont{Liberation Serif}

\usepackage{tabularx}

\pagestyle{empty}

\usepackage{geometry}
 \geometry{a4paper, total={190mm,280mm}, left=10mm, top=10mm}

 \usepackage{graphicx}
 \graphicspath{ {images/} }

 \usepackage{wrapfig}

\begin{document}

\begin{table}
 \small
 \begin{tabularx}{\textwidth}{ c X r }
  \begin{tabular}{ l }
   Εισηγητής: Λόλας Κωνσταντίνος \\
   Τεστ: Ανισώσεις
  \end{tabular}
   &  &
  \begin{tabular}{ r }
   Θεσσαλονίκη, 22 / 10 / 2020
  \end{tabular}
 \end{tabularx}
\end{table}

\part*{\centering{Τεστ Άλγεβρα Α Λυκείου}}

\section*{Θέμα Α}
Αν $2\le x < 5$ και $1\le y\le 3$, να βρείτε μεταξύ ποιών τιμών είναι η παράσταση $2x^2-3xy$

\section*{Θέμα Β}
Να βρείτε τους πραγματικούς αριθμούς $κ$ και $λ$ για τους οποίους ισχύει ότι
$$\frac{κ^2+λ^2}{2}=κ-λ-1$$

\section*{Θέμα Γ}
\begin{enumerate}
 \item Να δείξετε ότι για κάθε $α>0$ ισχύει $a+\frac{1}{a}\ge 2$
 \item Κάποιος ισχυρίζεται ότι $\frac{\pi}{6\sqrt{2}}+\frac{3\sqrt{2}}{2\pi} > 1$. Έχει δίκιο ή όχι και γιατί?
\end{enumerate}

\section*{Θέμα Δ}
Να χαρακτηρίσετε τις παρακάτω προτάσεις με Σωστό ή Λάθος
\begin{enumerate}
 \item Με $γ>0$, τότε $αγ<βγ$ $\Rightarrow$ $α<γ$ για $α$, $β$, $γ\in\mathbb{R}$
 \item $α<1 \Rightarrow α<α^2$
 \item $α<β \iff α^ν<β^ν$ για κάθε $α$, $β$, $ν\in\mathbb{N}$
 \item $α^2>0 \Rightarrow α\ne 0$ για κάθε $α\in\mathbb{R}$
 \item Αν $α<β$ και $γ<δ$ τότε $α-γ<β-δ$ για $α$, $β$, $γ$, $δ\in\mathbb{R}$
\end{enumerate}

\part*{\centering{Καλή επιτυχία}}

\end{document}
