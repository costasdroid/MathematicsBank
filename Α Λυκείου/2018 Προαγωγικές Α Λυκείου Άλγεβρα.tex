\documentclass[12pt]{article}

\usepackage{amsmath}
\usepackage{unicode-math}
\usepackage{xltxtra}
\usepackage{xgreek}

\setmainfont{Liberation Serif}

\usepackage{tabularx}

\pagestyle{empty}

\usepackage{geometry}
 \geometry{a4paper, total={190mm,280mm}, left=10mm, top=10mm}

 \usepackage{graphicx}
 \graphicspath{ {images/} }

 \usepackage{wrapfig}
\usepackage{lipsum}%% a garbage package you don't need except to create examples.

\begin{document}

\begin{table}
    \small
    \begin{tabularx}{\textwidth}{ c X r }
      \begin{tabular}{ c }
        \includegraphics[scale=0.4]{logo} \\
        ΕΛΛΗΝΙΚΗ ΔΗΜΟΚΡΑΤΙΑ \\
        ΥΠΟΥΡΓΕΙΟ ΠΑΙΔΕΙΑΣ, ΕΡΕΥΝΑΣ \& ΘΡΗΣΚΕΥΜΑΤΩΝ \\
        ΠΕΡΙΦΕΡΕΙΑΚΗ Δ/ΝΣΗ ΠΡΩΤ. \& ΔΕΥΤ/ΜΙΑΣ  ΕΚΠ/ΣΗΣ \\
        ΚΕΝΤΡΙΚΗΣ ΜΑΚΕΔΟΝΙΑΣ \\
        Δ/ΝΣΗ ΔΕΥΤΕΡΟΒΑΘΜΙΑΣ ΕΚΠ/ΣΗΣ ΑΝ. ΘΕΣ/ΝΙΚΗΣ \\
        10ο ΓΕΝΙΚΟ ΛΥΚΕΙΟ ΘΕΣ/ΝΙΚΗΣ
      \end{tabular}
      & &
      \begin{tabular}{ r }
        Σχολικό Έτος: 2017 - 2018 \\
        Εξ. Περίοδος: Μαΐου - Ιουνίου \\
        Μάθημα: Άλγεβρα Α Λυκείου\\
        Εισηγητές: Λόλας, Αξινιάρης \\ \\
        Θεσσαλονίκη, 14 / 06 / 2018
      \end{tabular}
    \end{tabularx}
\end{table}

\part*{\centering{Θέματα}}

\section*{Θέμα Α}
  \noindent
  \begin{enumerate}
    \item \textbf{[Μονάδες 15]} Αν η εξίσωση $αx^2+βx+γ=0$, $α\ne 0$, έχει δύο ρίζες πραγματικές τις $x_1$ και $x_2$, να δείξετε ότι
      \begin{enumerate}
        \item [α)] $x_1+x_2=-\frac{β}{α}$
        \item [β)] $x_1\cdot x_2=\frac{γ}{α}$
      \end{enumerate}
    \item \textbf{[Μονάδες 10]} Να χαρακτηρίσετε τις παρακάτω προτάσεις με Σωστό ή Λάθος
    \begin{enumerate}
      \item [α)] Αν $ν$ άρτιος τότε $\sqrt[ν]{α^ν}=α$.
      \item [β)] Η απόσταση δύο πραγματικών αριθμών είναι $d(α,β)=|β-α|$.
      \item [γ)] Ο αναδρομικός τύπος μίας αριθμητικής προόδου είναι $α_ν=α_{ν-1}+ν\cdot ω$.
      \item [δ)] Αν $α^2+β^2=0$ τότε $α=β=0$.
      \item [ε)] Η ανίσωση $|x|<α$ έχει πάντα λύσεις τις $-α<x<α$.
    \end{enumerate}
  \end{enumerate}

\section*{Θέμα Β}
  \noindent
    \begin{enumerate}
      \item \textbf{[Μονάδες 15]} Να λυθεί η εξίσωση $$\frac{| 2x+1|}{2}-1=| 2x+1|-\frac{4| 2x+1|+5}{3}$$
      \item \textbf{[Μονάδες 10]} Να λυθεί η ανίσωση $$\frac{2| x-2|-3}{4}-\frac{2-| x-2|}{3}<1-3| x-2|$$
    \end{enumerate}

\section*{Θέμα Γ}
  \noindent
  \begin{enumerate}
    \item \textbf{[Μονάδες 10]} Αν οι αριθμοί $x$, $2x+3$, $2x+7$ είναι διαδοχικοί όροι αριθμητικής προόδου, να βρείτε το $x$.
    \item Για $x=1$
    \begin{enumerate}
      \item[α)] \textbf{[Μονάδες 10]} Αν ο $2x+7$ είναι ο δέκατος όρος της αριθμητικής προόδου του προηγούμενου ερωτήματος, να βρείτε την διαφορά $ω$ και τον πρώτο όρο $α_1$ της αριθμητικής προόδου.
      \item[β)] \textbf{[Μονάδες 5]} Να υπολογίσετε το άρθροισμα των 40 πρώτων όρων της αριθμητικής προόδου.
    \end{enumerate}
  \end{enumerate}

\section*{Θέμα Δ}
  \noindent
  Δίνεται η εξίσωση: $x^2-(2λ-1)x+1=0$, $λ\in \mathbb{R}$.
  \begin{enumerate}
    \item \textbf{[Μονάδες 5]} Να δείξετε ότι η διακρίνουσα της εξίσωσης είναι $Δ=4λ^2-4λ-3$.
    \item \textbf{[Μονάδες 10]} Να βρείτε τις τιμές του $λ\in\mathbb{R}$ ώστε η εξίσωση να έχει δύο ρίζες πραγματικές και άνισες.
    \item \textbf{[Μονάδες 10]} Αν $x_1$ και $x_2$ είναι οι ρίζες της εξίσωσης, να βρείτε τις τιμές του $λ\in\mathbb{R}$ ώστε να ισχύει $$\frac{1}{x_1}+\frac{1}{x_2}<1$$
  \end{enumerate}

\vspace{3\baselineskip}

\part*{\centering{Καλή επιτυχία}}
\begin{table}[htb]
    \begin{tabularx}{\textwidth}{ X c X c X}
      &
      \begin{tabular}[t]{ c }
        Ο Δ/ντης
        \\ \\ \\ \\ \\
        Παπαδημητρίου Χρήστος
      \end{tabular}
      & &
      \begin{tabular}[t]{ c }
        Οι εισηγητές \\ \\
        \multicolumn{1}{l}{1. Λόλας Κωνσταντίνος} \\ \\
        \multicolumn{1}{l}{2. Αξινιάρης Βασίλης}
      \end{tabular}
      &
    \end{tabularx}
\end{table}

\vspace*{\fill}
 \textbf{Οδηγίες}
 \begin{enumerate}
   \item Να απαντήσετε σε όλα τα θέματα
   \item Μην ξεχάσετε να γράψετε το ονοματεπώνυμό σας σε κάθε φύλλο που σας δώσουν.
   \item Όλες οι απαντήσεις να δωθούν στο φύλλο απαντήσεων. Οτιδήποτε γραφτεί στη σελίδα με τα θέματα δεν θα ληφθεί υπόψιν.
   \item Τα Σωστό - Λάθος δεν χρειάζονται αιτιολόγηση.
 \end{enumerate}
\end{document}
