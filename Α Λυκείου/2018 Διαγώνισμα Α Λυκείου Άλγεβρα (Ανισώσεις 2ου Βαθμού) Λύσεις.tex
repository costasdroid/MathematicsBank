\documentclass[12pt]{article}

\usepackage{amsmath}
\usepackage{unicode-math}
\usepackage{xltxtra}
\usepackage{xgreek}

\setmainfont{Liberation Serif}

\usepackage{tabularx}

\pagestyle{empty}

\usepackage{geometry}
 \geometry{a4paper, total={190mm,280mm}, left=10mm, top=10mm}

 \usepackage{graphicx}
 \graphicspath{ {images/} }

 \usepackage{wrapfig}

\begin{document}

\begin{table}
    \small
    \begin{tabularx}{\textwidth}{ c X r }
      \begin{tabular}{ l }
        Εισηγητής: Λόλας Κωνσταντίνος \\
        Τεστ: 2βάθμιες Ανισώσεις
      \end{tabular}
      & &
      \begin{tabular}{ r }
        Ομάδα: Α \\
        Θεσσαλονίκη, 19 / 03 / 2018
      \end{tabular}
    \end{tabularx}
\end{table}

\part*{\centering{Διαγώνισμα Άλγεβρα Α Λυκείου Λύσεις}}

\begin{enumerate}
  \item $$Δ=(-(3λ-2))^2-4(-3λ+2)=9λ^2-12λ+4+12λ-8=9λ^2-4$$
  \item $$Δ\ge 0 \Rightarrow 9λ^2-4\ge 0$$ και με πινακάκι $λ\in \left(-\infty,-\frac{2}{3}\right] \cup \left[\frac{2}{3},\infty\right)$.
  \item $$x_1^2x_2+x_1x_2^2<-9 \Rightarrow x_1x_2(x_1+x_2)<-9 \Rightarrow P\cdot S<-9 \Rightarrow (-3λ+2)(3λ-2)<-9$$ και με πράξεις $$9λ^2-12λ-5>0$$ και με πινακάκι $λ>\frac{5}{3}$ ή $λ<-\frac{1}{3}$.
  \item Για $λ=1$ η παράσταση έχει 2 ρίζες θετικές τις $λ=\frac{1-\sqrt{5}}{2}$ και $$λ=\frac{1+\sqrt{5}}{2}$$. Έξω από τις ρίζες η παράσταση είναι θετική και αφού $\frac{1+\sqrt{5}}{2}\approx \frac{1+2.5}{2} = 1,75$ και η τιμή που ελέγχω είναι η $1,8293>1,75$ η παράσταση είναι θετική.
\end{enumerate}

\pagebreak

\begin{table}
    \small
    \begin{tabularx}{\textwidth}{ c X r }
      \begin{tabular}{ l }
        Εισηγητής: Λόλας Κωνσταντίνος \\
        Τεστ: 2βάθμιες Ανισώσεις
      \end{tabular}
      & &
      \begin{tabular}{ r }
        Ομάδα: Β \\
        Θεσσαλονίκη, 19 / 03 / 2018
      \end{tabular}
    \end{tabularx}
\end{table}

\part*{\centering{Διαγώνισμα Άλγεβρα Α Λυκείου Λύσεις}}

\begin{enumerate}
  \item $$Δ=(-2(λ-1))^2-4(-2λ+2)=4(λ-1)^2+8λ-8=\ldots=4λ^2-4$$
  \item $$Δ\ge 0 \Rightarrow 4λ^2-4\ge 0$$ και με πινακάκι $λ\in \left(-\infty,-1\right] \cup \left[1,\infty\right)$
  \item $$x_1^2x_2+x_1x_2^2<-4 \Rightarrow x_1x_2(x_1+x_2)<-4 \Rightarrow P\cdot S<-4 \Rightarrow (-2λ+2)(2(λ-1))<-4$$ και με πράξεις $$λ^2-2λ>0$$ και με πινακάκι $λ<0$ ή $λ>2$.
  \item η παράσταση $$x^2-2x$$ είναι θετική για $x>2$ και $x<0$. Αφού το $2,95641>2$ η παράσταση είναι θετική.
\end{enumerate}

\end{document}
