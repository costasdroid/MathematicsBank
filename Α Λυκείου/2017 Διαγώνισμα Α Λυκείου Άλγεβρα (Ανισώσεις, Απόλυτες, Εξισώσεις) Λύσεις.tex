\documentclass[12pt]{article}

\usepackage{amsmath}
\usepackage{unicode-math}
\usepackage{xltxtra}
\usepackage{xgreek}

\setmainfont{Liberation Serif}

\usepackage{tabularx}

\pagestyle{empty}

\usepackage{geometry}
 \geometry{a4paper, total={190mm,280mm}, left=10mm, top=10mm}

 \usepackage{graphicx}
 \graphicspath{ {images/} }

 \usepackage{wrapfig}

\begin{document}

\part*{\centering{Λύσεις}}

\section*{Θέμα Α}
  \noindent
  \begin{enumerate}
    \item Απόδειξη από σχολικό βιβλίο.
    \item Η απόσταση του $α$ από το $β$ είναι μικρότερη του $γ$. Τότε θα ισχύει και $β-γ<α<β+γ$
    \item Να χαρακτηρίσετε τις παρακάτω προτάσεις με Σωστό ή Λάθος
    \begin{enumerate}
      \item [α)] Λάθος
      \item [β)] Σωστό
      \item [γ)] Λάθος
      \item [δ)] Σωστό
      \item [ε)] Λάθος
    \end{enumerate}
  \end{enumerate}

\section*{Θέμα Β}
  \noindent
  \begin{enumerate}
    \item $Α=\sqrt{2}\sqrt[3]{3\sqrt{3}}=\sqrt{2}\sqrt[3]{\sqrt{3^3}}=\sqrt{2}\sqrt[6]{3^3}=\sqrt{2}\sqrt{3}=\sqrt{6}$.
    \item Με συζυγή...
    $$Β=\frac{1}{5-2\sqrt{6}}+\frac{1}{5+2\sqrt{6}}=\frac{5+2\sqrt{6}}{25-24}+\frac{5-2\sqrt{6}}{25-24}=10$$
    \item $\sqrt{6}-10<x<\sqrt{6}+10$
  \end{enumerate}
  %\vspace{7\baselineskip}

\section*{Θέμα Γ}
  \noindent
  \begin{enumerate}
    \item  $αβ-2\le 2α-β\Leftrightarrow αβ+β-2α-2\le 0 \Leftrightarrow (α+1)β-2(α+1)\le 0\Leftrightarrow (α+1)(β-2)\le 0$.
    \item $γ^2-6γ \ge -9\Leftrightarrow γ^2-6γ +9\ge 0\Leftrightarrow (γ-3)^2\ge 0$
    \item $-2\le 2α < 6$ και $-6< -3β \le 6$, άρα...
    \item Από το πρώτο και από το 2ο ερώτημα $2α-αβ-β+2\ge 0$ και $γ^2-6γ+9 \ge 0$ άρα
    ή $α=-1$, $β\in\mathbb{R}$ και $γ=3$
    ή $α\in\mathbb{R}$, $β=2$ και $γ=3$
  \end{enumerate}

\section*{Θέμα Δ}
  \noindent
  $λ^2-4=0 \Leftrightarrow λ=\pm 2$
  \begin{enumerate}
    \item $λ=-2$
    \item $λ=2$
    \item $λ\ne \pm 2$
    \item $λ\ne 2$ και $(λ-2)(λ+2)λ=λ(λ-2)\Leftrightarrow λ(λ-2)(λ+2-1)=0$ άρα
    $λ=0$ ή $λ=-1$
  \end{enumerate}

\vspace{3\baselineskip}

\end{document}
