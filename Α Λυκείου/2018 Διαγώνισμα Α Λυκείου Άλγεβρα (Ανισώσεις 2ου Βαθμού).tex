\documentclass[12pt]{article}

\usepackage{amsmath}
\usepackage{unicode-math}
\usepackage{xltxtra}
\usepackage{xgreek}

\setmainfont{Liberation Serif}

\usepackage{tabularx}

\pagestyle{empty}

\usepackage{geometry}
 \geometry{a4paper, total={190mm,280mm}, left=10mm, top=10mm}

 \usepackage{graphicx}
 \graphicspath{ {images/} }

 \usepackage{wrapfig}

\begin{document}

\begin{table}
    \small
    \begin{tabularx}{\textwidth}{ c X r }
      \begin{tabular}{ l }
        Εισηγητής: Λόλας Κωνσταντίνος \\
        Τεστ: 2βάθμιες Ανισώσεις
      \end{tabular}
      & &
      \begin{tabular}{ r }
        Ομάδα: Α \\
        Θεσσαλονίκη, 19 / 03 / 2018
      \end{tabular}
    \end{tabularx}
\end{table}

\part*{\centering{Διαγώνισμα Άλγεβρα Α Λυκείου}}

Δίνεται το τριώνυμο $P(x)=x^2-(3λ-2)x-3λ+2$\begin{enumerate}
  \item Να δείξετε ότι η διακρίνουσα της εξίσωσης $P(x)=0$ είναι η $Δ=9λ^2-4$.
  \item Να δείξετε ότι η εξίσωση $P(x)=0$ έχει πραγματικές ρίζες για $λ\in \left(-\infty,-\frac{2}{3}\right] \cup \left[\frac{2}{3},\infty\right)$.
  \item Αν $x_1$, $x_2$ οι δύο πραγματικές ρίζες της εξίσωσης $P(x)=0$ να βρείτε τις τιμές του $λ$ ώστε $x_1^2x_2+x_1x_2^2<-9$.
  \item Να βρείτε το πρόσημο της παράστασης $1,8293^2-(3-2)1,8293-3+2$.
\end{enumerate}

\part*{\centering{Καλή επιτυχία}}

\pagebreak

\begin{table}
    \small
    \begin{tabularx}{\textwidth}{ c X r }
      \begin{tabular}{ l }
        Εισηγητής: Λόλας Κωνσταντίνος \\
        Τεστ: 2βάθμιες Ανισώσεις
      \end{tabular}
      & &
      \begin{tabular}{ r }
        Ομάδα: Β \\
        Θεσσαλονίκη, 19 / 03 / 2018
      \end{tabular}
    \end{tabularx}
\end{table}

\part*{\centering{Διαγώνισμα Άλγεβρα Α Λυκείου}}

Δίνεται το τριώνυμο $P(x)=x^2-2(λ-1)x-2λ+2$\begin{enumerate}
  \item Να δείξετε ότι η διακρίνουσα της εξίσωσης $P(x)=0$ είναι η $Δ=4λ^2-4$.
  \item Να δείξετε ότι η εξίσωση $P(x)=0$ έχει πραγματικές ρίζες για $λ\in \left(-\infty,-1\right] \cup \left[1,\infty\right)$.
  \item Αν $x_1$, $x_2$ οι δύο πραγματικές ρίζες της εξίσωσης $P(x)=0$ να βρείτε τις τιμές του $λ$ ώστε $x_1^2x_2+x_1x_2^2<-4$.
  \item Να βρείτε το πρόσημο της παράστασης $2,95641^2-2(2-1)2,95641-2+2$.
\end{enumerate}

\part*{\centering{Καλή επιτυχία}}

\end{document}
