\documentclass[12pt]{article}

\usepackage{amsmath}
\usepackage{unicode-math}
\usepackage{xltxtra}
\usepackage{xgreek}

\setmainfont{Liberation Serif}

\usepackage{tabularx}

\pagestyle{empty}

\usepackage{geometry}
 \geometry{a4paper, total={190mm,280mm}, left=10mm, top=10mm}

 \usepackage{graphicx}
 \graphicspath{ {images/} }

 \usepackage{wrapfig}
\usepackage{lipsum}%% a garbage package you don't need except to create examples.

\begin{document}

\part*{\centering{Θέματα}}

\section*{Θέμα Α}
  \noindent
  \begin{enumerate}
    \item \textbf{[Μονάδες 15]} Αν η εξίσωση $αx^2+βx+γ=0$, $α\ne 0$, έχει δύο ρίζες πραγματικές τις $x_1$ και $x_2$, να δείξετε ότι
      \begin{enumerate}
        \item [α)] $x_1+x_2=-\frac{β}{α}$
        \item [β)] $x_1\cdot x_2=\frac{γ}{α}$
      \end{enumerate}
    \item \textbf{[Μονάδες 10]} Να χαρακτηρίσετε τις παρακάτω προτάσεις με Σωστό ή Λάθος
    \begin{enumerate}
      \item [α)] Για $α\cdot β\ge 0$ ισχύει $\sqrt{α}+\sqrt{β}=\sqrt{α+β}$
      \item [β)] Αν $θ>0$, τότε $|x|>θ\iff -θ<x<θ$
      \item [γ)] Αν η εξίσωση $αx^2+βx+γ=0$, $α\ne 0 $ έχει $Δ>0$ τότε έχει δύο ρίζες πραγματικές και άνισες.
      \item [δ)] Αν $α_ν$ αριθμητική πρόοδος με διαφορά $ω$, τότε $α_ν=α_1+(ν-1)ω$
      \item [ε)] Αν $α_ν$ γεωμετρική πρόοδος με λόγο $λ\ne 0$, τότε $α_{ν+1}=α_ν\cdot λ$
    \end{enumerate}
  \end{enumerate}

\section*{Θέμα Β (14512)}
  \noindent
    \begin{enumerate}
      \item [α)] \textbf{[Μονάδες 9]} Να λύσετε τις εξισώσεις $x^2=1$ και $x^2=9$
      \item [β)] Να διατάξετε τις λύσεις των εξισώσεων του α) ερωτήματος σε αύξουσα σειρά και στη συνέχεια
        \begin{enumerate}
          \item [i.] \textbf{[Μονάδες 9]} να δείξετε ότι με αυτή τη σειρά αποτελούν διαδοχικούς αριθμούς αριθμητικής προόδου $(α_ν)$ της οποίας να βρείτε την διαφορά $ω$.
          \item [ii.] \textbf{[Μονάδες 7]} να δείξετε ότι ο αριθμός 46 δεν αποτελεί όρο της προόδου $(α_ν)$
        \end{enumerate}
    \end{enumerate}

\section*{Θέμα Γ}
  \noindent
  \begin{enumerate}
    \item \textbf{[Μονάδες 10]} Να λύσετε την εξίσωση $|x-3|=2$.
    \item \textbf{[Μονάδες 10]} Να λύσετε την ανίσωση $|x+1|\ge 2$.
    \item \textbf{[Μονάδες 5]} Να βρείτε τις κοινές λύσεις της εξίσωσης και της ανίσωσης.
  \end{enumerate}

\section*{Θέμα Δ (1486)}
  \noindent
  Δίνεται το τριώνυμο $f(x)=x^2-6x+λ-3$, $λ\in \mathbb{R}$
  \begin{enumerate}
    \item [α)] \textbf{[Μονάδες 5]} Να υπολογίσετε την διακρίνουσα $Δ$ του τριωνύμου.
    \item [β)] \textbf{[Μονάδες 7]} Να βρείτε τις τιμές του $λ$ για τις οποίες το τριώνυμο έχει δύο άνισες πραγματικές ρίζες.
    \item [γ)] Αν $3<λ<12$, τότε:
    \begin{enumerate}
      \item [(i)] \textbf{[Μονάδες 6]} Να δείξετε ότι το τριώνυμο έχει δύο άνισες θετικές ρίζες.
      \item [(ii)] \textbf{[Μονάδες 7]} Αν $x_1$, $x_2$ με $x_1<x_2$ είναι οι δύο ρίζες του τριωνύμου και $κ$, $μ$ είναι δύο αριθμοί με $κ<0$ και $x_1<μ<x_2$, να προσδιορίσετε το πρόσημο του γινομένου $κ\cdot f(κ)\cdot μ \cdot f(μ)$. Να αιτιολογήσετε την απάντησή σας.
    \end{enumerate}
  \end{enumerate}

\vspace{3\baselineskip}

\part*{\centering{Καλή επιτυχία}}
\begin{table}[htb]
    \begin{tabularx}{\textwidth}{ X c X c X}
      &
      \begin{tabular}[t]{ c }
        Ο Δ/ντης
        \\ \\ \\ \\ \\
        Παπαδημητρίου Χρήστος
      \end{tabular}
      & &
      \begin{tabular}[t]{ c }
        Οι εισηγητές \\ \\
        \multicolumn{1}{l}{1. Λόλας Κωνσταντίνος} \\ \\
        \multicolumn{1}{l}{2. Αδάμ Μιλτιάδης}
      \end{tabular}
      &
    \end{tabularx}
\end{table}

\vspace*{\fill}
 \textbf{Οδηγίες}
 \begin{enumerate}
   \item Να απαντήσετε σε όλα τα θέματα
   \item Μην ξεχάσετε να γράψετε το ονοματεπώνυμό σας σε κάθε φύλλο που σας δώσουν.
   \item Όλες οι απαντήσεις να δωθούν στο φύλλο απαντήσεων. Οτιδήποτε γραφτεί στη σελίδα με τα θέματα δεν θα ληφθεί υπόψιν.
   \item Τα Σωστό - Λάθος δεν χρειάζονται αιτιολόγηση.
 \end{enumerate}
\end{document}
