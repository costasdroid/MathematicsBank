\documentclass[12pt]{article}

\usepackage{amsmath}
\usepackage{unicode-math}
\usepackage{xltxtra}
\usepackage{xgreek}

\setmainfont{Liberation Serif}

\usepackage{tabularx}

\pagestyle{empty}

\usepackage{geometry}
 \geometry{a4paper, total={190mm,280mm}, left=10mm, top=10mm}

 \usepackage{graphicx}
 \graphicspath{ {images/} }

 \usepackage{wrapfig}
\usepackage{lipsum}%% a garbage package you don't need except to create examples.

\begin{document}

\part*{\centering{Λύσεις}}

\section*{Θέμα Α}
  \noindent
  \begin{enumerate}
    \item Θεωρία
    \item Λ, Λ, Σ, Σ, Σ
  \end{enumerate}

\section*{Θέμα Β (14512)}
  \noindent
    \begin{enumerate}
      \item [α)] $x=\pm 1$, $x=\pm 3$
      \item [β)] $\{-3,-1,1,3\}$
        \begin{enumerate}
          \item [i.] $-3+2=-1$, $-1+2=1$, $1+2=3$ $\implies ω=2$
          \item [ii.] η αριθμητική πρόοδος αποτελείται από περιττούς
        \end{enumerate}
    \end{enumerate}

\section*{Θέμα Γ}
  \noindent
  \begin{enumerate}
    \item $x=5$ ή $x=1$
    \item $x\ge 1$ ή $x\le -3$
    \item Η κοινές λύσεις είναι οι $\{1,5\}$
  \end{enumerate}

\section*{Θέμα Δ (1486)}
  \noindent
  \begin{enumerate}
    \item [α)] $Δ=48-4λ$
    \item [β)] $48-4λ>0\implies λ<12$
    \item [γ)]
    \begin{enumerate}
      \item [(i)] $S=6$ και $P=λ-3$. Αλλά $3<λ\implies λ-3>0$, Άρα $S$ και $P$ θετικά και έχουμε δύο ρίζες άνισες. Άρα θετικές.
      \item [(ii)] Το $μ$ είναι ανάμεσα στις θετικές ρίζες άρα θετικό. Το $κ$ είναι μικρότερο του μηδέν. Το $f(μ)$ είναι η τιμή του τριωνύμου για $x=μ$ άρα αφού $μ$ ανάμεσα στις ρίζες, $f(μ)<0$. Με ίδιο τρόπο $f(κ)>0$. Άρα συνολικά 2 πλην και 2 συν, και ως αποτέλεσμα θα είναι θετικό.
    \end{enumerate}
  \end{enumerate}

\end{document}
